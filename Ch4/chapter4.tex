\graphicspath{ {./Ch4/}  } 
\DeclareGraphicsExtensions{.png,.pdf,.jpg}

%%%%%%%%%%%%%%%%%%%%%%%%%%%%%%%%%%%%%%%%%%%%%%%%%%%%%%%%%%%%%%%%%%%%%%%%%%%%%%%%%%%%%%%%%%
%
%  CITE GROUPS  %
%%%%%%%%%%%%%%%%%

% % JHU database [76,77]
% \newcommand{\jhuDNS}{jhudnsdata, li2008public}

%%%%%%%%%%%%%%%%%
%


%%%%%%%%%%%%%%%%%%%%%%%%%%%%%%%%%%%%%%%%%%%%%%%%%%%%%%%%%%%%%%%%%%%%%%%%%%%%%%%%%%%%%%%%%%
\chapter{Implementation Assessment Methodology}
\label{ch:4}

$A priori$ tests are used here to assess the relative accuracy and computational cost of various implementations of autonomic closure. Although $a priori$ tests alone cannot demonstrate that a subgrid closure will provide stable computations when implemented in an LES code, such tests are the most direct way of assessing the accuracy of any closure approach in representing subgrid stress fields $\tau_{ij}(\mathbf{x},t)$  and subgrid production fields $P(\mathbf{x},t)$. The present assessments are thus essential for understanding the accuracy and computational cost of various implementations of autonomic closure, to enable later $a posteriori$ tests of implementations that are found here to be both accurate and efficient. 

We also develop metrics to assess the accuracy of any implementation of autonomic closure. As noted in Section I.B, in highly intermittent fields such as the subgrid production $P(\mathbf{x},t)$ it may not be possible to have precise point-by-point agreement between the true field and that resulting from a subgrid stress closure. However a closure should (i) produce $P(\mathbf{x},t)$  fields that are structurally similar to the corresponding true production fields, with large values of forward and backward scatter concentrated in regions at the same locations and of the same size and shape, (ii) produce similar statistical distributions of positive and negative values as in the true  $P(\mathbf{x},t)$ fields, and (iii) produce $\langle P(\mathbf{x},t) \rangle \geq \varepsilon$, where  $\varepsilon$ is the true subgrid production rate. Therefore, in addition to comparing average values and statistical distributions of $\tau_{ij}$ and $P$ from autonomic closure to corresponding results from the true subgrid stress and production fields, we develop metrics to quantitatively compare the scale-dependent support-density fields on which large positive and negative subgrid production values are concentrated in the true $P(\mathbf{x},t)$  fields and in  $P^{F}(\mathbf{x},t)$ from autonomic closure. These provide sensitive measures of how well any closure for $\tau_{ij}$  represents the detailed spatial structure of regions in which large forward and backward scatter occur in the subgrid production field.
 
%%%%%%%%%%%%%%%%%%%%%%%%%%%%%%%%%%%%%%%%%%%%%%%%%%%%%%%%%%%%%%%%%%%%%%%%%%%%%%%%%%%%%%%%%%
\section{Pseudo-LES fields, test fields, and resulting stress fields} 

Direct numerical simulation (DNS) data from a $1024^3$ simulation of homogeneous isotropic turbulence at $ Re_{\lambda} = 433$ from the Johns Hopkins Turbulence Database \cite{\jhuDNS} were used for these $a priori$ assessments. Velocity $u_i(\mathbf{x},t)$  and pressure  $p(\mathbf{x},t)$ fields were first projected on a $256^3$ grid having regular grid spacing $\Delta$, on which all results are displayed. A spectrally sharp filter with cutoff at $k_{\widetilde{\Delta}} = 40$, well within the inertial range, was applied to produce pseudo-LES fields $\widetilde{u_i}(\mathbf{x},t)$ and $\widetilde{p_i}(\mathbf{x},t)$   on the $256^3$ display grid. This grid accommodates wavenumbers up to  $k_{\Delta} = 128$, allowing the nonlinear product fields $\widetilde{u}_i \widetilde{u}_j (\mathbf{x},t)$, which have wavenumbers up to $2k_{\widetilde{\Delta}}= 80$, to be represented without aliasing. The true subgrid stress fields $\tau_{ij}(\mathbf{x},t) = \widetilde{u_iu_j} - \widetilde{u}_i \widetilde{u}_j$  were then constructed on the same $256^3$ display grid. A second spectrally sharp test filter with cutoff at  $k_{\widetilde{\Delta}}= 20$, providing test-to-LES filter ratio  $\alpha \equiv \widehat{\Delta}/\widetilde{\Delta}= 2$, produced test fields $\widehat{\widetilde{u_i}}(\mathbf{x},t)$   and  $\widehat{\widetilde{p}}(\mathbf{x},t)$ on the same $256^3$ display grid. This grid allows the product fields $\widehat{\widetilde{u_i}}\widehat{\widetilde{u_j}}(\mathbf{x},t)$, which contain wavenumbers up to $2k_{\widetilde{\Delta}}= 40$, to also be represented without aliasing. The test stress fields  $T_{ij}(\mathbf{x},t) = \widehat{\widetilde{u_iu_j}} - \widehat{\widetilde{u}}_i \widehat{\widetilde{u}}_j$  were then constructed on the $256^3$ display grid.

The test stress fields $T_{ij}(\mathbf{x},t)$  and the test-filtered velocity  $\widehat{\widetilde{u_i}}(\mathbf{x},t)$  and pressure  $\widehat{\widetilde{p}}(\mathbf{x},t)$  fields are the only inputs needed for the autonomic closure methodology in Section II.A. To determine  $\tau_{ij}$ at any point $\mathbf{x}$ on the display grid, since  $\widehat{\Delta}/\Delta = k_{\Delta}/k_{\widehat{\Delta}} = 128/20 = 6$   the local test-scale grid consisted of every sixth point along each direction on the $256^3$ display grid within the specified bounding box volume centered on $\mathbf{x}$. Due to periodicity of the underlying DNS data, the largest possible bounding box size $n^3$ spans the entire test field domain, thus  $n^3 = 256^3/6^3 = 44^3$. At the other extreme, the smallest bounding box accommodates just one training point centered on the  $3 \times 3 \times 3$ test-grid stencil $\widehat{\mathbf{S}}$, thus $n^3 = 3^3$. For each of $M$ equally spaced training points in the bounding box, the values of $\widehat{\widetilde{u_i}}$  and  $\widehat{\widetilde{p}}$ at each point on the test-grid stencil $\widehat{\mathbf{S}}$  centered on that training point provide the inputs for one row of the matrix $\widehat{\mathbf{V}}$  in (8), and the test stress value  $T_{ij}$ at that training point is the corresponding component of the  $\mathbf{T}_{ij}$ vector. For each $ij$ pair the resulting $M \times N$  system in (8) is then solved via (9) to determine the coefficients $\mathbf{h}_{ij}$  at $\mathbf{x}$, which then provide $\tau_{ij}^{F}$   via (7) from the values of $\widetilde{\mathbf{u}}$  and $\widetilde{p}$   at each point on the LES-grid stencil $\widetilde{\mathbf{S}}$  centered on $\mathbf{x}$.  

%
%    FIGURE     % 
%%%%%%%%%%%%%%%%%
\begin{figure}
	\begin{center}
	\includegraphics[width=\maxwidth{4in}]{Fig3_y213-03.eps}
	\caption{ Typical subgrid production field $P(\mathbf{x},t)$, showing (a) true field, (b) result from autonomic closure, (c) true support-density field $G(\mathbf{x},t)$ , and (d) corresponding support-density field from autonomic closure. }
	\label{F:3}
	\end{center}
\end{figure}
%%%%%%%%%%%%%%%%%
%
%

%%%%%%%%%%%%%%%%%%%%%%%%%%%%%%%%%%%%%%%%%%%%%%%%%%%%%%%%%%%%%%%%%%%%%%%%%%%%%%%%%%%%%%%%%%
\section{Statistical comparisons of $\tau_{ij}(\mathbf{x},t)$  and  $P(\mathbf{x},t)$ } 

For each implementation in Table 1 the resulting subgrid stress fields  $\tau_{ij}^{F}(\mathbf{x},t)$   and production fields $P^{F}(\mathbf{x},t) \equiv -\tau_{ij}^{F} \widetilde{S}_{ij}$ from autonomic closure are compared with the corresponding true stress and production fields  $\tau_{ij}^{F}(\mathbf{x},t)$  and $P(\mathbf{x},t) \equiv -\tau_{ij} \widetilde{S}_{ij}$  . These comparisons include probability densities of stresses and production to assess if the implementation produces similar distributions of positive and negative values as in the true $\tau_{ij}^{F}(\mathbf{x},t)$  and  $P(\mathbf{x},t)$ fields. We also compare the average subgrid production $\langle P^F \rangle$  from each implementation to the true value $\varepsilon$.

%
%    FIGURE     % 
%%%%%%%%%%%%%%%%%
\begin{figure}
	\begin{center}
	\includegraphics[width=\maxwidth{\textwidth}]{Fig4.eps}
	\caption{ Subgrid production fields $P(\mathbf{x},t)$ (leftmost column) and associated support-density fields $G(\mathbf{x},t)$ filtered at successive scale ratios $\Delta_{\Gamma}/\widetilde{\Delta}$, showing (top row) true production field and (bottom row) result from autonomic closure. }
	\label{F:4}
	\end{center}
\end{figure}
%%%%%%%%%%%%%%%%%
%
%
However, probability densities only give the distributions of magnitudes in these fields, but provide no information about the spatial structure of the true fields and those from autonomic closure. To determine whether large magnitudes of forward and backward scatter in the subgrid production fields $P^{F}(\mathbf{x},t)$  are concentrated in regions at the same spatial locations and of the same size and shape as in the true $P(\mathbf{x},t)$  fields, support-density fields for the subgrid production are obtained as described in Sections III.C and III.D. These are then used to obtain the scale-dependent metrics $M_1$  and $M_2$  described in Section III.E that quantify how closely spatial structures in which large forward and backward scatter are concentrated in $P^{F}(\mathbf{x},t)$  from autonomic closure compare with those in the true subgrid production fields $P(\mathbf{x},t)$ .


%%%%%%%%%%%%%%%%%%%%%%%%%%%%%%%%%%%%%%%%%%%%%%%%%%%%%%%%%%%%%%%%%%%%%%%%%%%%%%%%%%%%%%%%%%
\section{Support fields for $P(\mathbf{x},t)$} 

Figures 3a,b show a typical comparison of the subgrid production field $P^{F}(\mathbf{x},t)$ from autonomic closure with the corresponding true field $P(\mathbf{x},t)$. Probability densities of subgrid stress and production are used in Section IV to compare magnitudes in these fields. However Fig. 3 also shows structural similarities in  $P(\mathbf{x},t)$ and $P^{F}(\mathbf{x},t)$, even in many of the detailed features of these fields, including regions where large positive and negative values of $P$ are clustered. Of central importance, large magnitudes in $P^{F}(\mathbf{x},t)$  in Fig. 3b are clustered in regions at about the same locations and of about the same size and shape as those in $P(\mathbf{x},t)$ in Fig. 3a. 

However, despite the clear similarities in Figs. 3a,b in the location, size, and shape of regions where large magnitudes of subgrid production are concentrated, the precise point-by-point $rms$ differences between  $P(\mathbf{x},t)$ and $P^F(\mathbf{x},t)$, scaled by $P'_{rms}$ , are nevertheless of $O(1)$. This is due to the strong intermittency in these fields, which leads to large $rms$ differences even if the two fields appear nearly identical at all but the smallest scales. Metrics other than the simple $rms$ difference are needed that quantify the spatial structure of these fields, focusing on the regions in which large production magnitudes are concentrated. Such structural metrics should not primarily address the production values themselves, since these are already compared in the probability densities of $P$ and $P^F$  , but should focus on the structure of the $spatial support$ on which large production values are concentrated.  

The support of a field is the subset of the domain on which the field values are non-zero. Here we define the support on which large magnitudes of the subgrid production fields  $P(\mathbf{x},t)$ and $P^F(\mathbf{x},t)$ are concentrated, by thresholding the absolute value of each field at a fixed fraction  $\gamma$ of $P'_{rms}$ . This defines $\Sigma(\mathbf{x},t)$  as either zero or one, depending on whether the absolute value of the subgrid production is below or above the threshold. Points where  $\Sigma = 1$ are on the support of large production magnitudes and those where  $\Sigma = 0$  are off the support. Thresholding $\Sigma$  at $\gamma = 0.75$   provides clear identification of the sensible support on which large magnitudes of the subgrid production fields are concentrated. 

%%%%%%%%%%%%%%%%%%%%%%%%%%%%%%%%%%%%%%%%%%%%%%%%%%%%%%%%%%%%%%%%%%%%%%%%%%%%%%%%%%%%%%%%%%
\section{Support-density fields $G(\mathbf{x},t)$ }

The support for each of the subgrid production fields can be separated into different scales to allow scale-by-scale comparisons of  $P(\mathbf{x},t)$ and $P^F(\mathbf{x},t)$   . From the support  $\Sigma(\mathbf{x},t)$  we define the corresponding support-density field  $G(\mathbf{x},t)$  as 
%
%    EQUATION   %  
%%%%%%%%%%%%%%%%%
\begin{equation}
	\label{E:10}
	G(\mathbf{x},t) 
	\equiv 
	\int_{V} \Sigma(\mathbf{x'},t)\Gamma_{\Delta}(|\mathbf{x} - \mathbf{x'}|) \mathbf{x'}, 
\end{equation}
%%%%%%%%%%%%%%%%%
%
%                
where $\Gamma_{\Delta}(|\mathbf{x} - \mathbf{x'}|)$  is a convolution filter kernel with filter length scale $\Delta_{\Gamma}$ . We use standard Gaussian filters for $\Gamma_{\Delta}$  in (\ref{E:10}). Whereas the support $\Sigma(\mathbf{x},t)$  is a discontinuous binary-valued field, the support-density $G(\mathbf{x},t)$  is a continuous real-valued field to which standard error measures can be applied. Figures 3c,d show the support-density fields $G(\mathbf{x},t)$  and  $G^{F}(\mathbf{x},t)$ corresponding to   $P(\mathbf{x},t)$  and  $P^{F}(\mathbf{x},t)$   in Figs. 3a,b. It is apparent that these $G$ fields accurately identify the locations, sizes, and shapes of the regions in which large subgrid production values are concentrated. 
Successive filter length scales  $\Delta_{\Gamma}$ in (\ref{E:10}) allow scale-dependent structure in the support-density fields to be determined. Comparisons at the same filter scale between true production support-density fields and those obtained from the closure allow quantitative assessment of scale-by-scale agreement in these fields. Figure 4 shows an example of such scale-dependent comparisons of the support-density fields   $G(\mathbf{x},t)$  and  $G^{F}(\mathbf{x},t)$   for the subgrid production fields   $P(\mathbf{x},t)$  and  $P^{F}(\mathbf{x},t)$   in Fig. 3 at successive scale ratios $\Delta_{\Gamma}/\widetilde{\Delta}$ . 

%%%%%%%%%%%%%%%%%%%%%%%%%%%%%%%%%%%%%%%%%%%%%%%%%%%%%%%%%%%%%%%%%%%%%%%%%%%%%%%%%%%%%%%%%%
\section{Support-density metrics  $M_1$ and $M_2$ }

From support-density fields   $G(\mathbf{x},t)$  and  $G^{F}(\mathbf{x},t)$   for   $P(\mathbf{x},t)$  and  $P^{F}(\mathbf{x},t)$    as in Fig. 4, we use two metrics to quantitatively compare their spatial structure. At any scale-ratio  $\Delta_{\Gamma}/\widetilde{\Delta}$ we define
%
%    EQUATION   %  
%%%%%%%%%%%%%%%%%
\begin{equation}
	\label{E:11}
	M_1 \equiv \frac{\langle G'(\mathbf{x},t) G'^{F}(\mathbf{x},t)  \rangle_{V}}
	{\langle G'(\mathbf{x},t) \rangle_{V}^{1/2}  
	 \langle G'^{F}(\mathbf{x},t) \rangle_{V}^{1/2}} 
	\quad \textnormal{and} \quad
	 M_2 \equiv \sqrt{\frac{\langle \big[ G(\mathbf{x},t) G^{F}(\mathbf{x},t) \big]^{2} \rangle_{V}}
	{\langle \big[ G(\mathbf{x},t) \big]^2 \rangle_{V}}}
\end{equation}
%%%%%%%%%%%%%%%%%
%
%                
where   is the correlation between the support-density fields $G(\mathbf{x},t)$  and  $G^{F}(\mathbf{x},t)$, with $G' \equiv G - \langle G \rangle_V$, and $M_2$  is the normalized $rms$ difference between  $G(\mathbf{x},t)$  and  $G^{F}(\mathbf{x},t)$. The volume averages are over the entire domain. Note $M_1 \rightarrow 1$  as the two support-density fields become perfectly correlated, and  $M_2 \rightarrow 0$  as the two support-density fields become identical. The variations in  $M_1$ and $M_2$  with scale-ratio $\Delta_{\Gamma}/\widetilde{\Delta}$  allow quantitative comparisons of the spatial support-densities on which the true production field  $P(\mathbf{x},t)$  and its associated $P^{F}(\mathbf{x},t)$ from the $\tau_{ij}$  closure are concentrated. 

%%%%%%%%%%%%%%%%%%%%%%%%%%%%%%%%%%%%%%%%%%%%%%%%%%%%%%%%%%%%%%%%%%%%%%%%%%%%%%%%%%%%%%%%%%
\section{Computational cost}

We also evaluate the computational time for each implementation of autonomic closure. Only the time associated with the subgrid stress evaluation is considered; all other factors are independent of the choice of closure. In autonomic closure there are three contributors to the computational time:  
%
%    NUM LIST   % 
%%%%%%%%%%%%%%%%%
\begin{enumerate} 
	%
	\item{The time to build the $\widehat{\mathbf{V}}$  matrix from the $N$ degrees of freedom in $F_{ij}$  for each of the $M$ training points; since  $\widehat{\mathbf{V}}$ is an  $M \times N$ matrix, the matrix build time should scale as $M \cdot N$.}
	%
	\item{The time to solve (9) for the coefficients $\mathbf{h}_{ij}$; since $\widehat{\mathbf{V}}^{T} \widehat{\mathbf{V}}$   is an $N \times N$  matrix, the time required for the damped least-squares solution is expected to scale as $N^2$ .}
	%
	\item{The time to construct  $\tau_{ij}$ from  $\mathbf{h}_{ij}$ and  $\widetilde{\mathbf{V}}$ via (7); since  $\mathbf{h}_{ij}$ and  $\widetilde{\mathbf{V}}$  are respectively $N \times 1$  and $1 \times N$  arrays, this simply involves $N$ multiplications and thus scales as $N$.}
	%
\end{enumerate}
%%%%%%%%%%%%%%%%%
%
%
The last of these should be negligible. In the second, due to the operations involved the scaling prefactor might be expected to be larger than for the first, which would make this the limiting step. The overall computational time would then scale to leading order as $N^2$, and thus be independent of the number $M$ of training points. This might suggest using all training points available in the bounding box, but Section II.B notes that there is diminishing benefit from increasing $M$ once the training point spacing  $(V_B/M)^{1/3}$ has become so small that little additional independent training information is being gained.  

