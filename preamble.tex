%%%%%%%%%%%%%%%%%%%%%%%%%%%%%%%%%%%%%%%%%%%%%%%%%%%%%%%%%%%%%%%%%%%%%%%%%%%%%%
% ASU Dissertation Template
%%%%%%%%%%%%%%%%%%%%%%%%%%%%%%%%%%%%%%%%%%%%%%%%%%%%%%%%%%%%%%%%%%%%%%%%%%%%%%
% Copyright 2015 Robert W. Kutter (robert@kutterconsulting.com)
% 
%   See also: http://kutterconsulting.com
% 
% For guidance on using this file, see the README. 
%  
%%%%%%%%%%%%%%%%%%%%%%%%%%%%%%%%%%%%%%%%%%%%%%%%%%%%%%%%%%%%%%%%%%%%%%%%%%%%%%
% Preamble
%%%%%%%%%%%%%%%%%%%%%%%%%%%%%%%%%%%%%%%%%%%%%%%%%%%%%%%%%%%%%%%%%%%%%%%%%%%%%%
\newcommand*{\pointsize}{12pt}          %<Set the font size; make sure the size is correct
                                        %   for the font you will use
\documentclass[letterpaper,             % Use US letter-size paper
               oneside,                 % No verso and recto differences
               \pointsize]              % Uses the font size defined above
               {memoir}
\renewcommand{\cleardoublepage}%        % \cleardoublepage will create entirely blank 
  {\clearpage}%                         %   pages depending on settings (e.g., usually 
                                        %   before start of \mainmatter); redefine it here
                                        %   so that no entirely blank pages are created
                                        %   automatically
%%%%%%%%%%%%%%%%%%%%%%%%%%%%%%%%%%%%%%%
% (Some) Packages
%%%%%%%%%%%%%%%%%%%%%%%%%%%%%%%%%%%%%%%
\usepackage{graphicx}                   % For importing image files
\usepackage{etoolbox}                   % For advanced commands throughout preamble
\usepackage{microtype}                  %~Improves kerning and protrusion (optional); 
                                        %   See here for an introduction: 
                                        %   http://www.khirevich.com/latex/microtype/

\providetoggle{usemicrotype}            % TRUE = microtype is being used
\makeatletter                           %   (Used to turn of microtype protrusion in the 
\@ifpackageloaded{microtype}%           %   table of contents.)
  {\settoggle{usemicrotype}{true}}%
  {\settoggle{usemicrotype}{false}}
\makeatother
\usepackage{changepage}                 % For changing page layout (e.g., margins) in the 
                                        %   middle of the document
\usepackage{calc}					              % Calculate text widths; used in page layout 
                                        %   changes
%%%%%%%%%%%%%%%%%%%%%%%%%%%%%%%%%%%%%%%
% Title page, input
%%%%%%%%%%%%%%%%%%%%%%%%%%%%%%%%%%%%%%%
\listadd{\titlelines}%
  {Autonomic Closure for Turbulent Flows}%<Enter the title of the dissertation
%\listadd{\titlelines}%
%  {Second Line of the Title}            % If you want to 
                                        %   split the title across lines,
                                        %   use another \listadd command for the 
                                        %   second line
\newcommand*\Author{Abhinav Kshitij}    %<Enter your name; must match official transcript
\newcommand*{\documentname}%
  {Dissertation}                        %<Enter the type of document (capitalized)
\newcommand*{\degreename}
  {Doctor of Philosophy}                %<Enter the type of degree (capitalized)
\newcommand*\defdate{December 2018}     %<Give month (written out fully) and year of 
                                        %   the oral defense
\listadd{\committeechair}{Prof. Werner J.A. Dahm} %<Enter committee chair name; use \listadd for
\newcommand*{\chairlabel}{Chair}        
\listadd{\committeemember}{Dr. Marcus Herrmann} 
\listadd{\committeemember}{Dr. Hans Mittelmann} 
\listadd{\committeemember}{Dr. Peter Hamlington} 
\listadd{\committeemember}{Dr. Yulia Peet} 

\newcommand*{\gradmonth}{December}         %<Enter the graduation date; month can only be: 
                                        %  May, August, or December
\newcommand*{\gradyear}{2018}           %<Enter the graduation year, e.g. 2014

\listadd{\keywords}{Turbulence}          %<Enter keywords; use \listadd for
\listadd{\keywords}{Combustion}          %   additional names (up to 6)
\listadd{\keywords}{Machine learning}    %   additional names (up to 6)

\newcommand*{\graddate}{\gradmonth%     % Compose full graduation date
  \space\gradyear}    

%%%%%%%%%%%%%%%%%%%%%%%%%%%%%%%%%%%%%%%
% Page layout
%%%%%%%%%%%%%%%%%%%%%%%%%%%%%%%%%%%%%%%
\settrimmedsize{\stockheight}%          % Specifies \paperheight and \paperwidth
  {\stockwidth}{*} 
\settrims{0pt}{0pt}                     % Set location of page in relation to the stock. 
                                        % Paper and stock size are equivalent, 
                                        % so both \trimtop and \trimedge are set to 0pt
\newlength{\forfootskip}
\setlength{\forfootskip}%
  {3\baselineskip}
\newlength{\textblockheight}            % Calculate height of text block to leave room
\setlength{\textblockheight}{9.0in}     %   for footers, keeping page numbers outside
\addtolength{\textblockheight}%         %   the 1in vertical margins
  {-\forfootskip}
\settypeblocksize{\textblockheight}%    % Calculated by 1.0in vertical margins and 
  {*}{*}                                %   letting margins set the width of the typeblock                      
\setulmargins{1.0in}{*}{*}              % Set upper margin (\uppermargin, not \topmargin); 
                                        %   calculate the bottom margin
\setlrmarginsandblock{1.25in}{1.25in}{*}%~Set margins and calculate width of typeblock
\setheaderspaces{*}{0.5\baselineskip}{*}% Arguments: '\headdrop', '\headsep', and/or ratio
                                        %   Note: This is only used in the list of 
                                        %   contents sections
\setheadfoot{\baselineskip}%            % Set '\headheight' and '\footskip'
  {\forfootskip}
\checkandfixthelayout                   % Required by memoir package after setting layout
\settypeoutlayoutunit{in}               % Write layout dimensions to log file in inches

%%%%%%%%%%%%%%%%%%%%%%%%%%%%%%%%%%%%%%%
% Fonts
%%%%%%%%%%%%%%%%%%%%%%%%%%%%%%%%%%%%%%%
\usepackage[T1]{fontenc}                % Standard option to handle, e.g., accented 
                                        %   characters like 'ö' better
\usepackage{amssymb,mathtools}          % For AMS-LaTeX, see here for more: 
                                        %     http://www.ams.org/publications/authors/tex/amslatex
                                        % ('mathtools' loads and extends 'amsmath')
\usepackage{ifxetex,ifluatex}           % Can check if XeTeX or LuaTeX was used to typeset
\usepackage{fixltx2e}                   % Provides \textsubscript
\IfFileExists{upquote.sty}%             % Use upquote if available, for 
  {\usepackage{upquote}}{}              %   straight quotes in verbatim environments
                                        
% Load fonts depending on the 
%   typesetting engine
\ifnum 0\ifxetex 1\fi\ifluatex 1\fi=0   % If pdftex
  \usepackage[utf8]{inputenc}           %   'utf8' should match the encoding of this file
                                        %
                                        %~Set up your font in pdftex here
                                        %
\else 									% If xetex or luatex 
  \ifxetex                              % If xetex
    \usepackage{mathspec}               % Matches non-math open-type font to math 
                                        %   open-type font (use 'mathspec' if you want to
                                        %   write math in unicode)
    \usepackage{xunicode}               % Convert LaTeX character macros to unicode
  \else                                 % If luatex\usepackage{fontspec}
    \usepackage{fontspec}               % Use fontspec for (open type) font selection
  \fi
  \defaultfontfeatures{Mapping=tex-text,% Font spec setting
    Scale=MatchLowercase} 
  \newcommand*{\euro}{€}
  \setmainfont{Times New Roman}         %<Set the main font; make sure the font is correct
                                        %   for the font size (See ASU Style Guide)
%  \setmathfont(Digits,Latin,Greek)%     %~Uncomment two lines to set a font for math%
%    {MATHFONT} 
\fi

%%%%%%%%%%%%%%%%%%%%%%%%%%%%%%%%%%%%%%%
% Line spacing
%%%%%%%%%%%%%%%%%%%%%%%%%%%%%%%%%%%%%%%
\DoubleSpacing                         % True double spacing
\BeforeBeginEnvironment{quote}         % Memoir leaves most special material 
  {\SingleSpacing}                     %   single spaced, but makes block quotes 
\AfterEndEnvironment{quote}%           %   double-spaced; fix to follow ASU style guide
  {\vspace{-\baselineskip} %
  \DoubleSpacing}
\BeforeBeginEnvironment{quotation}%
  {\SingleSpacing}
\AfterEndEnvironment{quotation}%
  {\vspace{-\baselineskip} %
  \DoubleSpacing}

\setlength{\footnotesep}{\baselineskip}% Double space *between* footnotes
\renewcommand*{\footnoterule}{%        % Redefine footnoterule so that initial footnote
  \kern-3pt%                           %   still appears right under the rule (changing
  \hrule width 0.4\columnwidth         %   \footnotesep also changes the space between the
  \kern 2.6pt                          %   rule and the first footnote
  \vspace{-0.5\baselineskip}           % (Here is the vertical space adjustment)
  }

\usepackage{enumitem}                  % Control spacing in enumerate environment
\setlist{noitemsep}                    % Remove extra vertical spacing between items in lists
                                       % \setlist{nosep} to leave no space around whole list

%%%%%%%%%%%%%%%%%%%%%%%%%%%%%%%%%%%%%%%
% Page numbering
%%%%%%%%%%%%%%%%%%%%%%%%%%%%%%%%%%%%%%%
\makepagestyle{ASU}
  \makeevenfoot{ASU}{}{\thepage}{}
  \makeoddfoot{ASU}{}{\thepage}{}

%%%%%%%%%%%%%%%%%%%%%%%%%%%%%%%%%%%%%%%
% Title page, formatting
%%%%%%%%%%%%%%%%%%%%%%%%%%%%%%%%%%%%%%%
\newlength{\savedfootskip}
\setlength{\savedfootskip}{\footskip}
\newcommand{\titlepagesetup}{%          % Page layout for title page
  \changepage%                          % Adjustment to page dimensions: 
    {\savedfootskip}%                   %   text height
    {}%                                 %   text width
    {}%                                 %   even-side margin
    {}%                                 %   odd-side margin
    {}%                                 %   column sep.
    {}%                                 %   topmargin
    {}%                                 %   headheight
    {}%                                 %   headsep
    {-\savedfootskip}%                  %   footskip
}

\newcommand{\closetitlepagesetup}{%     % Undo set up for title page
  \changepage{-\savedfootskip}{}{}{}{}%
    {}{}{}{\savedfootskip}%
}

\makeatletter                           % Do not modify this section; Enter info above
\newcommand*{\titlepageASU}{
  \titlepagesetup
  \clearpage
  \begin{center}
  \SingleSpacing
  \thispagestyle{empty}
    \renewcommand*{\do}[1]{##1 \\[\baselineskip]} 
    \dolistloop{\titlelines}
    by \\[\baselineskip]
    \Author \\[4\baselineskip]
    A \documentname~Presented in Partial Fulfillment \\
    of the Requirements for the Degree \\
    \degreename \\
    \vfill                              % Vertically center the portion below
    Approved \defdate~by the \\
    Graduate Supervisory Committee: \\[\baselineskip]
    \renewcommand*{\do}[1]{##1, \chairlabel \\} 
    \dolistloop{\committeechair} 
    \renewcommand*{\do}[1]{##1 \\} 
    \dolistloop{\committeemember}
    \vfill                              % Vertically center the portion above
    ARIZONA STATE UNIVERSITY \\[\baselineskip]
    \graddate
  \end{center}
  \clearpage
  \closetitlepagesetup
}
\makeatother

%%%%%%%%%%%%%%%%%%%%%%%%%%%%%%%%%%%%%%%
% Heading styles
%%%%%%%%%%%%%%%%%%%%%%%%%%%%%%%%%%%%%%%
% Note: memoir also has \book and \part commands; do not use these
\makechapterstyle{ASU}{%                % Define chapter heading style
  \renewcommand*{\chapterheadstart}{}   % Chapter title flush with top margin
  \renewcommand*{\chapnamefont}%        % Set font for 'Chapter' or 'Appendix' 
    {\normalfont}
  \renewcommand*{\chapnumfont}%         % Set font for number in chapter headings
    {\normalfont}
  \renewcommand*{\afterchapternum}%     % Insert a double line break after 
    {\\[\baselineskip]}                 %   chapter number
  \renewcommand*{\chaptitlefont}%       % Set font for chapter title name
    {\normalfont}
  \setlength{\afterchapskip}{0pt}       % Set vertical space between chapter title and
                                        %   first paragraph; equivalent to one line break
                                        %   (vertical space = \afterchapskip + \baselineskip)
                                        % Note: This \afterchapskip value is only used in 
                                        %   front matter
  \renewcommand*{\printchapternum}{%    % Center justify chapter number
    \centering \chapnumfont %
    \thechapter}                                        
  \renewcommand*{\printchaptertitle}[1]%% Center justify 
    {\expandafter\centering %           %   \MakeUppercase has issues; see here for some 
    \expandafter\chaptitlefont %        %   details: https://tex.stackexchange.com/questions/35680/uppercase-in-newcommand
    \expandafter\MakeUppercase %        %   Accented characters and some fonts may not
    \expandafter{##1}}                  %   uppercase correctly; if that happens, just 
                                        %   type the chapter title in uppercase
}

\setsecnumdepth{all}                    %~Enter the levels that you want to have numbered
                                        %   (Default is to number all [5 levels deep].)

\newcommand{\divisionbeforeskip}%       % Create default formatting for headings
  {\baselineskip}
\newcommand{\divisionindent}%
  {0.5em}
\newcommand{\divisionfont}{\normalfont} % Font must be \normalfont
\newcommand{\divisionafterskip}%
  {\baselineskip}

\setbeforesecskip{\divisionbeforeskip}  % Apply default formatting to all heading levels
\setsecindent{\divisionindent}          % Note: If you change \setsecnumdepth above, you 
\setsecheadstyle{\divisionfont}         %   will need to set the indent for all lower 
\setaftersecskip{\divisionafterskip}    %   levels to '0pt'; otherwise, they will be 
                                        %   preceded by unnecessary space
\setbeforesubsecskip{\divisionbeforeskip}
\setsubsecindent{\divisionindent}
\setsubsecheadstyle{\divisionfont}
\setaftersubsecskip{\divisionafterskip}

\setbeforesubsubsecskip{\divisionbeforeskip}
\setsubsubsecindent{\divisionindent}
\setsubsubsecheadstyle{\divisionfont}
\setaftersubsubsecskip{\divisionafterskip}

\setbeforeparaskip{\divisionbeforeskip}
\setparaindent{\divisionindent}
\setparaheadstyle{\divisionfont}
\setafterparaskip{\divisionafterskip}

\setbeforesubparaskip{\divisionbeforeskip}
\setsubparaindent{\divisionindent}
\setsubparaheadstyle{\divisionfont}
\setaftersubparaskip{\divisionafterskip}

%%%%%%%%%%%%%%%%%%%%%%%%%%%%%%%%%%%%%%%
% Paragraph formatting
%%%%%%%%%%%%%%%%%%%%%%%%%%%%%%%%%%%%%%%
%\sloppybottom                          % Reduce the chances of widows
\raggedbottom                           % Loosens vertical spacing requirements, so 
                                        %   \sloppybottom doesn't make pages look bad; 
                                        %   it also prevents large gaps in the middle of
                                        %   pages and pushes them to the bottom of pages
\indentafterchapter                     % Overrides the default which is not to indent 
                                        %   the first paragraph in a chapter, but it 
                                        %   looks odd in some places to not indent
                                        %   paragraphs

%%% List Titles %%%
\renewcommand{\contentsname}%           % Set heading for each list
  {Table of Contents}%                  %   Formatted as chapter headings by default, so
\renewcommand{\listtablename}%          %   no additional heading formatting is needed
  {List of Tables}
\renewcommand{\listfigurename}% 
  {List of Figures}

%%% Depth %%%
\settocdepth{subparagraph}              % Include 5 levels deep (all levels) in TOC

%%% Fonts %%%
\makeatletter% 
\patchcmd{\l@part}%                     % Patch the command that writes part-level entries
    {\cftpartfont {#1}}%                %   to the table of contents, so they are in 
    {\normalfont \texorpdfstring{%      %   'normalfont' and uppercase
      \uppercase{#1}}{{#1}} }%
    {\typeout{Success: Patch %
      'l@part' to uppercase %
      part-level headings in the %
      table of contents.}}%
    {\typeout{Fail: Patch %
      'l@part' to uppercase % 
      part-level headings in the %
      table of contents.}}%
\makeatother%

\makeatletter% 
\patchcmd{\l@chapapp}%                  % Patch the command that writes chapter-level 
    {\cftchapterfont {#1}}%             %   entries to the table of contents, so they are 
    {\normalfont \texorpdfstring{%      %   in 'normalfont' and uppercase
      \uppercase{#1}}{{#1}} }%
    {\typeout{Success: Patch %
      'l@chapapp' to uppercase %
      part-level headings in the %
      table of contents.}}%
    {\typeout{Fail: Patch %
      'l@chapapp' to uppercase %
      part-level headings in the %
      table of contents.}}%
\makeatother%

% If not using 'hyperref', use the following commands to adjust 'part' and 'chapter' 
%   level headings in the TOC
%\renewcommand*{\cftpartfont}%         % Uppercase 'part' and 'chapter' headings
%  {\normalfont\MakeTextUppercase}     % Note: Sending \MakeTextUppercase to the TOC 
%\renewcommand*{\cftchapterfont}%      %   conflicts with hyperref and breaks it!
%  {\normalfont\MakeTextUppercase}%    

%\usepackage{titlecaps}                  % Set up headline style for captions in the 
                                        %   lists of tables and figures
                                        % Note: ASU style guide does not provide 
                                        %   comprehensive guidelines for headlines, so 
                                        %   Chicago style for headline style is used
                                        % Note: Last word in title is not explicitly 
                                        %   capitalized; in general, these settings are
                                        %   broadly correct, but captions should be 
                                        %   reviewed to ensure they are being capitalized
                                        %   properly
% \Resetlcwords
% \Addlcwords{a an the}                   % Leave articles lowercase
% \Addlcwords{and but for or nor}         % Leave conjunctions lowercase
% \Addlcwords{aboard about above across % % Leave all prepositions lowercase
%   after against along amid among anti % %   (This is a [non-exhaustive] list of common 
%   around as at before behind below %    %   one-word prepositions)
%   beneath beside besides between %
%   beyond but by concerning considering %
%   despite down during except excepting %
%   excluding following for from in %
%   inside into like minus near of off %
%   on onto opposite outside over past %
%   per plus regarding round save since %
%   than through to toward towards under %
%   underneath unlike until up upon %
%   versus vs via with within without}
% \Addlcwords{ according\space{to} %      % Leave two-word conjunctions lowercase
%   ahead\space{of} apart\space{from} %   %   (This is a [non-exhaustive] list of common 
%   as\space{for} as\space{of} %          %   two-word prepositions.) 
%   as\space{per} as\space{regards} %
%   aside\space{from} astern\space{of} %
%   back\space{to} because\space{of} %
%   close\space{to} due\space{to} %
%   except\space{for} far\space{from} %
%   in\space{to} inside\space{of} %
%   instead\space{of} left\space{of} %
%   near\space{to} next\space{to} %
%   on\space{to} opposite\space{of} %
%   opposite\space{to} out\space{from} %
%   out\space{of} outside\space{of} %
%   owing\space{to} prior\space{to} %
%   pursuant\space{to} rather\space{than} %
%   regardless\space{of} right\space{of} %
%   subsequent\space{to} such\space{as} %
%   thanks\space{to} that\space{of} %
%   up\space{to}} 
  
% \renewcommand{\cfttableaftersnumb}%     % Put table captions in List of Tables in title
%   {\titlecap}%                          %   case
% \renewcommand{\cftfigureaftersnumb}%    % Put table captions in List of Figures in title
%   {\titlecap}%                          %   case

% \newcommand{\macrocapwrap}[1]{%         % Use this macro to place other macros inside 
%   {\bgroup\bgroup{{#1}}\egroup\egroup}% %   captions, e.g., '\macrocapwrap{\ref{figure1}}'
% }%                                      % Note: Necessary due to the 'titlecaps' package
%                                         %   which modifies contents of captions

\renewcommand*{\cftpartpagefont}%       % Use normal font for all page numbers
  {\normalfont}
\renewcommand*{\cftchapterpagefont}%
  {\normalfont}
\renewcommand*{\cftsectionpagefont}%
  {\normalfont}
\renewcommand*{\cftsubsectionpagefont}%
  {\normalfont}
\renewcommand*{\cftsubsubsectionpagefont}%
  {\normalfont}
\renewcommand*{\cftsubsubsectionpagefont}%
  {\normalfont}
\renewcommand*{\cftparagraphpagefont}%
  {\normalfont}
\renewcommand*{\cftsubparagraphpagefont}%
  {\normalfont}
\renewcommand*{\cftfigurepagefont}%
  {\normalfont}
\renewcommand*{\cfttablepagefont}%
  {\normalfont}

\cftpagenumbersoff{part}                % Turn off page numbers for 'part's, which are 
                                        %   actually serving as headings within the TOC

%%% Vertical Space %%%
\setlength{\cftbeforepartskip}{0pt}     % Remove all additional vertical spacing so TOC
\setlength{\cftbeforechapterskip}{0pt}  %   is double spaced uniformly
\setlength{\cftbeforesectionskip}{0pt}
\setlength{\cftbeforesubsectionskip}{0pt}
\setlength{\cftbeforesubsubsectionskip}{0pt}
\setlength{\cftbeforeparagraphskip}{0pt}
\setlength{\cftbeforesubparagraphskip}{0pt}
\setlength{\cftbeforefigureskip}{0pt}
\setlength{\cftbeforetableskip}{0pt}

\renewcommand{\insertchapterspace}{%    % By default, extra vertical space (10pt) is 
  \addtocontents{lof}%                  %   inserted between tables and figures from 
    {\protect\addvspace{0pt}}%          %   different chapters; remove this extra space.
  \addtocontents{lot}%
    {\protect\addvspace{0pt}}%
}

%%% Horizontal Space %%%
\newlength{\levelindentincrement}       % Set indent to increase by the same amount for
\setlength{\levelindentincrement}{2em}  %   each level in the TOC; don't adjust figure
\newlength{\levelindent}                %   or table indents
\setlength{\levelindent}%
  {\levelindentincrement}
\setlength{\cftchapterindent}%
  {\levelindent}
\addtolength{\levelindent}%
  {\levelindentincrement}
\setlength{\cftsectionindent}%
  {\levelindent}
\addtolength{\levelindent}%
  {\levelindentincrement}
\setlength{\cftsubsectionindent}%
  {\levelindent}
\addtolength{\levelindent}%
  {\levelindentincrement}
\setlength{\cftsubsubsectionindent}%
  {\levelindent}
\addtolength{\levelindent}%
  {\levelindentincrement}
\setlength{\cftparagraphindent}%
  {\levelindent}
\addtolength{\levelindent}%
  {\levelindentincrement}
\setlength{\cftsubparagraphindent}%
  {\levelindent}
\addtolength{\levelindent}%
  {\levelindentincrement}

\setlength{\cftchapternumwidth}%        % Decrease space between number and heading for
  {0.85\cftchapternumwidth}             %   all heading levels
\setlength{\cftsectionnumwidth}%
  {0.85\cftsectionnumwidth}
\setlength{\cftsubsectionnumwidth}%
  {0.85\cftsubsectionnumwidth}
\setlength{\cftsubsubsectionnumwidth}%
  {0.85\cftsubsubsectionnumwidth}
\setlength{\cftparagraphnumwidth}%
  {0.85\cftparagraphnumwidth}
\setlength{\cftsubparagraphnumwidth}%
  {0.85\cftsubparagraphnumwidth}
\setlength{\cftfigurenumwidth}%         % Figure has the same 'level' as 'chapter' in the
  {\cftsectionnumwidth}                 %   figure list, so make the number spacing the 
                                        %   same as for chapters. Increased to 'section'.
\setlength{\cfttablenumwidth}%          % Table has the same 'level' as 'chapter' in the
  {\cftsectionnumwidth}                 %   table list, so make the number spacing the 
                                        %   same as for chapters. Increased to 'section'.


%%% Leaders/dots %%%
\renewcommand*{\cftdotsep}{1.7}         % Set distance between dots for all heading levels
\renewcommand*{\cftchapterleader}%      % Turn on dots for 'chapter' level
  {\normalfont\cftdotfill{\cftdotsep}}
\makeatletter                           % Bring leader dots over to page number (no gap)
  \renewcommand{\@pnumwidth}{1.55em}    %~Manually adjust
  \renewcommand{\@tocrmarg}{2.55em}
\makeatother

%%% Printing List Titles and Headers in Content Lists
% Table of Contents (TOC)
\copypagestyle{ASUtoc}{ASU}%            % Page style for regular page in TOC
  \makeevenhead{ASUtoc}%
    {\leftmark}{}{Page}
  \makeoddhead{ASUtoc}%
    {\leftmark}{}{Page}

\copypagestyle{ASUtocFirst}{ASU}%       % Custom page headers for first page of TOC 
  \makeevenhead{ASUtocFirst}%           %    (print out the title)
    {}%
    {\printchaptertitle{\contentsname}}%
    {} 
  \makeoddhead{ASUtocFirst}%
    {}%
    {\printchaptertitle{\contentsname}}%
    {}

\renewcommand{\tocheadstart}{}%         % Usually content list titles are printed like 
                                        %   chapter headings; empty that formatting 

\renewcommand{\printtoctitle}[1]{}%     % Don't print TOC title using default method; 
                                        %   it will be output in the header

\renewcommand{\aftertoctitle}{%         % On the first page of the TOC, print out the
  \thispagestyle{ASUtocFirst}%          %   TOC title using a custom page style and print 
  \hfill Page\par%                      %   the heading for the page below in the regular
  }%                                    %   textbox 
                                        % Note: Need '\par' before lists; see here: https://tex.stackexchange.com/questions/49882/yet-another-perhaps-a-missing-item-error

% List of Tables (LOT)
\copypagestyle{ASUlot}{ASU}%            % Page style for regular page in list of tables
  \makeevenhead{ASUlot}{Table}{}{Page}
  \makeoddhead{ASUlot}{Table}{}{Page}

\copypagestyle{ASUlotFirst}{ASU}%       % Custom page headers for first page of list of  
  \makeevenhead{ASUlotFirst}%           %   tables (print out the title)
    {}%
    {\printchaptertitle{\listtablename}}%
    {} 
  \makeoddhead{ASUlotFirst}%
    {}%
    {\printchaptertitle{\listtablename}}%
    {}

\renewcommand{\lotheadstart}{}%         % Usually content list titles are printed like 
                                        %   chapter headings; empty that formatting; 

\renewcommand{\printlottitle}[1]{}%     % Don't print LOT title using default method; 
                                        %   it will be output in the header

\renewcommand{\afterlottitle}{%         % On the first page of the list of tables, print
  \thispagestyle{ASUlotFirst}%          %   out the title using a custom page style and 
  \underline{Table}\hfill Page\par}%    %   print heading below in regular textbox

% List of Figures (LOF)
\copypagestyle{ASUlof}{ASU}
  \makeevenhead{ASUlof}{Figure}{}{Page}
  \makeoddhead{ASUlof}{Figure}{}{Page}

\copypagestyle{ASUlofFirst}{ASU}%       % Custom page headers for first page of list of  
  \makeevenhead{ASUlofFirst}%           %   figures (print out the title)
    {}%
    {\printchaptertitle{\listfigurename}}%
    {} 
  \makeoddhead{ASUlofFirst}%
    {}%
    {\printchaptertitle{\listfigurename}}%
    {}  

\renewcommand{\lofheadstart}{}%         % Usually content list titles are printed like 
                                        %   chapter headings; empty that formatting 

\renewcommand{\printloftitle}[1]{}%     % Don't print LOF title using default method; 
                                        %   it will be output in the header

\renewcommand{\afterloftitle}{%         % On the first page of the list of figures, print
  \thispagestyle{ASUlofFirst}%          %   out the title using a custom page style and 
  \underline{Figure}\hfill Page\par}    %   print heading below in regular textbox

%%% Page layout (dimensions) for Contents Lists
\newlength{\verticalpush}               % Set up to change page dimensions for the table 
                                        %   of contents
                                        % Push everything down so all the content is still
                                        %   1in from the top of the page, including the  
                                        %   header, so the header is available for titles
                                        %   on the first page of contents lists and then
                                        %   the headings on subsequent pages 
\setlength{\verticalpush}%              % Calculate difference between \headdrop and the 
  {1.0in - \headdrop}                   %   total upper margin (1in), so you can push 
                                        %   the top of the header down into the textbox

\newcommand{\contentslistsetup}{%       % Set up for contents lists
  \changepage%                          % Adjustment to page dimensions: 
    {-\baselineskip}%                   %   text height
    {}%                                 %   text width
    {}%                                 %   even-side margin
    {}%                                 %   odd-side margin
    {}%                                 %   column sep.
    {\verticalpush}%                    %   topmargin
    {}%                                 %   headheight
    {}%                                 %   headsep
    {-\verticalpush+\baselineskip}%     %   footskip
}

\newcommand{\closecontentslistsetup}{%  % Undo set up for contents lists
  \changepage{\baselineskip}{}{}{}{}%
    {-\verticalpush}{}{}{\verticalpush-\baselineskip}%
}

% Content lists can also be output directly. If the following command were used, all the 
%   headings would have to be output manually (i.e., can't rely on any memoir macros for 
%   formatting or setting in contents lists headings and lists). It would be best to 
%   create a custom macro, such as '\customtoc', to output headings and content lists 
%   following the style guide. 
%
% \makeatletter
%   \@starttoc{toc}
% \makeatother

% These pages partly explain why it's difficult to use 'afterpage' to change page layout 
%   settings (essentially, it's because everything inside \afterpage has a local scope). 
%   If it were possible to use 'afterpage' in that way, the content lists would  be 
%   easier to format. A new page layout could be called after the first page of each 
%   content  list. Instead, use page marks to get the layout required by the style guide. 
% https://tex.stackexchange.com/questions/97126/attempts-to-manually-change-linewidth-ignored-by-latex
% https://tex.stackexchange.com/questions/85729/page-styles-only-work-for-thispagestyle-under-afterpage

%%%%%%%%%%%%%%%%%%%%%%%%%%%%%%%%%%%%%%%
% Footnotes and Endnotes
%%%%%%%%%%%%%%%%%%%%%%%%%%%%%%%%%%%%%%%
\usepackage{chngcntr}                   % Modify counters (e.g., for figures, footnotes)
\counterwithout*{footnote}{chapter}     % Make footnote numbering continuous throughout

\providetoggle{useendnotes}
\settoggle{useendnotes}{false}           %<Set to 'true' if you want to use endnotes
\iftoggle{useendnotes}{%                % Use the command \pagenote to create endnotes
                                        %   in the running text. They will be collected
                                        %   and printed in a 'Notes' section at the end
                                        %   of the document

  \makepagenote                         % Required in preamble if using endnotes
  \continuousnotenums                   % Numbering does *not* reset after each chapter
  \renewcommand*{\pagenotesubhead}[3]{} % No subheads inside note list (default is to 
                                        %   divide them by chapter)
  \renewcommand*{\notenuminnotes}[1]%   % Remove extra space between note number and note
    {\normalfont #1.}                   %   text
  \renewcommand{\postnoteinnotes}%      % Double space *between* notes
    {\par\vspace{\baselineskip}}
}{}                                     % Do nothing here if not using endnotes


%%%%%%%%%%%%%%%%%%%%%%%%%%%%%%%%%%%%%%%
% Bibliography
%%%%%%%%%%%%%%%%%%%%%%%%%%%%%%%%%%%%%%%
\newcommand{\bibfilename}{library}%<Enter the name of the *.bib file containing the 
                                        %   reference information for sources cited in
                                        %   the text. God help you if you're doing 
                                        %   citations manually. 
\newcommand{\bibheading}{References}    %<Enter the heading for the references section:
                                        %   'References', 'Works Cited', or 'Bibliography'

\providetoggle{usebiblatex}             % True = a biblatex package is being used; 
                                        %   False = 'natbib' is being used
\settoggle{usebiblatex}{true}           %~Set to 'false' to use 'natbib' intead of 
                                        %   biblatex; I strongly recommend using biblatex
                                        %   because natbib is rather old and will break 
                                        %   for innocuous things like underscores in URLs 
\iftoggle{usebiblatex}{%                % Settings for citation package
%                                       % Settings for 'biblatex' or a version of 
%                                       %   'biblatex'
  % \usepackage[authordate,%                 
  %             backend=biber,%           % Recommend to use 'biber' instead of 'bibtex'
  %             doi=only,%                % Avoid printing URLs
  %             isbn=false]%              % Don't print ISBN numbers
  %             {biblatex-chicago}        %~Other possibilities include: 'biblatex', 
  %                                       %   'biblatex-apa', and 'biblatex-mla'

\usepackage[  style=numeric-comp,
              giveninits=true,
              backend=biber,%           % Recommend to use 'biber' instead of 'bibtex'
              isbn=false, %             % Don't print ISBN numbers 
              url=true,                 % Print URLs          
              sorting=none,
              maxbibnames=4, minbibnames=4, 
              hyperref]%           
              {biblatex}                %~Other possibilities include: 'biblatex', 
                                        % 'biblatex-apa',and 'biblatex-mla'  
  \bibliography{\bibfilename}
  \setlength{\bibitemsep}%              % Set vertical distance between 
    {0.5\baselineskip}%                 %   bibliography entries 
  \setcounter{biburlnumpenalty}{9000}   % Break URLs in bibliography across lines
  \setcounter{biburlucpenalty}{9000}
  \setcounter{biburllcpenalty}{9000}

  \usepackage[style=american,%          % Settings for quotation marks; load after 
    english=american]{csquotes}%        %   'inputenc'; only use with biblatex; throws 
  \MakeOuterQuote{"}%                   %   error when used with natbib
}{%                                     % Settings for 'natbib'
  \usepackage{natbib}%
  \newcommand{\natbibstyle}{asudis}%    %~Enter the name of the *.bst file to use to 
                                        %   format citations with natbib. Default is 
                                        %   'asudis'. I do not know where 'asudis' came
                                        %   from, but apparently it formats citations
                                        %   correctly because it was included with the 
                                        %   previous LaTeX template.   
}


%%%%%%%%%%%%%%%%%%%%%%%%%%%%%%%%%%%%%%%
% Tables and figures
%%%%%%%%%%%%%%%%%%%%%%%%%%%%%%%%%%%%%%%
\captiondelim{. }                       %~Use period (.) after caption number instead of 
                                        %   colon (:). Change according to style guide. 
\captionstyle[\centering]%              % Set justifcation for [one line captions] 
  {\raggedright}                        %   and {multiple line captions}
\setlength{\belowcaptionskip}{0pt}      % Bring caption down closer to figure/table

% \makeatletter                           % Consecutive numbering throughout 
%   \counterwithout{figure}{chapter}      %   (including back matter)
%   \counterwithout{table}{chapter} 
%   \renewcommand\@memfront@floats{} 
%   \renewcommand\@memmain@floats{} 
%   \renewcommand\@memback@floats{} 
% \makeatletter

%%% Tables %%%
%
% Note: 'memoir' natively supports commands from the following table-related packages: 
%   tabularx, ccaption, booktabs.
% Everyone has particular ideas about how tables should look, so you may need to 
%   load additional packages and modify the code below to get tables (and figures) to 
%   look the way you want them to. 
%\setfloatadjustment{table}{\raggedright}
\setfloatadjustment{table}{\centering}% Left justify material inside table floats
\usepackage{tabu}                       % 'tabu' is an excellent table package; it can 
                                        %   automatically size column widths and has a 
                                        %   lot of customizations that other packages do
                                        %   not. It also has a 'longtabu' environment that 
                                        %   emulates 'longtable' with additional features
                                        %   from the 'tabu' package. If you don't want 
                                        %   to use it, you can comment this line out. 
\BeforeBeginEnvironment{table}%         % Single space inside table environment
  {\SingleSpacing}
\AfterEndEnvironment{table}
  {\DoubleSpacing}

%%% Figures %%%
\setfloatadjustment{figure}%            % Left justify material inside figure floats
  {\centering}%{\raggedright}
\BeforeBeginEnvironment{figure}%        % Single space inside figure environment
  {\SingleSpacing}
\AfterEndEnvironment{figure}
  {\DoubleSpacing}

\makeatletter                           % Define custom macro called '\maxwidth{}' that
  \def\maxwidth#1{%                     %   allows you to specify the maximum width of an 
    \ifdim%                             %   imported image. See below for an example.
      \Gin@nat@width>#1 #1%             %   
    \else%                              % Source: http://tex.stackexchange.com/questions/86350/includegraphics-maximum-width
      \Gin@nat@width%
    \fi}
\makeatother
%
% Example \maxwidth: 
%
%   \includegraphics[width=\maxwidth]{\textwidth}]{image.pdf}
% 
% Note: This will keep an image inside the horizontal margins assuming the image starts 
%   on the right margin (i.e., no horizontal space before the image). 

%%%%%%%%%%%%%%%%%%%%%%%%%%%%%%%%%%%%%%%
% Hyperref settings
%%%%%%%%%%%%%%%%%%%%%%%%%%%%%%%%%%%%%%%

%%% URL Settings %%%
\PassOptionsToPackage{hyphens}{url}
\usepackage[breaklinks=true]{hyperref}  % 'hyperref' should be loaded at the end of the 
                                        %   preamble; Note: the uppercasing commands used 
                                        %   throughout the preamble can conflict with it, 
                                        %   especially when non-standard fonts or 
                                        %   different file encodings are used
\urlstyle{same}                         % Set URLs in the same font as regular text

\tolerance 1414                         % Help URLs from entering margins
\hbadness 1414                          % Source: https://tex.stackexchange.com/questions/3033/forcing-linebreaks-in-url
\emergencystretch 1.5em
\hfuzz 0.3pt
\widowpenalty=10000
\vfuzz \hfuzz

%%% Create metadata strings
\usepackage{hyperxmp}                   % For metadata 
\renewcommand*{\do}[1]{#1\ }%           % Build title string to output to pdf document
\newcommand*{\onelinetitle}{%        
  \dolistloop{\titlelines}%
}
\edef\theonelinetitle%
  {\onelinetitle}

\renewcommand*{\do}[1]{{#1}\ }%          % Build keyword string to output to pdf document
\newcommand*{\pdfkeywordsstring}{%        
  \dolistloop{\keywords}%
}
\edef\thepdfkeywordsstring%
  {\pdfkeywordsstring}

\newcommand*{\pdfcopyrightstring}%      % Build copyright message string
  {Copyright \copyright\space\gradyear\ by \Author.%
  {\space}All rights reserved.}

\ifpdf                                  % Build pdf creator string (for pdfTeX)
  \makeatletter
  \def\extractpdftexversion#1-#2-#3 #4%
    \@nil{#3}
  \edef\pdfcreator{pdfTeX \expandafter%
    \extractpdftexversion\pdftexbanner\@nil}
  \makeatother
\fi
\ifxetex                                % Build pdf creator string (for XeTeX)
  \edef\pdfcreator{XeTeX %
    \the\XeTeXversion\XeTeXrevision}
\fi

\edef\pdfsummary{%                      % Build pdf summary
  A \documentname Presented in\space
  Partial Fulfillment of the\space
  Requirements for a \degreename\space 
  from Arizona State University}

%%% Enter metadata and other settings
\hypersetup{                            % Set pdf metadata
  pdftitle={\theonelinetitle},          % Title
  pdfauthor={\Author},                  % Author
  pdfcreator={\pdfcreator},             % Enter the TeX writer for good documentation 
 %pdfproducer={},                       % Let 'pdfproducer' be filled automatically
  pdfsubject={\pdfsummary},             % Subject of the document
  pdfkeywords=\thepdfkeywordsstring,    % List of keywords
  hidelinks={true},                     % Links look like regular text (no colors, boxes)
  breaklinks={true},                    % Allow links to break across lines
}
\ifxetex                                % If processing with XeTeX
  \hypersetup{unicode=true}             % Must use 'true' in XeTeX
\else
  \hypersetup{unicode=true}             % Default is to use 'true' otherwise, as well
\fi
\ifpdf                                  % Copyright message; probably only works in pdfTeX
  \hypersetup{
    pdfcopyright={\pdfcopyrightstring},   
    pdfinfo={%
      Copyright=\pdfcopyrightstring%      
    }%
  }
\fi

\usepackage%
  [numbered,%                           % Include numbers of sections in bookmarks
  open%                                 % Bookmark tree already expanded when PDF opened
  ]%
  {bookmark}
\bookmark[page=1,rellevel=0,%           % Create bookmark of title page at root level
  keeplevel=true]{Title Page}
\preto{\tableofcontents}{%              % Create bookmark for TOC 
  \hypertarget{tocpage}{}%
  \bookmark[dest=tocpage,rellevel=0,%
    keeplevel=true]{\contentsname}%
}

%%%%%%%%%%%%%%%%%%%%%%%%%%%%%%%%%%%%%%%
% Copyright page
%%%%%%%%%%%%%%%%%%%%%%%%%%%%%%%%%%%%%%%
\newcommand{\copyrightpageASU}{%          % Create copyright page
  \thispagestyle{empty}
  ~\\ \vfill
  \parbox{\textwidth}{%
    \begin{center}
      \copyright\gradyear\space%
      \Author\\%
      All Rights Reserved%
    \end{center}%
  }%
  \clearpage%
}

%%%%%%%%%%%%%%%%%%%%%%%%%%%%%%%%%%%%%%%
% Sample settings
%%%%%%%%%%%%%%%%%%%%%%%%%%%%%%%%%%%%%%%
% \providetoggle{sample}                  % True = demonstration of template
% \settoggle{sample}{false}
%  \iftoggle{sample}{%
%   \newcounter{tablecounter}
%   \setcounter{tablecounter}{1}
%   \newcounter{figurecounter}
%   \setcounter{figurecounter}{1}
% }{%
% }

%%%%%%%%%%%%%%%%%%%%%%%%%%%%%%%%%%%%%%%
% Debugging Help
%%%%%%%%%%%%%%%%%%%%%%%%%%%%%%%%%%%%%%%
\usepackage{lipsum}                     % Outputs dummy text

