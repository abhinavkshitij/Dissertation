\graphicspath{ {./Ch8/}  } 
\DeclareGraphicsExtensions{.png,.pdf,.jpg}

%%%%%%%%%%%%%%%%%%%%%%%%%%%%%%%%%%%%%%%%%%%%%%%%%%%%%%%%%%%%%%%%%%%%%%%%%%%%%%%%%%%%%%%%%%
\chapter{Frame-invariant Volterra Series Representation }

We recently used the results of Smith (1971) to obtain the most general representation of the symmetric subgrid stress tensor   in terms of the strain and rotation rate tensors S and R and rank-two products up to second order of their gradients   and  . In particular, Smith’s framework guarantees that the resulting tensor polynomial representation satisfies the rank, symmetry, rotation, and reflection properties of the stress tensor   – essential for any such representation to be strictly valid.  Moreover, Smith’s formulation was shown by Pennisi & Trovato (1987) to be minimal, thus there is no smaller number of symmetric basis tensors that can form a complete polynomial representation of  .   
The resulting representation, while complete and minimal, is computationally expensive to evaluate.  It consists of 1570 tensor terms that each involve products up to fourth order in the underlying rank-two tensors Mi and Wp , each of which in turn involve products up to second order in S, R,   and  .  Since the strain and rotation rate tensors S and R, and gradients   and  , are linear in the components of the velocities   at the m = 1, 2, … , P = 27 points on the local   stencil, each term in this frame-invariant representation ultimately involves products up to eighth order in these velocity components.  
One might suspect that some of the many resulting velocity component products appearing in each of the 1570 tensor bases may be the same. Leaving them grouped as they appear in this 1570-term tensor polynomial form involves the smallest number of coefficients of any complete frame-invariant symmetric tensor representation. However, that may be less computationally efficient than if these repeated velocity component products were instead grouped together in an equivalent series having more coefficients but involving far fewer arithmetic operations to evaluate all components of all the basis tensors. However, due to the enormous number of such repeated velocity component products in this tensor polynomial, identifying such alternate groupings “by hand” is completely impractical, and even our efforts to use symbolic mathematics software to identify repeated velocity component products have not led to success.
Alternatively, it may be possible to use the results of Smith (1971) to obtain a different, and potentially more computationally efficient, complete and minimal tensor polynomial in terms of the P = 27 velocity vectors   on our   stencil, rather than via the 1570-term tensor polynomial in S, R,   and  . Smith’s formulation in fact provides the complete and minimal tensor bases for representing any symmetric rank-two tensor in terms of any number of vectors and rank-two tensors.  Rather than forming the tensor polynomial representation in terms of rank-two tensors from S, R,   and  , we will here use Smith’s formulation to obtain the tensor polynomial representation directly in terms of the m = 1, … , P = 27 velocity vectors   on the stencil.  Such a series will naturally have each possible velocity component product appear only once, and thus should provide the potentially efficient alternative representation noted above.
In fact, as we will see, this  representation in terms of tensor bases formed directly from the   will be very similar to the original ad hoc Volterra series we began with.  However, while the original Volterra series representation was not frame invariant, this new representation will be. Moreover, our original Volterra series was arbitrarily truncated after second-order products in the velocity components, but in this new representation we will see that in any valid representation of   in terms of   there cannot be any velocity component products of order higher than two. Thus in this new representation no truncation is needed, even while preserving the completeness of the tensor representation. 
We may anticipate that the resulting new polynomial in the basis tensors formed from the   will have more terms than does our recent 1570-term polynomial in the basis tensors formed from S, R,   and  .  Yet this new polynomial may nevertheless be more computationally efficient, since each component of its basis tensors involve only a simple second-order velocity component product, and we are guaranteed that there are no repeated products in this representation. 

%%%%%%%%%%%%%%%%%%%%%%%%%%%%%%%%%%%%%%%%%%%%%%%%%%%%%%%%%%%%%%%%%%%%%%%%%%%%%%%%%%%%%%%%%%
\section{Complete and Minimal Tensor Representation of   in  } 

Here, we seek a tensorally correct representation for the symmetric rank-two stress tensor   in terms of the m = 1, … , P = 27 velocity vectors   of the form  , meaning a representation that preserves the rank, symmetry, rotation and reflection properties of  .  Following Smith (1971; Eq. 4.5), any such rank-two symmetric tensor   can be represented in a complete and minimal tensor polynomial in terms of any number P of vectors   as
   ,                   (1)
where   denotes the “dyadic product” (or “outer product”, or “tensor product”) defined for any two vectors   and   as
     .                                               (2)
Unlike our original Volterra series representation, note that all six independent components of   in (1) have the same set of coefficients c, so that (1) preserves all invariants of   in any coordinate frame, whereas our original Volterra representation was not invariant-preserving and thus had separate coefficients for each ij-component of  .
Since in our case the   are the velocities at each of the 27 stencil points, the symmetric dyadic products in (1) and (2) are simply all possible invariance-preserving combinations of all second-order products (colocated and non-colocated) of the velocity components.  The first sum gives the 27 colocated velocity products, which include both square ( ) and non-square 
( ) terms. The second sum gives the   non-colocated velocity products; note how the dyadic products in this term are constructed to preserve the ij-symmetry of  .
Strictly speaking, representation theory dictates that the scalar-valued coefficients c in (1) can, at most, be functions of all scalar invariants that can be formed from the set of vectors  . From Smith (1971; Eq. 2.41), for a representation involving only the vectors  , the only such scalar invariants are   and   where   and  .  In principle, one could try to somehow discover how each of the c’s in (1) depend on the   scalar invariants   and the   scalar invariants  , and if successful in doing so would have arrived at the only possible tensorally-correct representation of the form  .  However, the whole point of autonomic closure is precisely to avoid a need to propose models for coefficients   in representations for  , and to instead use tools of data science to autonomically discover the best values for these coefficients at each simulation point and time. 
Returning to (1), it might be tempting to discard the term   by arguing that the definition of  , namely 
       or equivalently         ,                   (3a,b)
requires   when  . One might thus consider forcing   to ensure (1) satisfies this requirement.  However, recall from above that   is in principle a scalar-valued function   of the scalar invariants   and  , so when all the   then these scalar invariants are all zero, in which case it is possible (even likely) that  .  This then allows   when   and thereby satisfies   when   even when the   term in (1) is retained.  Furthermore, when   we may expect  , in which case again the   term in (1) must be retained.  Thus in all cases, the   term in (1) must be retained.  Only when   in (1) is the deviatoric stress can we set  .
There is, however, a problem with (1) that can be easily recognized and that originates from a subtle assumption in Smith’s representation.  It is well known that to leading order the stress   depends linearly on the strain rate  ; numerous such linear stress-strain models, including the basic and dynamic Smagorinsky models, are fairly successful by assuming a linear relation between   and S, namely   or equivalently  , where   and   is a Smagorinsky coefficient or, more generally, an eddy viscosity.  However, (1) does not allow any such linear stress-strain form, because it has no terms that are linear in the velocities   and thereby does not allow any linear variations of   with S.
The reason (1) has no linear terms in the velocities   is because all the vectors   in Smith’s representation are implicitly assumed to be located at the same point at which   is being represented. When the vectors   are at the same point then all linear combinations of them, namely sums and differences of the vectors  , are also vectors (rank-one tensors) – there are no possible linear combinations of the   that form rank-two tensors.
By contrast, in autonomic closure the   are not strictly located at a single point, but instead on a discrete spatial stencil centered on the point at which   is being represented.  As a result, it is possible for linear combinations of the   together with their spatial locations to form a rank-two symmetric tensor. In fact, there is just one rank-two symmetric tensor that can be formed in this way, namely  .
Accordingly, to account for the fact that the   are not all at the same spatial point at which   is being represented, but rather on a spatial stencil centered on the point, we must add to (1) the only allowable rank-two symmetric tensor that is linear in the  , namely  . Since this is linear, symmetric, and preserves the rotation and reflection properties of  , it must be included in any complete representation of   in terms of the velocities   on a spatial stencil around the point at which   is being represented.
Thus, from Smith (1971) and the additional reasoning above regarding linear terms, the most general and complete representation for   in the stencil velocities   that preserves the rank, symmetry, rotation, and reflection properties of   is 
       (4)
where the second term now correctly brings in the leading-order linear stress-strain relation. 
Note that, since the c’s in (4) are the same for all six independent components of  , there are only 1 + 1 + 27 + 351 = 380 coefficients involved in the entire representation of  , even though (4) includes all possible invariance-preserving non-colocated velocity products. The original Volterra series was not invariance-preserving, and required   coefficients to represent the entire tensor  . If only colocated velocity products were retained in the original Volterra series it required   coefficients, whereas in (4) including only colocated velocity products amounts to omitting the final term, which leaves just 1 + 1 + 27 = 29 coefficients that must be found to completely represent  . 
It might be argued that, since we have included the strain rate tensor in the linear term in (4), we should also include all higher-order rank-two symmetric invariance-preserving combinations of S and R – that is what our earlier representation in terms of S, R,   and   did. However, the two sums in (4) already account for all possible second-order velocity component products on the stencil, so in a sense including second-order terms that can be formed from S, R,   and   might be duplicative. In fact, the general representation of Smith (1971) does not allow any higher-order velocity component products in (1), which might suggest that there can also be no higher-order products of S, R,   and   in (4). Yet that argument is not a strong as we might like, since strictly speaking Smith’s representation assumes all quantities to be at the same point.  There is no fully correct answer to this, since we are mixing Smith’s formulation – which assumes all the vectors to be at the same point – with an additional linear term that seeks to account for the fact that the vectors are not at the same point.  It seems best not to assert that (4) is complete and minimal, and instead to say that it has been truncated after second-order velocity products similar to what we did with the original Volterra representation and with our 1570-term invariance-preserving representation in S, R,   and  .
Regardless whether (4) can be considered complete or truncated, it definitely preserves the rank, symmetry, rotation, and reflection properties of  , and thus can be used as the basis for an invariance-preserving implementation of autonomic closure. While it remains an open question whether it implicitly involves truncation of higher-order velocity products, we would likely truncate it anyway after second-order products, since our work to date has suggested this is sufficient to obtain very accurate representations of  . The real advantages of (4) over the original Volterra representation include:
•	Unlike the original Volterra representation, (4) is tensorally correct in that it represents the entire tensor   as a single polynomial in which each term is a rank-two tensor that preserves the symmetry, rotation and reflection properties of  . 
•	While (4) as written is not Galilean invariant, it can be readily made to satisfy translation invariance by simply subtracting the stencil center point velocity from all velocities on the stencil, and letting   denote the resulting velocity differences.  Note this forces the resulting   at the stencil center point to be zero, thereby reducing P from 27 to 26 independent velocity vectors.  This in turn reduces the number of coefficients in (4) to 
1 + 1 + 26 + (262 – 26)/2 = 353, and if only colocated velocity products are included then the number of coefficients is 1 + 1 + 26 = 28.
•	The representation in (4) involves far fewer coefficients than did the original Volterra series, and thus would allow far faster “on-the-fly” implementations of autonomic closure in an LES code. It involves a factor of   fewer degrees-of-freedom N in its representation of   than did the truncated Volterra series in the components of  . Since the computational burden of autonomic closure scales as  , with   this reduction in N should reduce the computational time by a factor of   relative to the original truncated Volterra series.
•	A further advantage of (4) has to do with the number M of training points needed. Due to the large reduction in the number of coefficients needed to represent  , far fewer training points should be needed to determine these coefficients while keeping N/M the same as we previously used. That should allow even smaller bounding boxes than the current 103 box used in the CL24 implementation of the Volterra series representation. This would allow a more local closure, and thus presumably a more accurate closure, in addition to a further reduction in computational cost from the smaller number of training points.
•	The representation for   in (4) also involves fewer terms and fewer coefficients c than does the 1570-term velocity-only invariance-preserving representation formulated in S, R,   and  .  
These factors suggest the representation in (4) is a superior form for use in autonomic closure than either the ad hoc truncated Volterra series.  It may also be a more computationally efficient but less complete form than is the 1570-term representation formulated in S, R,   and  .
We should thus test (4) as a generalized invariance-preserving representation of   in terms of velocities   on a   stencil. Despite some similarities with the original series representation we should not refer to (4) as a Volterra series representation, but rather as an “invariance-preserving representation for  ” motivated by the Smith (1971) formulation, but with the addition of the only possible linear term to account for the fact that the velocities are not strictly at the same point but on a spatial stencil.

