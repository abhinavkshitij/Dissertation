\graphicspath{ {./Ch3/}  } 
\DeclareGraphicsExtensions{.png,.pdf,.jpg}

%%%%%%%%%%%%%%%%%%%%%%%%%%%%%%%%%%%%%%%%%%%%%%%%%%%%%%%%%%%%%%%%%%%%%%%%%%%%%%%%%%%%%%%%%%
\chapter{Autonomic closure}

This section presents a description of the autonomic closure methodology [58,59] and the implementation choices within it that affect its accuracy and computational cost. 
 
%%%%%%%%%%%%%%%%%%%%%%%%%%%%%%%%%%%%%%%%%%%%%%%%%%%%%%%%%%%%%%%%%%%%%%%%%%%%%%%%%%%%%%%%%%
\section{The autonomic closure methodology} 

 Traditional prescribed closure models represent a subgrid term in a predefined way in terms of specified parameters, such as the resolved strain rate  , based on theoretical or other considerations, often with one or more model constants allowed to vary locally via a dynamical procedure [27,46,47]. In contrast, autonomic closure as proposed in Refs. [58,59] is based on a highly generalized nonparametric representation of subgrid terms using only the primitive variables in a simulation. This generalized nonparametric representation typically has O(102-104) degrees of freedom that are determined dynamically at every point x and time t in the simulation. Such a highly generalized representation in the primitive-variable values on a local set S of stencil points around each space-time point (x,t) removes the need for a predefined parametric model. Doing so allows far greater adaptability of the subgrid closure to the local turbulence state than is possible even with dynamical forms of traditional prescribed closure models. 
Here we apply autonomic closure to the subgrid stress  . Although the methodology can be generalized to multi-time stencils [58,59], here we consider a time-local implementation. An underlying general nonparametric representation   for the local subgrid stress   can then be expressed in the primitive variables   and   as
   ,                                (5)
where   is a set of points   that define a stencil (here  ) with separation   on the LES grid, as shown in Fig. 2. Equation (5) is a general nonparametric relation between the local subgrid stresses   and the resolved-scale variables   and   on the LES grid.   in (5) should ideally reflect the local turbulence state at x, including nonlinear, nonlocal, nonequilibrium and other effects that determine how the local   can be best obtained from the resolved-scale values   and   at the stencil points   centered on x. 
Analogous to the subgrid stresses  , we consider local test stresses   
[27-29,39,40,46-54], which can be obtained from the resolved velocities   by applying a test filter   having a larger length scale  . If   reflects the local turbulence state at x then, in the same way that   is related to the resolved-scale variables   and   on the set  , so also should   be related to the test-scale variables   and   on the corresponding set   of test-scale stencil points centered on x with separation  . In other words, analogous to (5), we have
   ,                              (6)
where the stencil   is the same as   but is defined on the test-scale grid, as shown in Fig. 2. 
The central idea in autonomic closure is that at each point x and time t, the known test stress value   and the known surrounding test-scale variables   and   on the stencil   in (6) can be used to obtain information about the local form of  . Repeating this at multiple training points within a bounding box centered on x allows   to be determined sufficiently accurately that it can be generalized, in a machine learning sense, from the test scale to the LES scale. The resulting   is then used to determine   via (5) from the surrounding variables   and   on the stencil   centered on x. The relation   in (5) and (6) has a sufficient number of degrees of freedom that it is free to adapt to the local turbulence state at x to make  , and thereby make  .
It is in this way that autonomic closure accesses internal training data within the simulation, namely the test stresses within a bounding box centered on x, to discover the local connection   between the local test stress   and the local test-scale primitive variables   and  . It then uses this   to evaluate the local subgrid stress   from the local resolved-scale primitive variables   and  . In effect the simulation itself provides the training data, which are used to solve the local, nonlinear, nonparametric system identification problem that discovers the local connection between the subgrid stress and the resolved-scale primitive variables. 
Galilean invariance is enforced by subtracting the velocity at the stencil center point from the velocities on the stencil. However   in (5) and (6) does not readily lend itself to explicitly imposing the tensor invariance or realizability properties of  . For example, formulating   in the tensor integrity basis [67,68,72-74] for the strain rate and rotation rate tensors,   and  , which is the basis for some prescribed subgrid stress models, presumes that   depends only on combinations of these two tensors, whereas the nonparametric formulation in (5) in the primitive variables   and   makes no such assumption. Instead, as is common in many machine learning methods [67,68], the tensor invariance and realizability properties are inherent in the training data  , which informs the learned   in (6) and thereby implicitly communicates these properties to   in (5). Results in Section IV show that   and   from this autonomic closure methodology are far more accurate than those from common traditional prescribed closures that explicitly enforce tensor invariance properties. 
Although any sufficiently general form for   could be used in (5) and (6), we here choose a Volterra series [75] in u and p, as in Refs. [58,59], namely the sum of all products of all orders of all variables at all points on the stencil, including all possible multi-point multi-variable products at each order. Such a representation is highly general. Even if truncated after second order with a   stencil,   for each ij consists of N = 5995 zeroth-, first-, and second-order products of u and p in (5) and (6), each having a separate coefficient   with  , where N is the number of degrees of freedom in  . This set of coefficients for each ij is thus an N-length column vector denoted  . To gain computational efficiency,   could be truncated at lower orders, or restricted to single-point products, or limited only to the velocities u on the stencil. Even if   is truncated after first order and limited to single-point velocities on a   stencil, it still contains N = 82 first-order terms in the u components for each ij, and thus has far more degrees of freedom than do dynamic versions of traditional prescribed closure models.  
With such a Volterra series for  , the stress value   at the stencil center point x can be written from (5) as
  ,                                                          (7)
where   is the N-length array containing the known values of all products of all orders of   and   in   at all points   on the stencil  . If   is known then the value of   at the stencil center point x can be obtained from (7). To determine  , (6) can be similarly written as  , where   is the N-length array containing the known values of all products of all orders of the test-filtered primitive variables   and   in   at all points   on the stencil  , and where the test stress value   at the stencil center point x is known. Repeating this with the stencil   centered at each of M training points within a local bounding box around the point x, in which variations in the turbulence state embodied in   are taken to be negligible, then   becomes an   matrix. With   denoting the corresponding M-length column vector consisting of the known   values at the M training points, we then have for each ij
  .                                                         (8)
The size of the bounding box, typically extending along homogeneous directions, determines the maximum number of available training points within it. The chosen number M of training points and the bounding box volume   determine the relative training point spacing  , which together with M determines how much effectively independent information is being used to characterize the local turbulence state via  , or equivalently via  , within the bounding box around the point x. The choice of bounding box size and the number of training points M are part of any implementation of autonomic closure. Regardless of the M and N values, since the vector   and the matrix   are known, the system in (8) may be solved by any number of means. Here we use a damped least-squares solution [59] of the form
                                                    (9)
where   is the damping coefficient. When   the value of   is unimportant and is set to  ; when   then   is set to  . Once the coefficients   at x have been determined via (9), they are used in (7) to evaluate   at x. 
Note the resulting local   is used only once to evaluate   at the bounding box center point x for the current time t. A new set of coefficients   is obtained for each   in the simulation. As a result, autonomic closure does not provide a fixed set of coefficients   and thus does not provide a closure model for  . Instead, it is the autonomic methodology itself that is the closure for  .

%%%%%%%%%%%%%%%%%%%%%%%%%%%%%%%%%%%%%%%%%%%%%%%%%%%%%%%%%%%%%%%%%%%%%%%%%%%%%%%%%%%%%%%%%%
\section{Implementations of autonomic closure }

Implementing autonomic closure involves choices of  , N, M, and   that impact the generalizability of  , in a machine learning sense, from the test scale to the LES scale, and also determine the computational cost of this closure methodology. For example, the implementation in Ref. [59] used a second-order, non-colocated, velocity-pressure Volterra series for  , which provided a large number (N = 5995) of coefficients   in  , and also used the largest possible bounding box, which allowed a large number (M  = 15,625) of training points to determine  . Results from that implementation verified that autonomic closure produces subgrid stress fields   and subgrid production fields   that represent the true   and   fields over essentially all resolved scales far more accurately than do existing prescribed closure models, such as the dynamic Smagorinsky model [59]. 
However while the implementation in Ref. [59] is accurate, as can be seen in Fig. 1, it is too computationally costly in comparison with traditional prescribed closure models to serve as a widely-usable alternative closure for practical applications of LES. Therefore, in the following sections we evaluate the effects of various implementation choices on the accuracy and computational cost of autonomic closure. Specifically, we consider specific combinations of the following key implementation choices: 
•	local vs. nonlocal forms based on the bounding box volume   
•	velocity-only vs. velocity-pressure Volterra series representations Fij
•	colocated vs. non-colocated products on the stencils   and  
•	first-order vs. second-order Volterra series representations Fij
•	varying numbers M of training points relative to the number N of coefficients  
•	varying training point spacing  
These allow implementations ranging from a large, second-order, velocity-pressure, non-colocated, nonlocal formulation with  , which provides a large number N of degrees of freedom in   but is computational costly, to a minimal first-order, velocity-only, colocated, local formulation with  , which has the lowest computational cost but can be expected to be less accurate. The main objective is to determine which implementations of autonomic closure provide high accuracy in   and   at acceptable computational cost.
For instance, because velocities at distant points affect local pressure values on the stencil, a velocity-pressure implementation adds nonlocality beyond the stencil points in   and also increases its number of degrees of freedom N. Both effects can increase generalizability of the resulting   from the test scale to the LES scale, but at increased computational cost over a velocity-only formulation. Similarly, including non-colocated products of the stencil-point variables or increasing the truncation order of the Volterra series for   increase N and introduce additional physical information in  , but at an increased computational cost that may not be merited. Larger numbers M of training points should also lead to increased accuracy, but for any bounding box volume   the spacing   between training points becomes smaller as M increases, thus providing relatively less additional independent training information to determine   while increasing the computational cost. These expectations suggest a non-trivial tradeoff between N, M, and  .  
It can also be expected that large bounding box volumes  , which allow larger numbers M of widely-spaced (and thus more independent) training points, will provide greater accuracy. However, large bounding boxes may contain substantially different turbulence states, and thus the training points will lead to coefficients   that pertain, in part, to turbulence states other than that at the point of interest x. Smaller bounding boxes give a more-local implementation that ensures training points are relevant to the local turbulence state at x, but inherently limit the number of available training points and their relative independence. This suggests that the most accurate implementation may involve a tradeoff between increased locality via smaller bounding boxes and increased training information via larger bounding boxes. 
In the following sections, we develop and apply quantitative assessment metrics to determine implementations of autonomic closure that provide high fidelity in the resulting   fields, and particularly in the corresponding   fields, yet do so at computational costs that are sufficiently low to enable practical use of autonomic closure in large eddy simulations. 
