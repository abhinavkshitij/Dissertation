Autonomic closure for large eddy simulations replaces the need for traditional prescribed subgrid closure models with an adaptive self-optimizing closure that solves a local, nonlinear, nonparametric system identification problem for each subgrid term at every point and time in a simulation. Here we develop accurate and efficient implementations of autonomic closure for the subgrid stress, with particular attention to the resulting detailed spatial structure of forward and backward scatter in the associated subgrid production fields, including metrics for assessing the scale-dependent support-density fields on which large values of subgrid production are concentrated. A relatively local, second-order, velocity-only, colocated implementation is found to produce subgrid stress and production fields that closely match not only statistics but also most details of the spatial structure in the exact fields at essentially all resolved scales, at a computational cost $O(10^3)$ lower than the original implementation. 

We then use the most general complete rank-2 tensor polynomial basis (Smith 1971) to determine a frame-invariant tensor formulation for autonomic closure of the subgrid stress in terms of the strain rate and rotation rate, S and R, and their gradients   $\mathbf{\nabla S}$  and   $\mathbf{\nabla R}$ . The rank-3 gradient tensors are contracted to form symmetric and anti-symmetric rank-2 tensors. The lack of an equivalent to the Cayley-Hamilton theorem for rank-3 tensors allows infinitely many such contractions, thus we limit these to the complete set of 19 unique second-order rank-2 contractions of the form   $\mathbf{\nabla S^2}$,   $\mathbf{\nabla S}$$\mathbf{\nabla R}$,and   $\mathbf{\nabla R^2}$. From these, together with I, S, and R, a complete and minimal symmetric tensor basis   $\mathbf{m}_{S}^{\alpha}$   is obtained. The resulting frame-invariant tensor polynomial for the turbulent stress   $\tau_{ij}$   involves 1570 terms, but when truncated to retain only terms up to second order in the velocity components  $u_i$  it involves only 21 terms. These 21 terms include all multi-point second-order velocity products, which in a velocity-only Volterra series representation required 3403 terms. This complete frame-invariant tensor formulation of autonomic closure uses all of the information available in the velocities on a  3 x 3 x 3 stencil, and does so with the smallest possible number of terms, while ensuring frame invariance.

Finally we use the general formulation of (Smith 1971) to obtain the invariance-preserving tensor representation for autonomic closure of the subgrid stress   directly in terms of the velocity vectors u at the 27 stencil points. This provides a more computationally efficient way of implementing an invariance-preserving representation of   in autonomic closure, since it uses exactly the same information that is used in the more complex representation in terms of S, R, and their gradients.  The resulting series has strong similarities with our prior truncated Volterra series representation, but does not involve the ad hoc assumption of a Volterra series representation. Interestingly, the general formulation of Smith (1971) shows that there can be no velocity component products higher than order-two in an invariance-preserving representation in terms of velocities at a point, which may eliminate the need for second-order truncation in a Volterra series. This new invariance-preserving should reduce the computational burden of autonomic closure by a factor of roughly 543 over the ad hoc Volterra series representation, and by a factor of 71 over the previous invariance-preserving representation in S, R, and their gradients.
