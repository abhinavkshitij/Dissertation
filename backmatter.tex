% %%%%%%%%%%%%%%%%%%%%%%%%%%%%%%%%%%%%%%%%%%%%%%%%%%%%%%%%%%%%%%%%%%%%%%%%%%%%%%
% % ASU Dissertation Template
% %%%%%%%%%%%%%%%%%%%%%%%%%%%%%%%%%%%%%%%%%%%%%%%%%%%%%%%%%%%%%%%%%%%%%%%%%%%%%%
% % Copyright 2015 Robert W. Kutter (robert@kutterconsulting.com)
% % 
% %   See also: http://kutterconsulting.com
% % 
% % For guidance on using this file, see the README. 
% %  
% %%%%%%%%%%%%%%%%%%%%%%%%%%%%%%%%%%%%%%%%%%%%%%%%%%%%%%%%%%%%%%%%%%%%%%%%%%%%%%
% % Preamble
% %%%%%%%%%%%%%%%%%%%%%%%%%%%%%%%%%%%%%%%%%%%%%%%%%%%%%%%%%%%%%%%%%%%%%%%%%%%%%%
% \newcommand*{\pointsize}{12pt}          %<Set the font size; make sure the size is correct
%                                         %   for the font you will use
% \documentclass[letterpaper,             % Use US letter-size paper
%                oneside,                 % No verso and recto differences
%                \pointsize]              % Uses the font size defined above
%                {memoir}
% \renewcommand{\cleardoublepage}%        % \cleardoublepage will create entirely blank 
%   {\clearpage}%                         %   pages depending on settings (e.g., usually 
%                                         %   before start of \mainmatter); redefine it here
%                                         %   so that no entirely blank pages are created
%                                         %   automatically
% %%%%%%%%%%%%%%%%%%%%%%%%%%%%%%%%%%%%%%%
% % (Some) Packages
% %%%%%%%%%%%%%%%%%%%%%%%%%%%%%%%%%%%%%%%
% \usepackage{graphicx}                   % For importing image files
% \usepackage{etoolbox}                   % For advanced commands throughout preamble
% \usepackage{microtype}                  %~Improves kerning and protrusion (optional); 
%                                         %   See here for an introduction: 
%                                         %   http://www.khirevich.com/latex/microtype/

% \providetoggle{usemicrotype}            % TRUE = microtype is being used
% \makeatletter                           %   (Used to turn of microtype protrusion in the 
% \@ifpackageloaded{microtype}%           %   table of contents.)
%   {\settoggle{usemicrotype}{true}}%
%   {\settoggle{usemicrotype}{false}}
% \makeatother
% \usepackage{changepage}                 % For changing page layout (e.g., margins) in the 
%                                         %   middle of the document
% \usepackage{calc}					              % Calculate text widths; used in page layout 
%                                         %   changes

% %%%%%%%%%%%%%%%%%%%%%%%%%%%%%%%%%%%%%%%
% % Title page, input
% %%%%%%%%%%%%%%%%%%%%%%%%%%%%%%%%%%%%%%%
% \listadd{\titlelines}%
%   {Autonomic Closure for Turbulent Flows}%<Enter the title of the dissertation
% %\listadd{\titlelines}%
% %  {Second Line of the Title}            % If you want to 
%                                         %   split the title across lines,
%                                         %   use another \listadd command for the 
%                                         %   second line
% \newcommand*\Author{Abhinav Kshitij}    %<Enter your name; must match official transcript
% \newcommand*{\documentname}%
%   {Dissertation}                        %<Enter the type of document (capitalized)
% \newcommand*{\degreename}
%   {Doctor of Philosophy}                %<Enter the type of degree (capitalized)
% \newcommand*\defdate{December 2018}     %<Give month (written out fully) and year of 
%                                         %   the oral defense
% \listadd{\committeechair}{Prof. Werner J.A. Dahm} %<Enter committee chair name; use \listadd for
% \newcommand*{\chairlabel}{Chair}        
% \listadd{\committeemember}{Dr. Marcus Herrmann} 
% \listadd{\committeemember}{Dr. Hans Mittelmann} 
% \listadd{\committeemember}{Dr. Peter Hamlington} 
% \listadd{\committeemember}{Dr. Yulia Peet} 

% \newcommand*{\gradmonth}{December}         %<Enter the graduation date; month can only be: 
%                                         %  May, August, or December
% \newcommand*{\gradyear}{2018}           %<Enter the graduation year, e.g. 2014

% \listadd{\keywords}{Turbulence}          %<Enter keywords; use \listadd for
% \listadd{\keywords}{Combustion}          %   additional names (up to 6)
% \listadd{\keywords}{Machine learning}    %   additional names (up to 6)

% \newcommand*{\graddate}{\gradmonth%     % Compose full graduation date
%   \space\gradyear}    

% %%%%%%%%%%%%%%%%%%%%%%%%%%%%%%%%%%%%%%%
% % Page layout
% %%%%%%%%%%%%%%%%%%%%%%%%%%%%%%%%%%%%%%%
% \settrimmedsize{\stockheight}%          % Specifies \paperheight and \paperwidth
%   {\stockwidth}{*} 
% \settrims{0pt}{0pt}                     % Set location of page in relation to the stock. 
%                                         % Paper and stock size are equivalent, 
%                                         % so both \trimtop and \trimedge are set to 0pt
% \newlength{\forfootskip}
% \setlength{\forfootskip}%
%   {3\baselineskip}
% \newlength{\textblockheight}            % Calculate height of text block to leave room
% \setlength{\textblockheight}{9.0in}     %   for footers, keeping page numbers outside
% \addtolength{\textblockheight}%         %   the 1in vertical margins
%   {-\forfootskip}
% \settypeblocksize{\textblockheight}%    % Calculated by 1.0in vertical margins and 
%   {*}{*}                                %   letting margins set the width of the typeblock                      
% \setulmargins{1.0in}{*}{*}              % Set upper margin (\uppermargin, not \topmargin); 
%                                         %   calculate the bottom margin
% \setlrmarginsandblock{1.25in}{1.25in}{*}%~Set margins and calculate width of typeblock
% \setheaderspaces{*}{0.5\baselineskip}{*}% Arguments: '\headdrop', '\headsep', and/or ratio
%                                         %   Note: This is only used in the list of 
%                                         %   contents sections
% \setheadfoot{\baselineskip}%            % Set '\headheight' and '\footskip'
%   {\forfootskip}
% \checkandfixthelayout                   % Required by memoir package after setting layout
% \settypeoutlayoutunit{in}               % Write layout dimensions to log file in inches

% %%%%%%%%%%%%%%%%%%%%%%%%%%%%%%%%%%%%%%%
% % Fonts
% %%%%%%%%%%%%%%%%%%%%%%%%%%%%%%%%%%%%%%%
% \usepackage[T1]{fontenc}                % Standard option to handle, e.g., accented 
%                                         %   characters like 'ö' better
% \usepackage{amssymb,mathtools}          % For AMS-LaTeX, see here for more: 
%                                         %     http://www.ams.org/publications/authors/tex/amslatex
%                                         % ('mathtools' loads and extends 'amsmath')
% \usepackage{ifxetex,ifluatex}           % Can check if XeTeX or LuaTeX was used to typeset
% \usepackage{fixltx2e}                   % Provides \textsubscript
% \IfFileExists{upquote.sty}%             % Use upquote if available, for 
%   {\usepackage{upquote}}{}              %   straight quotes in verbatim environments
                                        
% % Load fonts depending on the 
% %   typesetting engine
% \ifnum 0\ifxetex 1\fi\ifluatex 1\fi=0   % If pdftex
%   \usepackage[utf8]{inputenc}           %   'utf8' should match the encoding of this file
%                                         %
%                                         %~Set up your font in pdftex here
%                                         %
% \else 									% If xetex or luatex 
%   \ifxetex                              % If xetex
%     \usepackage{mathspec}               % Matches non-math open-type font to math 
%                                         %   open-type font (use 'mathspec' if you want to
%                                         %   write math in unicode)
%     \usepackage{xunicode}               % Convert LaTeX character macros to unicode
%   \else                                 % If luatex\usepackage{fontspec}
%     \usepackage{fontspec}               % Use fontspec for (open type) font selection
%   \fi
%   \defaultfontfeatures{Mapping=tex-text,% Font spec setting
%     Scale=MatchLowercase} 
%   \newcommand*{\euro}{€}
%   \setmainfont{Times New Roman}         %<Set the main font; make sure the font is correct
%                                         %   for the font size (See ASU Style Guide)
% %  \setmathfont(Digits,Latin,Greek)%     %~Uncomment two lines to set a font for math%
% %    {MATHFONT} 
% \fi

% %%%%%%%%%%%%%%%%%%%%%%%%%%%%%%%%%%%%%%%
% % Line spacing
% %%%%%%%%%%%%%%%%%%%%%%%%%%%%%%%%%%%%%%%
% \DoubleSpacing                         % True double spacing
% \BeforeBeginEnvironment{quote}         % Memoir leaves most special material 
%   {\SingleSpacing}                     %   single spaced, but makes block quotes 
% \AfterEndEnvironment{quote}%           %   double-spaced; fix to follow ASU style guide
%   {\vspace{-\baselineskip} %
%   \DoubleSpacing}
% \BeforeBeginEnvironment{quotation}%
%   {\SingleSpacing}
% \AfterEndEnvironment{quotation}%
%   {\vspace{-\baselineskip} %
%   \DoubleSpacing}

% \setlength{\footnotesep}{\baselineskip}% Double space *between* footnotes
% \renewcommand*{\footnoterule}{%        % Redefine footnoterule so that initial footnote
%   \kern-3pt%                           %   still appears right under the rule (changing
%   \hrule width 0.4\columnwidth         %   \footnotesep also changes the space between the
%   \kern 2.6pt                          %   rule and the first footnote
%   \vspace{-0.5\baselineskip}           % (Here is the vertical space adjustment)
%   }

% \usepackage{enumitem}                  % Control spacing in enumerate environment
% \setlist{noitemsep}                    % Remove extra vertical spacing between items in lists
%                                        % \setlist{nosep} to leave no space around whole list

% %%%%%%%%%%%%%%%%%%%%%%%%%%%%%%%%%%%%%%%
% % Page numbering
% %%%%%%%%%%%%%%%%%%%%%%%%%%%%%%%%%%%%%%%
% \makepagestyle{ASU}
%   \makeevenfoot{ASU}{}{\thepage}{}
%   \makeoddfoot{ASU}{}{\thepage}{}

% %%%%%%%%%%%%%%%%%%%%%%%%%%%%%%%%%%%%%%%
% % Title page, formatting
% %%%%%%%%%%%%%%%%%%%%%%%%%%%%%%%%%%%%%%%
% \newlength{\savedfootskip}
% \setlength{\savedfootskip}{\footskip}
% \newcommand{\titlepagesetup}{%          % Page layout for title page
%   \changepage%                          % Adjustment to page dimensions: 
%     {\savedfootskip}%                   %   text height
%     {}%                                 %   text width
%     {}%                                 %   even-side margin
%     {}%                                 %   odd-side margin
%     {}%                                 %   column sep.
%     {}%                                 %   topmargin
%     {}%                                 %   headheight
%     {}%                                 %   headsep
%     {-\savedfootskip}%                  %   footskip
% }

% \newcommand{\closetitlepagesetup}{%     % Undo set up for title page
%   \changepage{-\savedfootskip}{}{}{}{}%
%     {}{}{}{\savedfootskip}%
% }

% \makeatletter                           % Do not modify this section; Enter info above
% \newcommand*{\titlepageASU}{
%   \titlepagesetup
%   \clearpage
%   \begin{center}
%   \SingleSpacing
%   \thispagestyle{empty}
%     \renewcommand*{\do}[1]{##1 \\[\baselineskip]} 
%     \dolistloop{\titlelines}
%     by \\[\baselineskip]
%     \Author \\[4\baselineskip]
%     A \documentname~Presented in Partial Fulfillment \\
%     of the Requirements for the Degree \\
%     \degreename \\
%     \vfill                              % Vertically center the portion below
%     Approved \defdate~by the \\
%     Graduate Supervisory Committee: \\[\baselineskip]
%     \renewcommand*{\do}[1]{##1, \chairlabel \\} 
%     \dolistloop{\committeechair} 
%     \renewcommand*{\do}[1]{##1 \\} 
%     \dolistloop{\committeemember}
%     \vfill                              % Vertically center the portion above
%     ARIZONA STATE UNIVERSITY \\[\baselineskip]
%     \graddate
%   \end{center}
%   \clearpage
%   \closetitlepagesetup
% }
% \makeatother

% %%%%%%%%%%%%%%%%%%%%%%%%%%%%%%%%%%%%%%%
% % Heading styles
% %%%%%%%%%%%%%%%%%%%%%%%%%%%%%%%%%%%%%%%
% % Note: memoir also has \book and \part commands; do not use these
% \makechapterstyle{ASU}{%                % Define chapter heading style
%   \renewcommand*{\chapterheadstart}{}   % Chapter title flush with top margin
%   \renewcommand*{\chapnamefont}%        % Set font for 'Chapter' or 'Appendix' 
%     {\normalfont}
%   \renewcommand*{\chapnumfont}%         % Set font for number in chapter headings
%     {\normalfont}
%   \renewcommand*{\afterchapternum}%     % Insert a double line break after 
%     {\\[\baselineskip]}                 %   chapter number
%   \renewcommand*{\chaptitlefont}%       % Set font for chapter title name
%     {\normalfont}
%   \setlength{\afterchapskip}{0pt}       % Set vertical space between chapter title and
%                                         %   first paragraph; equivalent to one line break
%                                         %   (vertical space = \afterchapskip + \baselineskip)
%                                         % Note: This \afterchapskip value is only used in 
%                                         %   front matter
%   \renewcommand*{\printchapternum}{%    % Center justify chapter number
%     \centering \chapnumfont %
%     \thechapter}                                        
%   \renewcommand*{\printchaptertitle}[1]%% Center justify 
%     {\expandafter\centering %           %   \MakeUppercase has issues; see here for some 
%     \expandafter\chaptitlefont %        %   details: https://tex.stackexchange.com/questions/35680/uppercase-in-newcommand
%     \expandafter\MakeUppercase %        %   Accented characters and some fonts may not
%     \expandafter{##1}}                  %   uppercase correctly; if that happens, just 
%                                         %   type the chapter title in uppercase
% }

% \setsecnumdepth{all}                    %~Enter the levels that you want to have numbered
%                                         %   (Default is to number all [5 levels deep].)

% \newcommand{\divisionbeforeskip}%       % Create default formatting for headings
%   {\baselineskip}
% \newcommand{\divisionindent}%
%   {0.5em}
% \newcommand{\divisionfont}{\normalfont} % Font must be \normalfont
% \newcommand{\divisionafterskip}%
%   {\baselineskip}

% \setbeforesecskip{\divisionbeforeskip}  % Apply default formatting to all heading levels
% \setsecindent{\divisionindent}          % Note: If you change \setsecnumdepth above, you 
% \setsecheadstyle{\divisionfont}         %   will need to set the indent for all lower 
% \setaftersecskip{\divisionafterskip}    %   levels to '0pt'; otherwise, they will be 
%                                         %   preceded by unnecessary space
% \setbeforesubsecskip{\divisionbeforeskip}
% \setsubsecindent{\divisionindent}
% \setsubsecheadstyle{\divisionfont}
% \setaftersubsecskip{\divisionafterskip}

% \setbeforesubsubsecskip{\divisionbeforeskip}
% \setsubsubsecindent{\divisionindent}
% \setsubsubsecheadstyle{\divisionfont}
% \setaftersubsubsecskip{\divisionafterskip}

% \setbeforeparaskip{\divisionbeforeskip}
% \setparaindent{\divisionindent}
% \setparaheadstyle{\divisionfont}
% \setafterparaskip{\divisionafterskip}

% \setbeforesubparaskip{\divisionbeforeskip}
% \setsubparaindent{\divisionindent}
% \setsubparaheadstyle{\divisionfont}
% \setaftersubparaskip{\divisionafterskip}

% %%%%%%%%%%%%%%%%%%%%%%%%%%%%%%%%%%%%%%%
% % Paragraph formatting
% %%%%%%%%%%%%%%%%%%%%%%%%%%%%%%%%%%%%%%%
% %\sloppybottom                          % Reduce the chances of widows
% \raggedbottom                           % Loosens vertical spacing requirements, so 
%                                         %   \sloppybottom doesn't make pages look bad; 
%                                         %   it also prevents large gaps in the middle of
%                                         %   pages and pushes them to the bottom of pages
% \indentafterchapter                     % Overrides the default which is not to indent 
%                                         %   the first paragraph in a chapter, but it 
%                                         %   looks odd in some places to not indent
%                                         %   paragraphs

% %%% List Titles %%%
% \renewcommand{\contentsname}%           % Set heading for each list
%   {Table of Contents}%                  %   Formatted as chapter headings by default, so
% \renewcommand{\listtablename}%          %   no additional heading formatting is needed
%   {List of Tables}
% \renewcommand{\listfigurename}% 
%   {List of Figures}

% %%% Depth %%%
% \settocdepth{subparagraph}              % Include 5 levels deep (all levels) in TOC

% %%% Fonts %%%
% \makeatletter% 
% \patchcmd{\l@part}%                     % Patch the command that writes part-level entries
%     {\cftpartfont {#1}}%                %   to the table of contents, so they are in 
%     {\normalfont \texorpdfstring{%      %   'normalfont' and uppercase
%       \uppercase{#1}}{{#1}} }%
%     {\typeout{Success: Patch %
%       'l@part' to uppercase %
%       part-level headings in the %
%       table of contents.}}%
%     {\typeout{Fail: Patch %
%       'l@part' to uppercase % 
%       part-level headings in the %
%       table of contents.}}%
% \makeatother%

% \makeatletter% 
% \patchcmd{\l@chapapp}%                  % Patch the command that writes chapter-level 
%     {\cftchapterfont {#1}}%             %   entries to the table of contents, so they are 
%     {\normalfont \texorpdfstring{%      %   in 'normalfont' and uppercase
%       \uppercase{#1}}{{#1}} }%
%     {\typeout{Success: Patch %
%       'l@chapapp' to uppercase %
%       part-level headings in the %
%       table of contents.}}%
%     {\typeout{Fail: Patch %
%       'l@chapapp' to uppercase %
%       part-level headings in the %
%       table of contents.}}%
% \makeatother%

% % If not using 'hyperref', use the following commands to adjust 'part' and 'chapter' 
% %   level headings in the TOC
% %\renewcommand*{\cftpartfont}%         % Uppercase 'part' and 'chapter' headings
% %  {\normalfont\MakeTextUppercase}     % Note: Sending \MakeTextUppercase to the TOC 
% %\renewcommand*{\cftchapterfont}%      %   conflicts with hyperref and breaks it!
% %  {\normalfont\MakeTextUppercase}%    

% %\usepackage{titlecaps}                  % Set up headline style for captions in the 
%                                         %   lists of tables and figures
%                                         % Note: ASU style guide does not provide 
%                                         %   comprehensive guidelines for headlines, so 
%                                         %   Chicago style for headline style is used
%                                         % Note: Last word in title is not explicitly 
%                                         %   capitalized; in general, these settings are
%                                         %   broadly correct, but captions should be 
%                                         %   reviewed to ensure they are being capitalized
%                                         %   properly
% % \Resetlcwords
% % \Addlcwords{a an the}                   % Leave articles lowercase
% % \Addlcwords{and but for or nor}         % Leave conjunctions lowercase
% % \Addlcwords{aboard about above across % % Leave all prepositions lowercase
% %   after against along amid among anti % %   (This is a [non-exhaustive] list of common 
% %   around as at before behind below %    %   one-word prepositions)
% %   beneath beside besides between %
% %   beyond but by concerning considering %
% %   despite down during except excepting %
% %   excluding following for from in %
% %   inside into like minus near of off %
% %   on onto opposite outside over past %
% %   per plus regarding round save since %
% %   than through to toward towards under %
% %   underneath unlike until up upon %
% %   versus vs via with within without}
% % \Addlcwords{ according\space{to} %      % Leave two-word conjunctions lowercase
% %   ahead\space{of} apart\space{from} %   %   (This is a [non-exhaustive] list of common 
% %   as\space{for} as\space{of} %          %   two-word prepositions.) 
% %   as\space{per} as\space{regards} %
% %   aside\space{from} astern\space{of} %
% %   back\space{to} because\space{of} %
% %   close\space{to} due\space{to} %
% %   except\space{for} far\space{from} %
% %   in\space{to} inside\space{of} %
% %   instead\space{of} left\space{of} %
% %   near\space{to} next\space{to} %
% %   on\space{to} opposite\space{of} %
% %   opposite\space{to} out\space{from} %
% %   out\space{of} outside\space{of} %
% %   owing\space{to} prior\space{to} %
% %   pursuant\space{to} rather\space{than} %
% %   regardless\space{of} right\space{of} %
% %   subsequent\space{to} such\space{as} %
% %   thanks\space{to} that\space{of} %
% %   up\space{to}} 
  
% % \renewcommand{\cfttableaftersnumb}%     % Put table captions in List of Tables in title
% %   {\titlecap}%                          %   case
% % \renewcommand{\cftfigureaftersnumb}%    % Put table captions in List of Figures in title
% %   {\titlecap}%                          %   case

% % \newcommand{\macrocapwrap}[1]{%         % Use this macro to place other macros inside 
% %   {\bgroup\bgroup{{#1}}\egroup\egroup}% %   captions, e.g., '\macrocapwrap{\ref{figure1}}'
% % }%                                      % Note: Necessary due to the 'titlecaps' package
% %                                         %   which modifies contents of captions

% \renewcommand*{\cftpartpagefont}%       % Use normal font for all page numbers
%   {\normalfont}
% \renewcommand*{\cftchapterpagefont}%
%   {\normalfont}
% \renewcommand*{\cftsectionpagefont}%
%   {\normalfont}
% \renewcommand*{\cftsubsectionpagefont}%
%   {\normalfont}
% \renewcommand*{\cftsubsubsectionpagefont}%
%   {\normalfont}
% \renewcommand*{\cftsubsubsectionpagefont}%
%   {\normalfont}
% \renewcommand*{\cftparagraphpagefont}%
%   {\normalfont}
% \renewcommand*{\cftsubparagraphpagefont}%
%   {\normalfont}
% \renewcommand*{\cftfigurepagefont}%
%   {\normalfont}
% \renewcommand*{\cfttablepagefont}%
%   {\normalfont}

% \cftpagenumbersoff{part}                % Turn off page numbers for 'part's, which are 
%                                         %   actually serving as headings within the TOC

% %%% Vertical Space %%%
% \setlength{\cftbeforepartskip}{0pt}     % Remove all additional vertical spacing so TOC
% \setlength{\cftbeforechapterskip}{0pt}  %   is double spaced uniformly
% \setlength{\cftbeforesectionskip}{0pt}
% \setlength{\cftbeforesubsectionskip}{0pt}
% \setlength{\cftbeforesubsubsectionskip}{0pt}
% \setlength{\cftbeforeparagraphskip}{0pt}
% \setlength{\cftbeforesubparagraphskip}{0pt}
% \setlength{\cftbeforefigureskip}{0pt}
% \setlength{\cftbeforetableskip}{0pt}

% \renewcommand{\insertchapterspace}{%    % By default, extra vertical space (10pt) is 
%   \addtocontents{lof}%                  %   inserted between tables and figures from 
%     {\protect\addvspace{0pt}}%          %   different chapters; remove this extra space.
%   \addtocontents{lot}%
%     {\protect\addvspace{0pt}}%
% }

% %%% Horizontal Space %%%
% \newlength{\levelindentincrement}       % Set indent to increase by the same amount for
% \setlength{\levelindentincrement}{2em}  %   each level in the TOC; don't adjust figure
% \newlength{\levelindent}                %   or table indents
% \setlength{\levelindent}%
%   {\levelindentincrement}
% \setlength{\cftchapterindent}%
%   {\levelindent}
% \addtolength{\levelindent}%
%   {\levelindentincrement}
% \setlength{\cftsectionindent}%
%   {\levelindent}
% \addtolength{\levelindent}%
%   {\levelindentincrement}
% \setlength{\cftsubsectionindent}%
%   {\levelindent}
% \addtolength{\levelindent}%
%   {\levelindentincrement}
% \setlength{\cftsubsubsectionindent}%
%   {\levelindent}
% \addtolength{\levelindent}%
%   {\levelindentincrement}
% \setlength{\cftparagraphindent}%
%   {\levelindent}
% \addtolength{\levelindent}%
%   {\levelindentincrement}
% \setlength{\cftsubparagraphindent}%
%   {\levelindent}
% \addtolength{\levelindent}%
%   {\levelindentincrement}

% \setlength{\cftchapternumwidth}%        % Decrease space between number and heading for
%   {0.85\cftchapternumwidth}             %   all heading levels
% \setlength{\cftsectionnumwidth}%
%   {0.85\cftsectionnumwidth}
% \setlength{\cftsubsectionnumwidth}%
%   {0.85\cftsubsectionnumwidth}
% \setlength{\cftsubsubsectionnumwidth}%
%   {0.85\cftsubsubsectionnumwidth}
% \setlength{\cftparagraphnumwidth}%
%   {0.85\cftparagraphnumwidth}
% \setlength{\cftsubparagraphnumwidth}%
%   {0.85\cftsubparagraphnumwidth}
% \setlength{\cftfigurenumwidth}%         % Figure has the same 'level' as 'chapter' in the
%   {\cftsectionnumwidth}                 %   figure list, so make the number spacing the 
%                                         %   same as for chapters. Increased to 'section'.
% \setlength{\cfttablenumwidth}%          % Table has the same 'level' as 'chapter' in the
%   {\cftsectionnumwidth}                 %   table list, so make the number spacing the 
%                                         %   same as for chapters. Increased to 'section'.


% %%% Leaders/dots %%%
% \renewcommand*{\cftdotsep}{1.7}         % Set distance between dots for all heading levels
% \renewcommand*{\cftchapterleader}%      % Turn on dots for 'chapter' level
%   {\normalfont\cftdotfill{\cftdotsep}}
% \makeatletter                           % Bring leader dots over to page number (no gap)
%   \renewcommand{\@pnumwidth}{1.55em}    %~Manually adjust
%   \renewcommand{\@tocrmarg}{2.55em}
% \makeatother

% %%% Printing List Titles and Headers in Content Lists
% % Table of Contents (TOC)
% \copypagestyle{ASUtoc}{ASU}%            % Page style for regular page in TOC
%   \makeevenhead{ASUtoc}%
%     {\leftmark}{}{Page}
%   \makeoddhead{ASUtoc}%
%     {\leftmark}{}{Page}

% \copypagestyle{ASUtocFirst}{ASU}%       % Custom page headers for first page of TOC 
%   \makeevenhead{ASUtocFirst}%           %    (print out the title)
%     {}%
%     {\printchaptertitle{\contentsname}}%
%     {} 
%   \makeoddhead{ASUtocFirst}%
%     {}%
%     {\printchaptertitle{\contentsname}}%
%     {}

% \renewcommand{\tocheadstart}{}%         % Usually content list titles are printed like 
%                                         %   chapter headings; empty that formatting 

% \renewcommand{\printtoctitle}[1]{}%     % Don't print TOC title using default method; 
%                                         %   it will be output in the header

% \renewcommand{\aftertoctitle}{%         % On the first page of the TOC, print out the
%   \thispagestyle{ASUtocFirst}%          %   TOC title using a custom page style and print 
%   \hfill Page\par%                      %   the heading for the page below in the regular
%   }%                                    %   textbox 
%                                         % Note: Need '\par' before lists; see here: https://tex.stackexchange.com/questions/49882/yet-another-perhaps-a-missing-item-error

% % List of Tables (LOT)
% \copypagestyle{ASUlot}{ASU}%            % Page style for regular page in list of tables
%   \makeevenhead{ASUlot}{Table}{}{Page}
%   \makeoddhead{ASUlot}{Table}{}{Page}

% \copypagestyle{ASUlotFirst}{ASU}%       % Custom page headers for first page of list of  
%   \makeevenhead{ASUlotFirst}%           %   tables (print out the title)
%     {}%
%     {\printchaptertitle{\listtablename}}%
%     {} 
%   \makeoddhead{ASUlotFirst}%
%     {}%
%     {\printchaptertitle{\listtablename}}%
%     {}

% \renewcommand{\lotheadstart}{}%         % Usually content list titles are printed like 
%                                         %   chapter headings; empty that formatting; 

% \renewcommand{\printlottitle}[1]{}%     % Don't print LOT title using default method; 
%                                         %   it will be output in the header

% \renewcommand{\afterlottitle}{%         % On the first page of the list of tables, print
%   \thispagestyle{ASUlotFirst}%          %   out the title using a custom page style and 
%   \underline{Table}\hfill Page\par}%    %   print heading below in regular textbox

% % List of Figures (LOF)
% \copypagestyle{ASUlof}{ASU}
%   \makeevenhead{ASUlof}{Figure}{}{Page}
%   \makeoddhead{ASUlof}{Figure}{}{Page}

% \copypagestyle{ASUlofFirst}{ASU}%       % Custom page headers for first page of list of  
%   \makeevenhead{ASUlofFirst}%           %   figures (print out the title)
%     {}%
%     {\printchaptertitle{\listfigurename}}%
%     {} 
%   \makeoddhead{ASUlofFirst}%
%     {}%
%     {\printchaptertitle{\listfigurename}}%
%     {}  

% \renewcommand{\lofheadstart}{}%         % Usually content list titles are printed like 
%                                         %   chapter headings; empty that formatting 

% \renewcommand{\printloftitle}[1]{}%     % Don't print LOF title using default method; 
%                                         %   it will be output in the header

% \renewcommand{\afterloftitle}{%         % On the first page of the list of figures, print
%   \thispagestyle{ASUlofFirst}%          %   out the title using a custom page style and 
%   \underline{Figure}\hfill Page\par}    %   print heading below in regular textbox

% %%% Page layout (dimensions) for Contents Lists
% \newlength{\verticalpush}               % Set up to change page dimensions for the table 
%                                         %   of contents
%                                         % Push everything down so all the content is still
%                                         %   1in from the top of the page, including the  
%                                         %   header, so the header is available for titles
%                                         %   on the first page of contents lists and then
%                                         %   the headings on subsequent pages 
% \setlength{\verticalpush}%              % Calculate difference between \headdrop and the 
%   {1.0in - \headdrop}                   %   total upper margin (1in), so you can push 
%                                         %   the top of the header down into the textbox

% \newcommand{\contentslistsetup}{%       % Set up for contents lists
%   \changepage%                          % Adjustment to page dimensions: 
%     {-\baselineskip}%                   %   text height
%     {}%                                 %   text width
%     {}%                                 %   even-side margin
%     {}%                                 %   odd-side margin
%     {}%                                 %   column sep.
%     {\verticalpush}%                    %   topmargin
%     {}%                                 %   headheight
%     {}%                                 %   headsep
%     {-\verticalpush+\baselineskip}%     %   footskip
% }

% \newcommand{\closecontentslistsetup}{%  % Undo set up for contents lists
%   \changepage{\baselineskip}{}{}{}{}%
%     {-\verticalpush}{}{}{\verticalpush-\baselineskip}%
% }

% % Content lists can also be output directly. If the following command were used, all the 
% %   headings would have to be output manually (i.e., can't rely on any memoir macros for 
% %   formatting or setting in contents lists headings and lists). It would be best to 
% %   create a custom macro, such as '\customtoc', to output headings and content lists 
% %   following the style guide. 
% %
% % \makeatletter
% %   \@starttoc{toc}
% % \makeatother

% % These pages partly explain why it's difficult to use 'afterpage' to change page layout 
% %   settings (essentially, it's because everything inside \afterpage has a local scope). 
% %   If it were possible to use 'afterpage' in that way, the content lists would  be 
% %   easier to format. A new page layout could be called after the first page of each 
% %   content  list. Instead, use page marks to get the layout required by the style guide. 
% % https://tex.stackexchange.com/questions/97126/attempts-to-manually-change-linewidth-ignored-by-latex
% % https://tex.stackexchange.com/questions/85729/page-styles-only-work-for-thispagestyle-under-afterpage

% %%%%%%%%%%%%%%%%%%%%%%%%%%%%%%%%%%%%%%%
% % Footnotes and Endnotes
% %%%%%%%%%%%%%%%%%%%%%%%%%%%%%%%%%%%%%%%
% \usepackage{chngcntr}                   % Modify counters (e.g., for figures, footnotes)
% \counterwithout*{footnote}{chapter}     % Make footnote numbering continuous throughout

% \providetoggle{useendnotes}
% \settoggle{useendnotes}{true}           %<Set to 'true' if you want to use endnotes
% \iftoggle{useendnotes}{%                % Use the command \pagenote to create endnotes
%                                         %   in the running text. They will be collected
%                                         %   and printed in a 'Notes' section at the end
%                                         %   of the document

%   \makepagenote                         % Required in preamble if using endnotes
%   \continuousnotenums                   % Numbering does *not* reset after each chapter
%   \renewcommand*{\pagenotesubhead}[3]{} % No subheads inside note list (default is to 
%                                         %   divide them by chapter)
%   \renewcommand*{\notenuminnotes}[1]%   % Remove extra space between note number and note
%     {\normalfont #1.}                   %   text
%   \renewcommand{\postnoteinnotes}%      % Double space *between* notes
%     {\par\vspace{\baselineskip}}
% }{}                                     % Do nothing here if not using endnotes

% %%%%%%%%%%%%%%%%%%%%%%%%%%%%%%%%%%%%%%%
% % Bibliography
% %%%%%%%%%%%%%%%%%%%%%%%%%%%%%%%%%%%%%%%
% \newcommand{\bibfilename}{library}%<Enter the name of the *.bib file containing the 
%                                         %   reference information for sources cited in
%                                         %   the text. God help you if you're doing 
%                                         %   citations manually. 
% \newcommand{\bibheading}{References}    %<Enter the heading for the references section:
%                                         %   'References', 'Works Cited', or 'Bibliography'

% \providetoggle{usebiblatex}             % True = a biblatex package is being used; 
%                                         %   False = 'natbib' is being used
% \settoggle{usebiblatex}{true}           %~Set to 'false' to use 'natbib' intead of 
%                                         %   biblatex; I strongly recommend using biblatex
%                                         %   because natbib is rather old and will break 
%                                         %   for innocuous things like underscores in URLs 
% \iftoggle{usebiblatex}{%                % Settings for citation package
% %                                       % Settings for 'biblatex' or a version of 
% %                                       %   'biblatex'
%   % \usepackage[authordate,%                 
%   %             backend=biber,%           % Recommend to use 'biber' instead of 'bibtex'
%   %             doi=only,%                % Avoid printing URLs
%   %             isbn=false]%              % Don't print ISBN numbers
%   %             {biblatex-chicago}        %~Other possibilities include: 'biblatex', 
%   %                                       %   'biblatex-apa', and 'biblatex-mla'

% \usepackage[  backend=biber,%           % Recommend to use 'biber' instead of 'bibtex'
%               isbn=false, %             % Don't print ISBN numbers 
%               url=false]%               % Don't print URLs          
%               {biblatex}                %~Other possibilities include: 'biblatex', 
%                                         %   'biblatex-apa', and 'biblatex-mla'                                        
  
%   \bibliography{\bibfilename}
%   \setlength{\bibitemsep}%              % Set vertical distance between 
%     {0.5\baselineskip}%                 %   bibliography entries 
%   \setcounter{biburlnumpenalty}{9000}   % Break URLs in bibliography across lines
%   \setcounter{biburlucpenalty}{9000}
%   \setcounter{biburllcpenalty}{9000}

%   \usepackage[style=american,%          % Settings for quotation marks; load after 
%     english=american]{csquotes}%        %   'inputenc'; only use with biblatex; throws 
%   \MakeOuterQuote{"}%                   %   error when used with natbib
% }{%                                     % Settings for 'natbib'
%   \usepackage{natbib}%
%   \newcommand{\natbibstyle}{asudis}%    %~Enter the name of the *.bst file to use to 
%                                         %   format citations with natbib. Default is 
%                                         %   'asudis'. I do not know where 'asudis' came
%                                         %   from, but apparently it formats citations
%                                         %   correctly because it was included with the 
%                                         %   previous LaTeX template.   
% }


% %%%%%%%%%%%%%%%%%%%%%%%%%%%%%%%%%%%%%%%
% % Tables and figures
% %%%%%%%%%%%%%%%%%%%%%%%%%%%%%%%%%%%%%%%
% \captiondelim{. }                       %~Use period (.) after caption number instead of 
%                                         %   colon (:). Change according to style guide. 
% \captionstyle[\centering]%              % Set justifcation for [one line captions] 
%   {\raggedright}                        %   and {multiple line captions}
% \setlength{\belowcaptionskip}{0pt}      % Bring caption down closer to figure/table

% % \makeatletter                           % Consecutive numbering throughout 
% %   \counterwithout{figure}{chapter}      %   (including back matter)
% %   \counterwithout{table}{chapter} 
% %   \renewcommand\@memfront@floats{} 
% %   \renewcommand\@memmain@floats{} 
% %   \renewcommand\@memback@floats{} 
% % \makeatletter

% %%% Tables %%%
% %
% % Note: 'memoir' natively supports commands from the following table-related packages: 
% %   tabularx, ccaption, booktabs.
% % Everyone has particular ideas about how tables should look, so you may need to 
% %   load additional packages and modify the code below to get tables (and figures) to 
% %   look the way you want them to. 
% %\setfloatadjustment{table}{\raggedright}
% \setfloatadjustment{table}{\centering}% Left justify material inside table floats
% \usepackage{tabu}                       % 'tabu' is an excellent table package; it can 
%                                         %   automatically size column widths and has a 
%                                         %   lot of customizations that other packages do
%                                         %   not. It also has a 'longtabu' environment that 
%                                         %   emulates 'longtable' with additional features
%                                         %   from the 'tabu' package. If you don't want 
%                                         %   to use it, you can comment this line out. 
% \BeforeBeginEnvironment{table}%         % Single space inside table environment
%   {\SingleSpacing}
% \AfterEndEnvironment{table}
%   {\DoubleSpacing}

% %%% Figures %%%
% \setfloatadjustment{figure}%            % Left justify material inside figure floats
%   {\centering}%{\raggedright}
% \BeforeBeginEnvironment{figure}%        % Single space inside figure environment
%   {\SingleSpacing}
% \AfterEndEnvironment{figure}
%   {\DoubleSpacing}

% \makeatletter                           % Define custom macro called '\maxwidth{}' that
%   \def\maxwidth#1{%                     %   allows you to specify the maximum width of an 
%     \ifdim%                             %   imported image. See below for an example.
%       \Gin@nat@width>#1 #1%             %   
%     \else%                              % Source: http://tex.stackexchange.com/questions/86350/includegraphics-maximum-width
%       \Gin@nat@width%
%     \fi}
% \makeatother
% %
% % Example \maxwidth: 
% %
% %   \includegraphics[width=\maxwidth]{\textwidth}]{image.pdf}
% % 
% % Note: This will keep an image inside the horizontal margins assuming the image starts 
% %   on the right margin (i.e., no horizontal space before the image). 

% %%%%%%%%%%%%%%%%%%%%%%%%%%%%%%%%%%%%%%%
% % Hyperref settings
% %%%%%%%%%%%%%%%%%%%%%%%%%%%%%%%%%%%%%%%

% %%% URL Settings %%%
% \PassOptionsToPackage{hyphens}{url}
% \usepackage[breaklinks=true]{hyperref}  % 'hyperref' should be loaded at the end of the 
%                                         %   preamble; Note: the uppercasing commands used 
%                                         %   throughout the preamble can conflict with it, 
%                                         %   especially when non-standard fonts or 
%                                         %   different file encodings are used
% \urlstyle{same}                         % Set URLs in the same font as regular text

% \tolerance 1414                         % Help URLs from entering margins
% \hbadness 1414                          %   Source: https://tex.stackexchange.com/questions/3033/forcing-linebreaks-in-url
% \emergencystretch 1.5em
% \hfuzz 0.3pt
% \widowpenalty=10000
% \vfuzz \hfuzz

% %%% Create metadata strings
% \usepackage{hyperxmp}                   % For metadata 
% \renewcommand*{\do}[1]{#1\ }%           % Build title string to output to pdf document
% \newcommand*{\onelinetitle}{%        
%   \dolistloop{\titlelines}%
% }
% \edef\theonelinetitle%
%   {\onelinetitle}

% \renewcommand*{\do}[1]{{#1}\ }%          % Build keyword string to output to pdf document
% \newcommand*{\pdfkeywordsstring}{%        
%   \dolistloop{\keywords}%
% }
% \edef\thepdfkeywordsstring%
%   {\pdfkeywordsstring}

% \newcommand*{\pdfcopyrightstring}%      % Build copyright message string
%   {Copyright \copyright\space\gradyear\ by \Author.%
%   {\space}All rights reserved.}

% \ifpdf                                  % Build pdf creator string (for pdfTeX)
%   \makeatletter
%   \def\extractpdftexversion#1-#2-#3 #4%
%     \@nil{#3}
%   \edef\pdfcreator{pdfTeX \expandafter%
%     \extractpdftexversion\pdftexbanner\@nil}
%   \makeatother
% \fi
% \ifxetex                                % Build pdf creator string (for XeTeX)
%   \edef\pdfcreator{XeTeX %
%     \the\XeTeXversion\XeTeXrevision}
% \fi

% \edef\pdfsummary{%                      % Build pdf summary
%   A \documentname Presented in\space
%   Partial Fulfillment of the\space
%   Requirements for a \degreename\space 
%   from Arizona State University}

% %%% Enter metadata and other settings
% \hypersetup{                            % Set pdf metadata
%   pdftitle={\theonelinetitle},          % Title
%   pdfauthor={\Author},                  % Author
%   pdfcreator={\pdfcreator},             % Enter the TeX writer for good documentation 
%  %pdfproducer={},                       % Let 'pdfproducer' be filled automatically
%   pdfsubject={\pdfsummary},             % Subject of the document
%   pdfkeywords=\thepdfkeywordsstring,    % List of keywords
%   hidelinks={true},                     % Links look like regular text (no colors, boxes)
%   breaklinks={true},                    % Allow links to break across lines
% }
% \ifxetex                                % If processing with XeTeX
%   \hypersetup{unicode=true}             % Must use 'true' in XeTeX
% \else
%   \hypersetup{unicode=true}             % Default is to use 'true' otherwise, as well
% \fi
% \ifpdf                                  % Copyright message; probably only works in pdfTeX
%   \hypersetup{
%     pdfcopyright={\pdfcopyrightstring},   
%     pdfinfo={%
%       Copyright=\pdfcopyrightstring%      
%     }%
%   }
% \fi

% \usepackage%
%   [numbered,%                           % Include numbers of sections in bookmarks
%   open%                                 % Bookmark tree already expanded when PDF opened
%   ]%
%   {bookmark}
% \bookmark[page=1,rellevel=0,%           % Create bookmark of title page at root level
%   keeplevel=true]{Title Page}
% \preto{\tableofcontents}{%              % Create bookmark for TOC 
%   \hypertarget{tocpage}{}%
%   \bookmark[dest=tocpage,rellevel=0,%
%     keeplevel=true]{\contentsname}%
% }

% %%%%%%%%%%%%%%%%%%%%%%%%%%%%%%%%%%%%%%%
% % Copyright page
% %%%%%%%%%%%%%%%%%%%%%%%%%%%%%%%%%%%%%%%
% \newcommand{\copyrightpageASU}{%          % Create copyright page
%   \thispagestyle{empty}
%   ~\\ \vfill
%   \parbox{\textwidth}{%
%     \begin{center}
%       \copyright\gradyear\space%
%       \Author\\%
%       All Rights Reserved%
%     \end{center}%
%   }%
%   \clearpage%
% }

% %%%%%%%%%%%%%%%%%%%%%%%%%%%%%%%%%%%%%%%
% % Sample settings
% %%%%%%%%%%%%%%%%%%%%%%%%%%%%%%%%%%%%%%%
% % \providetoggle{sample}                  % True = demonstration of template
% % \settoggle{sample}{false}
% %  \iftoggle{sample}{%
% %   \newcounter{tablecounter}
% %   \setcounter{tablecounter}{1}
% %   \newcounter{figurecounter}
% %   \setcounter{figurecounter}{1}
% % }{%
% % }

% %%%%%%%%%%%%%%%%%%%%%%%%%%%%%%%%%%%%%%%
% % Debugging Help
% %%%%%%%%%%%%%%%%%%%%%%%%%%%%%%%%%%%%%%%
% \usepackage{lipsum}                     % Outputs dummy text

% %%%%%%%%%%%%%%%%%%%%%%%%%%%%%%%%%%%%%%%%%%%%%%%%%%%%%%%%%%%%%%%%%%%%%%%%%%%%%%
% % Document
% %%%%%%%%%%%%%%%%%%%%%%%%%%%%%%%%%%%%%%%%%%%%%%%%%%%%%%%%%%%%%%%%%%%%%%%%%%%%%%
% \begin{document}

% %%%%%%%%%%%%%%%%%%%%%%%%%%%%%%%%%%%%%%%
% % Title page
% %%%%%%%%%%%%%%%%%%%%%%%%%%%%%%%%%%%%%%%
% \titlepageASU

% %%%%%%%%%%%%%%%%%%%%%%%%%%%%%%%%%%%%%%%
% % Copyright page
% %%%%%%%%%%%%%%%%%%%%%%%%%%%%%%%%%%%%%%%
% \copyrightpageASU                       %~If you don't want to have a copyright page, 
%                                         %   comment out this line

% %%%%%%%%%%%%%%%%%%%%%%%%%%%%%%%%%%%%%%%
% % Front matter
% %%%%%%%%%%%%%%%%%%%%%%%%%%%%%%%%%%%%%%%
% \chapterstyle{ASU}
% \pagestyle{ASU}
% \frontmatter

% \chapter*{Abstract}                     % Abstract is required
% Autonomic closure for large eddy simulations replaces the need for traditional prescribed subgrid closure models with an adaptive self-optimizing closure that solves a local, nonlinear, nonparametric system identification problem for each subgrid term at every point and time in a simulation. Here we develop accurate and efficient implementations of autonomic closure for the subgrid stress, with particular attention to the resulting detailed spatial structure of forward and backward scatter in the associated subgrid production fields, including metrics for assessing the scale-dependent support-density fields on which large values of subgrid production are concentrated. A relatively local, second-order, velocity-only, colocated implementation is found to produce subgrid stress and production fields that closely match not only statistics but also most details of the spatial structure in the exact fields at essentially all resolved scales, at a computational cost $O(10^3)$ lower than the original implementation. 

We then use the most general complete rank-2 tensor polynomial basis (Smith 1971) to determine a frame-invariant tensor formulation for autonomic closure of the subgrid stress in terms of the strain rate and rotation rate, S and R, and their gradients   $\mathbf{\nabla S}$  and   $\mathbf{\nabla R}$ . The rank-3 gradient tensors are contracted to form symmetric and anti-symmetric rank-2 tensors. The lack of an equivalent to the Cayley-Hamilton theorem for rank-3 tensors allows infinitely many such contractions, thus we limit these to the complete set of 19 unique second-order rank-2 contractions of the form   $\mathbf{\nabla S^2}$,   $\mathbf{\nabla S}$$\mathbf{\nabla R}$,and   $\mathbf{\nabla R^2}$. From these, together with I, S, and R, a complete and minimal symmetric tensor basis   $\mathbf{m}_{S}^{\alpha}$   is obtained. The resulting frame-invariant tensor polynomial for the turbulent stress   $\tau_{ij}$   involves 1570 terms, but when truncated to retain only terms up to second order in the velocity components  $u_i$  it involves only 21 terms. These 21 terms include all multi-point second-order velocity products, which in a velocity-only Volterra series representation required 3403 terms. This complete frame-invariant tensor formulation of autonomic closure uses all of the information available in the velocities on a  3 x 3 x 3 stencil, and does so with the smallest possible number of terms, while ensuring frame invariance.

Finally we use the general formulation of (Smith 1971) to obtain the invariance-preserving tensor representation for autonomic closure of the subgrid stress   directly in terms of the velocity vectors u at the 27 stencil points. This provides a more computationally efficient way of implementing an invariance-preserving representation of   in autonomic closure, since it uses exactly the same information that is used in the more complex representation in terms of S, R, and their gradients.  The resulting series has strong similarities with our prior truncated Volterra series representation, but does not involve the ad hoc assumption of a Volterra series representation. Interestingly, the general formulation of Smith (1971) shows that there can be no velocity component products higher than order-two in an invariance-preserving representation in terms of velocities at a point, which may eliminate the need for second-order truncation in a Volterra series. This new invariance-preserving should reduce the computational burden of autonomic closure by a factor of roughly 543 over the ad hoc Volterra series representation, and by a factor of 71 over the previous invariance-preserving representation in S, R, and their gradients.
                 %<Enter the name of the .tex file containing your
%                                         %   your abstract or omit this line and type in 
%                                         %   your abstract here. 

% %\chapter*{Dedication}                   %~Dedication is optional
% %\clearpage                             %~If you don't wish to display the heading 
%                                         %   'Dedication', comment out the previous line 
%                                         %   and use this one instead.
% %\input{dedication}               %<Enter the name of the .tex file containing your
%                                         %   your dedication or omit this line and type in 
%                                         %   your dedication here. 

% \chapter*{Acknowledgements}             %~Acknowledgements are optional
% Discussions with Dr. Ryan King are gratefully acknowledged. PEH acknowledges support from the U.S. Air Force Academy under Award No. FA7000-16-2-2003 and from NASA under Award No. NNX15AU24A.         %<Enter the name of the .tex file containing your
%                                         %   acknowledgements or omit this line and type in
%                                         %   your acknowledgements here. 

% \iftoggle{usemicrotype}                 % If 'microtype' is in use, turn off protrusion
%   {\microtypesetup{protrusion=false}}%  %   for TOC
%   {}
% \clearpage                              % Output table of contents on a new page
% \contentslistsetup                      % Change page layout for contents lists

% \pagestyle{ASUtoc}
% \tableofcontents*                       % Starred version leaves TOC heading out of TOC
% \addtocontents{toc}%                    % List of ... needs to be on left margin, but 
%   {\setlength{\cftchapterindent}%       %   they inherit 'chapter' formatting, so override
%     {0em}%
%   } 

% \clearpage
% \pagestyle{ASUlot}
% \listoftables                           % List of Tables should appear in TOC, so use 

%                                         %   unstarred version of \listoftables
% \clearpage
% \pagestyle{ASUlof}
% \listoffigures                          % List of Figures should appear in TOC, so use 
%                                         %   unstarred version of \listoffigures

% \phantomsection                         % \phantomsection is needed before using 
%                                         %   \addtocontents when it contains certain macros
%                                         %   when also using 'hyperref' package
% \addtocontents{toc}%                    % Undo manual override above for chapter indent,
%   {\setlength{\cftchapterindent}%       %   so actual chapters in the TOC are indented
%     {\levelindentincrement}%            %   correctly 
%   } 
% \setlength{\afterchapskip}%             % Set vertical space between chapter title and
%   {\baselineskip}                       %   first paragraph; equivalent to two line breaks

% \phantomsection
% \addcontentsline{toc}{part}{Chapter}    % Add "Chapter" to TOC here at 'part' level
% \phantomsection
% \addtocontents{toc}%                    % Add this 'mark' to TOC so subsequent pages use
%   {\protect\markboth{CHAPTER}{Page}}    %   the "CHAPTER" heading

% \iftoggle{usemicrotype}                 % If 'microtype' is in use, turn protrusion back 
%   {\microtypesetup{protrusion=true}}%   %   on
%   {}

% \clearpage                              % Note: All these changes have to be above a 
%                                         %   a '\clearpage' before '\mainmatter'

% \pagestyle{ASU}                         % Switch back to regular page style for remainder  
%                                         %   of the document
% \closecontentslistsetup                 % Undo page layout for contents lists

% %\chapter{Definitions}                   %~OTHER LISTS (optional)
% %\input{definitions}                     %<Enter the name of the .tex file or omit this  
%                                         %   line and type in here.

% %\chapter{Preface}                       %~PREFACE (optional, less than 10 pages)
% %\input{preface}                         %<Enter the name of the .tex file or omit this  
%                                         %   line and type in here.

% %%%%%%%%%%%%%%%%%%%%%%%%%%%%%%%%%%%%%%%
% % Body
% %%%%%%%%%%%%%%%%%%%%%%%%%%%%%%%%%%%%%%%
% \mainmatter
% \graphicspath{ {./Ch1/}  } 
\DeclareGraphicsExtensions{.png,.pdf,.jpg}
% %%%%%%%%%%%%%%%%%%%%%%%%%%%%%%%%%%%%%%%%%%%%%%%%%%%%%%%%%%%%%%%%%%%%%%%%%%%%%%%%%%%%%%%%%%
% %
% %  CITE GROUPS  %
% %%%%%%%%%%%%%%%%%

% % Complex flows [1-11]
% \newcommand{\complexflow}{tyacke2016large, schmitt2007large, neophytou2015large, baurle2016hybrid, poubeau2014large, xiao2013large, bini2008large, aubard2013large, loginov2006large, piomelli1999large, mahesh2006large}

% % Near wall treatment [12-15]
% \newcommand{\nearwall}{bose2014dynamic, piomelli2008wall, piomelli2010wall,kawai2012wall}

% % Conserved scalars [16-19]
% \newcommand{\conservedscalars}{dianat2006large, burton2008nonlinear, burton2011study, mejia2015large}

% % Droplets and particles [20-23]
% \newcommand{\particledrop}{elghobashi1994predicting, xiao2016large, irannejad2014large, de2013large}

% % Phase change [22,23]
% \newcommand{\phasechange}{irannejad2014large, de2013large}

% % Reacting species [24-26]
% \newcommand{\reactingspecies}{pitsch2006large, franchetti2013large, tafti2014large}

% % Heat transfer [24-26]
% \newcommand{\heattransfer}{pitsch2006large, franchetti2013large, tafti2014large}

% % Prescribed subgrid models[27-33]
% \newcommand{\prescribedsubgrid}{meneveau2000scale, liu1994properties, sagaut2006large, menon1996effect, shi2008constrained, tejada2004dynamic, yu2017scale}

% % Large eddy simluations
% \newcommand{\LES}{piomelli1999large, da2004effect, ghosal1996analysis, meyers2003database, kravchenko1997effect, chow2003further}



% %%%%%%%%%%%%%%%%%
% %

%%%%%%%%%%%%%%%%%%%%%%%%%%%%%%%%%%%%%%%%%%%%%%%%%%%%%%%%%%%%%%%%%%%%%%%%%%%%%%%%%%%%%%%%%%
\chapter{Introduction}

Large eddy simulation (LES) is being increasingly applied to 
%\cite{\allrefs} 
complex flows \cite{\complexflow}
as the increasing availability of computing power and its decreasing cost make the computational burden of LES more acceptable. At the same time, there have been technical advances in the underlying methodology, such as modern wall treatments \cite{\nearwall}, that are further reducing the computational cost of LES to acceptable levels. These developments are making multiphysics large eddy simulations of complex flows increasingly practical (\citeauthor*{tyacke2016large},\citeyear{tyacke2016large}), in which the simulations address not only the underlying turbulent flow but also include numerous other coupled physical processes, such as transport of conserved scalars \cite{\conservedscalars}, droplet and particle dynamics \cite{\particledrop}, phase changes \cite{\phasechange}, reacting species \cite{\reactingspecies}, heat transfer \cite{\heattransfer}, and other phenomena.  

Each physical process introduces governing equations, such as equations for conservation of mass, momentum, energy, and scalars, expressed as a combination of linear and nonlinear terms in the velocity field  $\mathbf{u}(\mathbf{x},t)$ and various scalar fields $\varphi(\mathbf{x},t)$. Due to the spatial filtering inherent in LES, each nonlinear term in these equations creates an associated subgrid term when the equations are written in the corresponding resolved fields $\widetilde{\mathbf{u}}(\mathbf{x},t)$ and $\widetilde{\varphi}(\mathbf{x},t)$. Each of these subgrid terms must be related to the resolved fields to obtain a closed set of equations.  Closure of subgrid terms has traditionally been done by means of prescribed subgrid models \cite{\prescribedsubgrid} that typically involve substantial ad hoc treatments. Errors introduced by these models can be important contributors to the overall error in results from large eddy simulations \cite{\LES}. For this reason, developing a general method that provides accurate subgrid closures for large eddy simulations has been a central focus area of turbulence research over the past several decades. 

We recently used the results of Smith (1971) to obtain the most general representation of the symmetric subgrid stress tensor $\tau_{ij}$  in terms of the strain and rotation rate tensors $\mathbf{S}$ and $\mathbf{R}$ and rank-two products up to second order of their gradients $\mathbf{\nabla S}$ and $\mathbf{\nabla R}$. In particular, Smith’s framework guarantees that the resulting tensor polynomial representation satisfies the rank, symmetry, rotation, and reflection properties of the stress tensor $\tau_{ij}$  – essential for any such representation to be strictly valid.  Moreover, Smith’s formulation was shown by Pennisi and Trovato (1987) to be minimal, thus there is no smaller number of symmetric basis tensors that can form a complete polynomial representation of $\tau_{ij}$.

The resulting representation, while complete and minimal, is computationally expensive to evaluate.  It consists of 1570 tensor terms that each involve products up to fourth order in the underlying rank-two tensors $\mathbf{M}_i$ and $\mathbf{W}_p$, each of which in turn involve products up to second order in $\mathbf{S}$, $\mathbf{R}$, $\mathbf{\nabla S}$ and $\mathbf{\nabla R}$. Since the strain and rotation rate tensors $\mathbf{S}$ and $\mathbf{R}$, and gradients $\mathbf{\nabla S}$ and $\mathbf{\nabla R}$, are linear in the components of the velocities $\mathbf{u}_m$  at the $m = 1, 2, \,\dots\, , P = 27$ points on the local  stencil, each term in this frame-invariant representation ultimately involves products up to $eighth$ order in these velocity components.  

One might suspect that some of the many resulting velocity component products appearing in each of the 1570 tensor bases may be the same. Leaving them grouped as they appear in this 1570-term tensor polynomial form involves the smallest number of coefficients of any complete frame-invariant symmetric tensor representation. However, that may be less computationally efficient than if these repeated velocity component products were instead grouped together in an equivalent series having more coefficients but involving far fewer arithmetic operations to evaluate all components of all the basis tensors. However, due to the enormous number of such repeated velocity component products in this tensor polynomial, identifying such alternate groupings “by hand” is completely impractical, and even our efforts to use symbolic mathematics software to identify repeated velocity component products have not led to success.

Alternatively, it may be possible to use the results of Smith (1971) to obtain a different, and potentially more computationally efficient, complete and minimal tensor polynomial in terms of the P = 27 velocity vectors $\mathbf{u}_m$ on our 3 x 3 x 3   stencil, rather than via the 1570-term tensor polynomial in $\mathbf{S}$, $\mathbf{R}$, $\mathbf{\nabla S}$ and $\mathbf{\nabla R}$. Smith’s formulation in fact provides the complete and minimal tensor bases for representing any symmetric rank-two tensor in terms of any number of vectors and rank-two tensors.  Rather than forming the tensor polynomial representation in terms of rank-two tensors from $\mathbf{S}$, $\mathbf{R}$, $\mathbf{\nabla S}$ and $\mathbf{\nabla R}$, we will here use Smith’s formulation to obtain the tensor polynomial representation directly in terms of the $m = 1, \,\dots\, , P = 27$ velocity vectors   on the stencil.  Such a series will naturally have each possible velocity component product appear only once, and thus should provide the potentially efficient alternative representation noted above.

In fact, as we will see, this $\tau_{ij}$ representation in terms of tensor bases formed directly from the $\mathbf{u}_m$  will be very similar to the original $\emph{ad hoc}$ Volterra series we began with.  However, while the original Volterra series representation was not frame invariant, this new representation will be. Moreover, our original $\emph{ad hoc}$ Volterra series was arbitrarily truncated after second-order products in the velocity components, but in this new representation we will see that in any valid representation of $\tau_{ij}$  in terms of $\mathbf{u}_m$ there cannot be any velocity component products of order higher than two. Thus in this new representation no truncation is needed, even while preserving the completeness of the tensor representation.

We may anticipate that the resulting new polynomial in the basis tensors formed from the   will have more terms than does our recent 1570-term polynomial in the basis tensors formed from $\mathbf{S}$, $\mathbf{R}$, $\mathbf{\nabla S}$ and $\mathbf{\nabla R}$. Yet this new polynomial may nevertheless be more computationally efficient, since each component of its basis tensors involve only a simple second-order velocity component product, and we are guaranteed that there are no repeated products in this representation. 

%%%%%%%%%%%%%%%%%%%%%%%%%%%%%%%%%%%%%%%%%%%%%%%%%%%%%%%%%%%%%%%%%%%%%%%%%%%%%%%%%%%%%%%%%%
% \section{Conventional Turbulence Modeling Approaches} 
% \lipsum[1] 

% \subsection{The Reynolds-Averaged Navier-Stokes (RANS) Equations} 
% \lipsum[1] 

% \subsection{Large Eddy Simulation}
% \lipsum[2]

% \subsubsection{Scale-Similarity Models}
% \lipsum[1]

% \section{Non-Parametric Closure Approaches}
% \lipsum[1] 

% \section{Autonomic Closure: A Novel Non-Parametric Mechanism}
% \lipsum[1] \cite{krishnappa_adult_2012}

% \section{Present Study}
% \lipsum[1]

% \begin{figure}[htbp]
% \begin{center}
% \includegraphics[width=4in]{antidorcas}
% \caption{default}
% \label{default}
% \end{center}
% \end{figure}


% \subsection{Organization of this Dissertation}
% \lipsum[1]
%%%%%%%%%%%%%%%%%%%%%%%%%%%%%%%%%%%%%%%%%%%%%%%%%%%%%%%%%%%%%%%%%%%%%%%%%%%%%%%%%%%%%%%%%%


                 %<Insert your chapters here; I recommend to use
% \graphicspath{ {./Ch2/}  } 
\DeclareGraphicsExtensions{.png,.pdf,.jpg}

%%%%%%%%%%%%%%%%%%%%%%%%%%%%%%%%%%%%%%%%%%%%%%%%%%%%%%%%%%%%%%%%%%%%%%%%%%%%%%%%%%%%%%%%%%
%
%  CITE GROUPS  %
%%%%%%%%%%%%%%%%%

% % Subgrid stress [27-33]
% \newcommand{\subgridstress}{meneveau2000scale, liu1994properties, sagaut2006large, menon1996effect, shi2008constrained, tejada2004dynamic, yu2017scale}

% % LES [10, 34-38]
% \newcommand{\LargeEddy}{piomelli1999large, da2004effect, ghosal1996analysis, meyers2003database, kravchenko1997effect, chow2003further}

% % Spatial filter [29, 39-41] in Fig
% \newcommand{\spatialfilter}{sagaut2006large, germano1992turbulence, pope2000, lund1995experiments}

% % Subgrid Terms [27-34, 42-57]
% \newcommand{\subgridterms}{meneveau2000scale, liu1994properties, sagaut2006large, menon1996effect, shi2008constrained, tejada2004dynamic, yu2017scale, da2004effect, smag1963, bardina1980improved, bardina1983improved, horiuti1989role, germano1991dynamic, lilly1992proposed, moin1991dynamic, piomelli1993high, zang1993dynamic, akhavan2000subgrid, liu1995experimental, liu1999evolution, sarghini1999scale, lund1993numerical, ghosal1995dynamic, gravemeier2006consistent}

% %  Subgrid Stress [27-30,52-54]
% \newcommand{\sgstwo}{meneveau2000scale, liu1994properties, sagaut2006large, menon1996effect, liu1995experimental, liu1999evolution, sarghini1999scale}

% % DS [27,46-49]
% \newcommand{\DSmag}{meneveau2000scale, germano1991dynamic, lilly1992proposed, moin1991dynamic, piomelli1993high}

% % Scale-similarity [27, 43-45]
% \newcommand{\ScaleSim}{bardina1980improved, bardina1983improved, horiuti1989role, meneveau2000scale}

% % Mixed models [50-57]
% \newcommand{\mixed}{zang1993dynamic, akhavan2000subgrid, liu1995experimental, liu1999evolution, sarghini1999scale, lund1993numerical, ghosal1995dynamic, gravemeier2006consistent}

% % Model errors [10,27-30,34]
% \newcommand{\modelerrors}{piomelli1999large,meneveau2000scale,liu1994properties,sagaut2006large,menon1996effect,da2004effect}

% % Diffusion-limited chemical reactions [2-5,24-26]
% \newcommand{\chemical}{schmitt2007large, neophytou2015large,baurle2016hybrid,poubeau2014large, pitsch2006large, franchetti2013large, tafti2014large}

% % Particle/droplet agglomeration [6,7, 20-23]
% \newcommand{\particledropagg}{xiao2013large, bini2008large, elghobashi1994predicting, xiao2016large, irannejad2014large, de2013large}

% % Autonomic closure [58,59]
% \newcommand{\autonomic}{king2015autonomic, king2016autonomic}

% % Intermittent [10,27-30,52-54,60]
% \newcommand{\intermittent}{piomelli1999large, meneveau2000scale, liu1994properties, sagaut2006large, menon1996effect,liu1995experimental, liu1999evolution, sarghini1999scale, piomelli1991subgrid}

% % Backscatter [27,29,40], Mixed-models [50-57]
% \newcommand{\backscatter}{meneveau2000scale, sagaut2006large, pope2000, \mixed}

% % Accurate exchange [27-30, 42]
% \newcommand{\accuratexchange}{meneveau2000scale, liu1994properties, sagaut2006large, menon1996effect, smag1963}

% % Unstable backscatter [35-37,48]
% \newcommand{\unstablebackscatter}{ghosal1996analysis, meyers2003database, kravchenko1997effect, moin1991dynamic}

% % Hamlington-Dahm [61,62]
% \newcommand{\hamlingtondahm}{hamlington2008reynolds, hamlington2009nonlocal}

% % Prescribed model [27,46,47]
% \newcommand{\prescribed}{meneveau2000scale, germano1991dynamic, lilly1992proposed}

% % Data-driven [63-71]
% \newcommand{\datadriven}{zhang2015machine, duraisamy2016informing, singh2017augmentation, ling2015evaluation, ling2016reynolds, ling2016machine, wang2016data, wang2017physics, wu2017priori}

%%%%%%%%%%%%%%%%%
%

%%%%%%%%%%%%%%%%%%%%%%%%%%%%%%%%%%%%%%%%%%%%%%%%%%%%%%%%%%%%%%%%%%%%%%%%%%%%%%%%%%%%%%%%
\chapter{Volterra Series Representations}
\label{ch:2}

Large eddy simulation (LES) is being increasingly applied to complex flows \cite{\complexflow}, as the increasing availability of computing power and its decreasing cost make the computational burden of LES more acceptable. At the same time, there have been technical advances in the underlying methodology, such as modern wall treatments \cite{\nearwall}, that are further reducing the computational cost of LES to acceptable levels. These developments are making multiphysics large eddy simulations of complex flows increasingly practical, in which the simulations address not only the underlying turbulent flow but also include numerous other coupled physical processes, such as transport of conserved scalars \cite{\conservedscalars}, droplet and particle dynamics \cite{\particledrop}, phase changes \cite{\phasechange}, reacting species \cite{\reactingspecies}, heat transfer \cite{\heattransfer}, and other phenomena.  

Each physical process introduces governing equations, such as equations for conservation of mass, momentum, energy, and scalars, expressed as a combination of linear and nonlinear terms in the velocity field   $\mathbf{u}(\mathbf{x},t)$   and various scalar fields $\varphi(\mathbf{x},t)$. Due to the spatial filtering inherent in LES, each nonlinear term in these equations creates an associated subgrid term when the equations are written in the corresponding resolved fields   $\widetilde{\mathbf{u}}(\mathbf{x},t)$   and   $\widetilde{\varphi}(\mathbf{x},t)$. Each of these subgrid terms must be related to the resolved fields to obtain a closed set of equations.  Closure of subgrid terms has traditionally been done by means of prescribed subgrid models \cite{\subgridstress} that typically involve substantial ad hoc treatments. Errors introduced by these models can be important contributors to the overall error in results from large eddy simulations \cite{\LargeEddy}. For this reason, developing a general method that provides accurate subgrid closures for large eddy simulations has been a central focus area of turbulence research over the past several decades. 

%%%%%%%%%%%%%%%%%%%%%%%%%%%%%%%%%%%%%%%%%%%%%%%%%%%%%%%%%%%%%%%%%%%%%%%%%%%%%%%%%%%%%%%%%%
\section{Subgrid terms and their closure in LES}

In general, any governing equation can be written as a sum of linear terms  $L(\mathbf{u},\varphi)$  and nonlinear terms  $N(\mathbf{u},\varphi)$ as
%
%    EQUATION   %  
%%%%%%%%%%%%%%%%%
\begin{equation}
\label{E:1}
	L(\textbf{u},\varphi) + N(\textbf{u},\varphi) = 0
\end{equation}
%%%%%%%%%%%%%%%%%
%
%
Applying a suitable spatial filter  $\widetilde{(\,)}$  \cite{\spatialfilter} having characteristic length scale   $\widetilde{\Delta}$  then gives the corresponding governing equation in the resolved fields   $\widetilde{\mathbf{u}}(\mathbf{x},t)$   and   $\widetilde{\varphi}(\mathbf{x},t)$ as
%
%    EQUATION   % 
%%%%%%%%%%%%%%%%%
\begin{equation}
\label{E:2}
	L(\widetilde{\textbf{u}},\widetilde{\varphi}) + N(\widetilde{\textbf{u}},\widetilde{\varphi})
	= - \big[ \widetilde{N(\textbf{u},\varphi)} - N(\widetilde{\textbf{u}},\widetilde{\varphi}) \big],
\end{equation}
%%%%%%%%%%%%%%%%%
%
%
where the right side in (\ref{E:2}) are subgrid terms that appear because the linearity of  $L(\mathbf{u},\varphi)$  allows  $\widetilde{L(\textbf{u},\varphi)} = L(\widetilde{\textbf{u}},\widetilde{\varphi})$  while the nonlinearity of  $N(\mathbf{u},\varphi)$  leads to  $\widetilde{N(\textbf{u},\varphi)} \neq N(\widetilde{\textbf{u}},\widetilde{\varphi})$. All such subgrid terms must be dealt with in a way that provides a closed set of governing equations in the resolved variables  $\widetilde{\mathbf{u}}$  and  $\widetilde{\varphi}$. To date, such closure has been achieved by introducing prescribed subgrid models based on various approximations that relate subgrid terms to parameters that are obtainable from the resolved variables. Many such prescribed subgrid models have been proposed for subgrid terms in LES \cite{\subgridterms}. Errors from these prescribed models, as revealed for instance in $a priori$ tests, can be substantial even for subgrid terms that are fundamental to LES, such as the subgrid stress \cite{\sgstwo}.  

Taking the subgrid stress as an example, in the original momentum equation the nonlinear product  $\mathbf{u}_i\mathbf{u}_j$  in the advection term $\partial (u_i u_j)/\partial x_j$   leads via (\ref{E:2}) to a subgrid stress of the form 
%
%   EQUATION    % 
%%%%%%%%%%%%%%%%%
\begin{equation}
\label{E:3}
	\big[ \widetilde{N(\textbf{u})} - N(\widetilde{\textbf{u}}) \big] =
	 \widetilde{u_i u_j} - \widetilde{u_i}\widetilde{u_j} 
	 \equiv \tau_{ij}
\end{equation}
%%%%%%%%%%%%%%%%%
%
%
in the resolved momentum equation. Widely used models for the subgrid stress  $\tau_{ij}$  include the basic Smagorinsky model \cite{smag1963}, the dynamic Smagorinsky model \cite{\DSmag}, the scale-similarity model \cite{\ScaleSim}, and mixed models that combine a scale similarity model with a dissipative model \cite{\mixed}. All of these produce substantial errors in their representation of  $\tau_{ij}(\mathbf{x},t)$,  as has been shown in  $a priori$  tests \cite{\modelerrors}. The accuracy with which any such subgrid model represents  $\tau_{ij}(\mathbf{x},t)$  from the resolved variables  $\widetilde{\mathbf{u}}(\mathbf{x},t)$  and  $\widetilde{p}(\mathbf{x},t)$  determines how accurately it accounts for the detailed space- and time-varying momentum exchange and associated kinetic energy exchange between the resolved and subgrid scales in a simulation. 

If simulating the flow field were the only objective, then continued reliance on such traditional prescribed subgrid stress models might be acceptable, since the filter scale  $\widetilde{\Delta}$  could simply be made sufficiently small (albeit at greater computational cost) so that errors introduced by inaccuracies from the  $\tau_{ij}(\mathbf{x},t)$  model would not substantially affect much larger scales of the flow. However, LES is increasingly being used to simulate not only the flow field  $\widetilde{\mathbf{u}}(\mathbf{x},t)$,  but also other physical processes occurring in the flow, many of which depend strongly on the smallest scales in the resolved flow field, such as diffusion-limited chemical reactions \cite{\chemical} and droplet/particle transport and agglomeration \cite{\particledropagg}. In such cases, errors introduced at the smallest resolved scales from a substantially inaccurate  $\tau_{ij}$ model can create large errors throughout the resolved fields of primary interest. Achieving high fidelity in such multiphysics simulations may therefore require new approaches for representing subgrid terms, including the subgrid stress, that are substantially more accurate than current prescribed subgrid modeling approaches. 

%
%    FIGURE     % 
%%%%%%%%%%%%%%%%%
\begin{figure}
	\begin{center}
	\includegraphics[width=\maxwidth{3.5in}]{Fig1_y129_22.eps}
	\caption{Typical   $a priori$   test of autonomic closure, showing (a) true subgrid stress field $\tau_{ij}(\mathbf{x},t)$   and (c) true subgrid production field  $P(\mathbf{x},t)$ compared to corresponding results from implementation of autonomic closure [59] for (b) subgrid stress field  $\tau_{ij}^{\mathcal{F}}(\mathbf{x},t)$  and (d) subgrid production field  $P^{\mathcal{F}}(\mathbf{x},t)$.}
	\label{Fig:1}
	\end{center}
\end{figure}
%%%%%%%%%%%%%%%%%
%
%

It will be shown here that a recently proposed alternative approach \cite{\autonomic} to subgrid closure, referred to as “autonomic closure”, can be implemented in computationally efficient ways to enable representation of subgrid fields with significantly greater accuracy across all resolved scales than is possible with traditional prescribed subgrid closure models. Figure 1 shows typical results from  $a priori$  tests of an implementation \cite{king2016autonomic} of autonomic closure, comparing true subgrid stress fields  $\tau_{ij}(\mathbf{x},t)$  and associated subgrid kinetic energy production fields  $P(\mathbf{x},t)$  to the corresponding results from autonomic closure. The implementation of autonomic closure in Fig. \ref{Fig:1}, while undeniably accurate, is however far too computationally costly for practical use. Here we describe autonomic closure in detail and identify specific implementations that retain comparable accuracy in  $\tau_{ij}(\mathbf{x},t)$  and  $P(\mathbf{x},t)$  as seen in Fig. \ref{Fig:1}, even near the smallest resolved scales, but unlike the implementation in Ref. \cite{king2016autonomic} are computationally efficient enough for practical use in large eddy simulations. 

%%%%%%%%%%%%%%%%%%%%%%%%%%%%%%%%%%%%%%%%%%%%%%%%%%%%%%%%%%%%%%%%%%%%%%%%%%%%%%%%%%%%%%%%%%
\section{Subgrid stress closures and LES energetics} 

With regard to resolved-scale energetics and computational stability, even more important than the subgrid stress itself is the corresponding subgrid kinetic energy production field 
%
%   EQUATION    % 
%%%%%%%%%%%%%%%%%
\begin{equation}
\label{E:4}
	P(\mathbf{x},t) = -\tau_{ij}\widetilde{S}_{ij},
\end{equation}
%%%%%%%%%%%%%%%%%
%
%
where  $\widetilde{S}_{ij}$  is the resolved strain rate tensor, since this determines both the accuracy of energy exchange between the resolved and subgrid scales and the computational stability of the simulation itself. It is known from  $a priori$  tests 
\cite{\intermittent} that true  $P(\mathbf{x},t)$  fields in turbulent flows are highly intermittent, consisting of widely varying values that can be locally positive or negative, with magnitudes of $P$ in highly concentrated regions far exceeding the true average subgrid dissipation rate  $\langle P(\mathbf{x},t)\rangle \equiv \epsilon$. Large positive $P$ values concentrated in these regions correspond to local instantaneous kinetic energy transfer from the resolved scales into the subgrid scales (“forward scatter”), while negative values give the local rate of energy transfer from subgrid scales into the resolved scales (“backscatter”). 

Such large magnitudes of forward and backward scatter in  $P(\mathbf{x},t)$  can be seen for example in Fig. \ref{Fig:1}c, which shows the strong spatial intermittency that is characteristic of subgrid production fields. Large positive (red) and large negative (blue) $P$ values are clustered in relatively compact regions that occupy a small fraction of the domain in which the most intense forward and backward scatter are concentrated. For a  $\tau_{ij}(\mathbf{x},t)$  closure to accurately represent the precise space- and time-varying exchange of momentum and energy between resolved and subgrid scales, including near the smallest scales, it must allow forward and backward scatter in  $P(\mathbf{x},t)$  while providing the correct statistical distribution of $P$ values and accurately representing the highly intermittent regions in which large positive and negative  $P(\mathbf{x},t)$  values are concentrated.

Some  $\tau_{ij}$  models that allow for backscatter can induce instability in a simulation if their resulting  $P(\mathbf{x},t)$  fields are insufficiently accurate \cite{\backscatter}. This can occur if the average subgrid dissipation rate  $\langle P(\mathbf{x},t)\rangle$  is too low relative to the true average rate  $\epsilon$  of energy transfer into the subgrid scales. For this reason, scale-similarity models and other models are often combined with a purely dissipative model to give a sufficiently large average subgrid dissipation rate to maintain computational stability. However, even when the average subgrid dissipation rate is sufficiently large, a subgrid model could still induce instability if it produces incorrectly large local values of backscatter, or if the regions in which large values of backscatter are concentrated occur in the wrong locations or at the wrong times, or persist for too long. At the same time, the  $\tau_{ij}(\mathbf{x},t)$  closure must also produce the correct values of forward scatter in the correct locations at the correct times and for the correct durations. 

Subgrid models that are purely dissipative, such as the basic Smagorinsky model \cite{smag1963}, ensure stability but are unable to accurately represent the detailed momentum and energy exchange between resolved and subgrid scales in a simulation \cite{\accuratexchange}. The dynamic Smagorinsky model \cite{\DSmag} and various scale similarity models \cite{\ScaleSim} include backscatter to increase simulation fidelity, especially near the smallest resolved scales. However, some of these models produce insufficiently large  $\langle P(\mathbf{x},t)\rangle$   to maintain computational stability, and others may lead to instability if the modeled backscatter is too strong or appears at the wrong locations or the wrong times \cite{\unstablebackscatter}. For this reason, some of these models introduce backscatter limiters and other ad hoc adjustments to increase subgrid dissipation in order to ensure stable simulations.

Yet it is tautological that if a $\tau_{ij}$ closure exactly produces the complete details of the true  $\tau_{ij}(\mathbf{x},t)$  and  $P(\mathbf{x},t)$  fields in  $a priori$  tests, then in the absence of numerical errors from the LES code [35-38] the closure will be stable despite the required large backscatter. Presumably a  $\tau_{ij}$   closure can be less than perfect in this respect and still maintain stability. However, beyond the requirement that $\langle P(\mathbf{x},t)\rangle = \epsilon$ , relatively little is known about how accurately the subgrid stress $\tau_{ij}(\mathbf{x},t)$  must be represented to avoid backscatter instability while providing high fidelity in the detailed momentum and energy transfer even near the smallest scales of a simulation.

Although backscatter may be needed to achieve simulation accuracy in all resolved scales, the presence of backscatter alone is meaningless unless the backscattered energy is introduced in about the right places and the right times, and at the right magnitudes and for the right durations. Due to the highly intermittent nature of  $P(\mathbf{x},t)$  fields, as seen in Fig. \ref{Fig:1}c, it may not be possible in an $a priori$ sense to have exact point-by-point agreement between the true subgrid production field and that resulting from a closure for the subgrid stress. However, the subgrid production field from a subgrid stress closure should nevertheless be structurally similar to the true subgrid production field, having large values of forward and backward scatter concentrated in regions at the same locations and of the same size and shape as in the true  $P(\mathbf{x},t)$  field. With the ergodic hypothesis this also ensures that the closure will produce concentrations of forward and backward scatter at the correct times and for the correct durations.

Based on these considerations, it is expected that accuracy across all resolved scales can be achieved while maintaining computational stability if the following three conditions are met:
%
%    NUM LIST   % 
%%%%%%%%%%%%%%%%%
\begin{enumerate} 
	%
	\item {The  $\tau_{ij}$  closure should produce  $P(\mathbf{x},t)$  fields that, in $a priori$ tests, provide a sufficiently large average subgrid dissipation rate, namely $\langle P(\mathbf{x},t)\rangle \geq \epsilon$, and should ideally give  $\langle P(\mathbf{x},t)\rangle = \epsilon$.}
	%
	\item {Resulting  $P(\mathbf{x},t)$  fields from the   $\tau_{ij}$   closure should produce similar statistical distributions of positive and negative values (forward and backward scatter) as do the true $P(\mathbf{x},t)$  fields.}
	%
	\item {$P(\mathbf{x},t)$  fields from the  $\tau_{ij}$  closure should be structurally similar to the true  $P(\mathbf{x},t)$  fields in   $a priori$   tests, with large magnitudes of forward and backward scatter concentrated in regions at the right spatial locations and of the right size and shape, despite the highly intermittent nature of  $P(\mathbf{x},t)$  preventing the two fields from being exactly identical on a point-by-point basis.}
	%
\end{enumerate}
%%%%%%%%%%%%%%%%%
%
%
Although   $a priori$   tests of  $P(\mathbf{x},t)$  alone cannot determine if a closure will provide stable simulations, such tests are the most direct way to assess the accuracy at all resolved scales in the subgrid stress fields  $\tau_{ij}$   and the associated subgrid production fields   from a given closure. Implementing a closure in an LES code for a posteriori tests introduces additional effects from the code that can obscure insights into the underlying accuracy of the subgrid closure \cite{meneveau2000scale}. For these reasons,   $a priori$   tests are used here to assess the accuracy of various implementations of autonomic closure in representing   $\tau_{ij}$   and   $P(\mathbf{x},t)$  fields. For implementations that are found to be accurate in these tests, subsequent   $a posteriori$   tests can be conducted to determine their stability when implemented in an LES code.

Particular attention is paid here not only to the resulting statistical distributions of forward and backward scatter in   $P(\mathbf{x},t)$,   but also to the detailed spatial structure of regions in which large magnitudes of forward and backward scatter are concentrated. Results in the following sections show that efficient implementations of autonomic closure can represent momentum and energy exchange between resolved and subgrid scales, across essentially all resolved scales, far more accurately than do traditional prescribed subgrid closure models. 


%%%%%%%%%%%%%%%%%%%%%%%%%%%%%%%%%%%%%%%%%%%%%%%%%%%%%%%%%%%%%%%%%%%%%%%%%%%%%%%%%%%%%%%%%%
\section{A new approach: Autonomic closure} 

An entirely different approach to subgrid closures, termed “autonomic closure”, was recently proposed \cite{\autonomic} to circumvent the need to specify a particular fixed parametric closure relation, and instead allow a nonparametric fully-adaptive self-optimizing closure methodology. The closure is autonomic in the sense that the simulation itself determines the optimal relation at each point and time between the subgrid term and the primitive variables through solution of a local, nonparametric system identification problem. The closure is nonparametric in the sense that the subgrid term is formulated in the resolved primitive variables of the simulation, rather than in parameters formed from them that are presumed to be appropriate. The resulting large number of degrees of freedom allows autonomic closure to freely adapt to varying nonlinearity, nonlocality, nonequilibrium, and other characteristics \cite{\hamlingtondahm} of the turbulence state at each point in the flow. 

Autonomic closure can be regarded as a high-dimensional nonparametric generalization of the dynamic approach used with various traditional prescribed closure models \cite{\prescribed}. Viewed another way it can be regarded as a type of “data-driven” turbulence closure \cite{\datadriven}, in which machine-learning methods are used with available prior data to discover a closure model rather than prescribe one. However, unlike other data-driven approaches, the training data in autonomic closure is obtained internally at a test-filter scale at each point and time in the simulation itself, rather than being provided separately from prior simulations or experiments. Importantly, autonomic closure is not a closure model; instead it is a closure methodology that enables essentially model-free “on the fly” closure of any subgrid term.

The need in autonomic closure to solve a local system identification problem at each point and time in a simulation can make its computational cost far higher than that of traditional prescribed closure models. That is certainly the case when the number of degrees of freedom in the nonparametric relation is large; e.g., the implementation in Ref. [59] involved nearly 6000 degrees of freedom. Some additional computational cost is acceptable in order to gain the increased accuracy in   $\tau_{ij}$   and   $P(\mathbf{x},t)$   that autonomic closure provides, as seen in Fig. \ref{Fig:1}, since subgrid stress evaluation is typically only a small fraction of the total computational cost of a simulation. However, for the implementation in Ref. \cite{king2016autonomic} the subgrid stress evaluation was  $O(10^4)$  more costly than for traditional prescribed closure models. This cost must be reduced by several orders of magnitude, as is done here, to make autonomic closure practical for LES.  

The cost of autonomic closure can be controlled by varying the number of degrees of freedom in the underlying nonparametric relation and by other choices in its implementation, though these choices can affect the accuracy of the resulting   $\tau_{ij}$   and   $P(\mathbf{x},t)$  fields. Therefore the main issue addressed here is whether there are implementations of autonomic closure that are efficient enough to be practical for LES while retaining the accuracy in   $\tau_{ij}$   and   $P(\mathbf{x},t)$   seen in Fig. \ref{Fig:1} from the computationally costly implementation in Ref. \cite{king2016autonomic}. To address this we present results from   $a priori$   tests that quantify the effects of various implementation choices in autonomic closure. In particular, we compare autonomically determined    $\tau_{ij}$   and   $P(\mathbf{x},t)$    fields with corresponding true subgrid stress and subgrid production fields to find implementations that are both efficient and accurate.  

Of key interest is whether large forward and backward scatter in the production fields from such efficient implementations of autonomic closure remain at the right magnitudes in regions at the right locations and having the right sizes and shapes. To evaluate this we develop and apply metrics that quantify how the resulting spatial support on which large production values are concentrated compares with corresponding true  $P(\mathbf{x},t)$  fields. From this we identify highly efficient implementations of autonomic closure that remain nearly as accurate as that in Ref. \cite{king2016autonomic} but at computational costs that are $O(10^3)$ smaller. These implementations are accurate and efficient enough for practical use in large eddy simulations, allowing future $a posteriori$ testing of this new closure methodology. 

%
%    FIGURE     % 
%%%%%%%%%%%%%%%%%
\begin{figure}
	\begin{center}
	\includegraphics[width=\maxwidth{\textwidth}]{Fig2.png}
	\caption{ Stencils $\mathbf{\widehat{S}}$ and $\mathbf{\widetilde{S}}$ each centered on point-of-interest $\mathbf{x}$ (red dot); (a) test-grid stencil $\mathbf{\widehat{S}}$ on which $\mathbf{h}_{ij}$ is determined, (b) LES-grid stencil $\mathbf{\widetilde{S}}$ on which resulting  $\mathbf{h}_{ij}$ is used to evaluate  $\tau_{ij}^{\mathcal{F}}(\mathbf{x})$  }
	\label{Fig:2}
	\end{center}
\end{figure}
%%%%%%%%%%%%%%%%%
%
%

\Cref{ch:3} provides a detailed description of autonomic closure and various implementation choices within it. \Cref{ch:4} presents the metrics used here for assessing the accuracy of any implementation of autonomic closure or of any traditional prescribed closure model. These include metrics for quantifying the scale-dependent support-density fields on which large magnitudes in   $P(\mathbf{x},t)$  are concentrated. \Cref{ch:5} then applies these metrics to quantify the accuracy and efficiency of various implementations of autonomic closure, and identifies the implementation that provides the best performance. \Cref{ch:6} compares results from this recommended implementation of autonomic closure and traditional prescribed closure models. In \cref{ch:7} we present major conclusions and discuss implications of these results for achieving accuracy across essentially all resolved scales in multiphysics large eddy simulations. 

%%%%%%%%%%%%%%%%%%%%%%%%%%%%%%%%%%%%%%%%%%%%%%%%%%%%%%%%%%%%%%%%%%%%%%%%%%%%%%%%%%%%%%%%%%
                %   \include rather than \input for chapters
% %\graphicspath{ {./Ch3/}  } 
\DeclareGraphicsExtensions{.png,.pdf,.jpg}

%%%%%%%%%%%%%%%%%%%%%%%%%%%%%%%%%%%%%%%%%%%%%%%%%%%%%%%%%%%%%%%%%%%%%%%%%%%%%%%%%%%%%%%%%%
%
%  CITE GROUPS  %
%%%%%%%%%%%%%%%%%

% % Test stress [27-29,39,40,46-54]
% \newcommand{\teststress}{meneveau2000scale, liu1994properties, sagaut2006large, germano1992turbulence, pope2000, germano1991dynamic, lilly1992proposed, moin1991dynamic, piomelli1993high, zang1993dynamic, akhavan2000subgrid, liu1995experimental,liu1999evolution,sarghini1999scale}


% % Strain-rotation basis [67,68,72-74]
% \newcommand{\strainrotation}{ling2016reynolds, ling2016machine, pope1975more, oberlack1997invariant, razafindralandy2007analysis}

% % Machine learning [67,68]
% \newcommand{\machinelearning}{ling2016reynolds, ling2016machine}

% % Volterra [75]
% \newcommand{\volterra}{schetzen1980volterra}

%%%%%%%%%%%%%%%%%
%

%%%%%%%%%%%%%%%%%%%%%%%%%%%%%%%%%%%%%%%%%%%%%%%%%%%%%%%%%%%%%%%%%%%%%%%%%%%%%%%%%%%%%%%%%%
\chapter{Autonomic Closure}
\label{ch:3}

This section presents a description of the autonomic closure methodology \cite{\autonomic} and the implementation choices within it that affect its accuracy and computational cost. 
 
%%%%%%%%%%%%%%%%%%%%%%%%%%%%%%%%%%%%%%%%%%%%%%%%%%%%%%%%%%%%%%%%%%%%%%%%%%%%%%%%%%%%%%%%%%
\section{The Autonomic Closure Methodology} 
\label{sec:IIA}

Traditional prescribed closure models represent a subgrid term in a predefined way in terms of specified parameters, such as the resolved strain rate $\widetilde{S}_{ij} $, based on theoretical or other considerations, often with one or more model constants allowed to vary locally via a dynamical procedure \cite{\prescribed}. 
In contrast, autonomic closure as proposed in Refs. \cite{\autonomic} is based on a highly generalized nonparametric representation of subgrid terms using only the primitive variables in a simulation. 
This generalized nonparametric representation typically has $O(10^2-10^4)$ degrees of freedom that are determined dynamically at every point $\mathbf{x}$  and time $t$ in the simulation. 
Such a highly generalized representation in the primitive-variable values on a local set $\mathbf{S}$ of stencil points around each space-time point $(\mathbf{x},t)$ removes the need for a predefined parametric model. 
Doing so allows far greater adaptability of the subgrid closure to the local turbulence state than is possible even with dynamical forms of traditional prescribed closure models. 

Here we apply autonomic closure to the subgrid stress  $\tau_{ij}(\mathbf{x},t)$ . Although the methodology can be generalized to multi-time stencils \cite{\autonomic}, here we consider a time-local implementation. An underlying general nonparametric representation  $F_{ij}$ for the local subgrid stress $\tau_{ij}$  can then be expressed in the primitive variables $\widetilde{\mathbf{u}}$  and  ${\widetilde{p}}$  as
%
%    EQUATION   %  
%%%%%%%%%%%%%%%%%
\begin{equation}
	\label{E:5}
	\tau_{ij}(\mathbf{x}) \approx  \tau_{ij}^{F}(\mathbf{x})
	\equiv F_{ij} \big[ \widetilde{\mathbf{u}}(x + x'), \ \widetilde{p}\mathbf{(x + x')}
	\ \forall \ \mathbf{x'} \in \widetilde{\mathbf{S}} \big], 
\end{equation}
%%%%%%%%%%%%%%%%%
%
%                             
where ${\widetilde{\mathbf{S}}}$  is a set of points $\mathbf{x'}$  that define a stencil (here $3\times3\times3$ ) with separation $\widetilde{\Delta}$   on the LES grid, as shown in \Cref{F:2}. \Cref{E:5} is a general nonparametric relation between the local subgrid stresses $\tau_{ij} \equiv \widetilde{u_iu_j} - \widetilde{u_i} \widetilde{u_j}$  and the resolved-scale variables $\widetilde{\mathbf{u}}$  and ${\widetilde{p}}$  on the LES grid.  $F_{ij}$ in (\ref{E:5}) should ideally reflect the local turbulence state at $\mathbf{x}$, including nonlinear, nonlocal, nonequilibrium and other effects that determine how the local $\tau_{ij}(\mathbf{x})$  can be best obtained from the resolved-scale values   $\widetilde{\mathbf{u}}$  and  ${\widetilde{p}}$   at the stencil points ${\widetilde{\mathbf{S}}}$   centered on $\mathbf{x}$.

Analogous to the subgrid stresses  $\tau_{ij}$, we consider local test stresses  $T_{ij} \equiv \widehat{\widetilde{u_i} \widetilde{u_j}} - \widehat{\widetilde{u_i}}\widehat{\widetilde{u_j}}$ \cite{\teststress}, which can be obtained from the resolved velocities  $\widetilde{\mathbf{u}} $ by applying a test filter $\widehat{(\  )}$  having a larger length scale  $\widehat{\Delta} = \alpha \widetilde{\Delta}$. If $F_{ij}$  reflects the local turbulence state at $\mathbf{x}$ then, in the same way that  $\tau_{ij}$ is related to the resolved-scale variables   $\widetilde{\mathbf{u}}$  and  ${\widetilde{p}}$  on the set  ${\widetilde{\mathbf{S}}}$, so also should $T_{ij}$  be related to the test-scale variables   $\widehat{\widetilde{\mathbf{u}}}$  and  $\widehat{{\widetilde{p}}}$   on the corresponding set ${\widehat{\mathbf{S}}}$,  of test-scale stencil points centered on $\mathbf{x}$  with separation $\widehat{\Delta}$. In other words, analogous to (\ref{E:5}), we have
%
%    EQUATION   %  
%%%%%%%%%%%%%%%%%
\begin{equation}
	\label{E:6}
	T_{ij}(\mathbf{x}) \approx  T_{ij}^{F}(\mathbf{x})
	\equiv F_{ij} \big[ \widehat{\widetilde{\mathbf{u}}}(x + x'), \ \widehat{\widetilde{p}}\mathbf{(x + x')}
	\ \forall \ \mathbf{x'} \in \widehat{\mathbf{S}} \big], 
\end{equation}
%%%%%%%%%%%%%%%%%
%
%         
where the stencil  ${\widehat{\mathbf{S}}}$ is the same as ${\widetilde{\mathbf{S}}}$  but is defined on the test-scale grid, as shown in \Cref{F:2}. 

The central idea in autonomic closure is that at each point $\mathbf{x}$ and time $t$, the known test stress value  $T_{ij}(\mathbf{x})$ and the known surrounding test-scale variables   $\widehat{\widetilde{\mathbf{u}}}$  and  $\widehat{{\widetilde{p}}}$   on the stencil  ${\widehat{\mathbf{S}}}$ in (\ref{E:6}) can be used to obtain information about the local form of $F_{ij}(\mathbf{x})$. Repeating this at multiple training points within a bounding box centered on $\mathbf{x}$ allows  $F_{ij}(\mathbf{x})$ to be determined sufficiently accurately that it can be generalized, in a machine learning sense, from the test scale to the LES scale. The resulting  $F_{ij}(\mathbf{x})$ is then used to determine $\tau_{ij}(\mathbf{x})$  via (\ref{E:5}) from the surrounding variables   $\widetilde{\mathbf{u}}$  and  ${\widetilde{p}}$   on the stencil  $\widetilde{\mathbf{S}}$ centered on $\mathbf{x}$. The relation   in (\ref{E:5}) and (\ref{E:6}) has a sufficient number of degrees of freedom that it is free to adapt to the local turbulence state at $\mathbf{x}$  to make  $T_{ij}^{\mathcal{F}} \approx T_{ij}$, and thereby make $\tau_{ij}^{\mathcal{F}} \approx \tau_{ij}$.

%
%    FIGURE     % 
%%%%%%%%%%%%%%%%%
\begin{figure}
	\begin{center}
	\includegraphics[width=\maxwidth{\textwidth}]{Fig2.png}
	\caption{ Stencils $\mathbf{\widehat{S}}$ and $\mathbf{\widetilde{S}}$ each centered on point-of-interest $\mathbf{x}$ (red dot); (a) test-grid stencil $\mathbf{\widehat{S}}$ on which $\mathbf{h}_{ij}$ is determined, (b) LES-grid stencil $\mathbf{\widetilde{S}}$ on which resulting  $\mathbf{h}_{ij}$ is used to evaluate  $\tau_{ij}^{\mathcal{F}}(\mathbf{x})$  }
	\label{F:2}
	\end{center}
\end{figure}
%%%%%%%%%%%%%%%%%
%
%
\vspace{-12pt}

It is in this way that autonomic closure accesses internal training data within the simulation, namely the test stresses within a bounding box centered on $\mathbf{x}$, to discover the local connection  $F_{ij}(\mathbf{x})$  between the local test stress $T_{ij}(\mathbf{x})$   and the local test-scale primitive variables   $\widehat{\widetilde{\mathbf{u}}}$  and  $\widehat{{\widetilde{p}}}$. It then uses this $F_{ij}(\mathbf{x})$  to evaluate the local subgrid stress $\tau_{ij}(\mathbf{x})$  from the local resolved-scale primitive variables   $\widetilde{\mathbf{u}}$  and  ${\widetilde{p}}$. In effect the simulation itself provides the training data, which are used to solve the local, nonlinear, nonparametric system identification problem that discovers the local connection between the subgrid stress and the resolved-scale primitive variables. 

Galilean invariance is enforced by subtracting the velocity at the stencil center point from the velocities on the stencil. However $F_{ij}$  in (\ref{E:5}) and (\ref{E:6}) does not readily lend itself to explicitly imposing the tensor invariance or realizability properties of  $\tau_{ij}$. For example, formulating  $F_{ij}$ in the tensor integrity basis \cite{\strainrotation} for the strain rate and rotation rate tensors, $\widetilde{S}_{ij}$  and $\widetilde{R}_{ij}$, which is the basis for some prescribed subgrid stress models, presumes that $\tau_{ij}$  depends only on combinations of these two tensors, whereas the nonparametric formulation in (\ref{E:5}) in the primitive variables   $\widetilde{\mathbf{u}}$  and  ${\widetilde{p}}$   makes no such assumption. Instead, as is common in many machine learning methods \cite{\machinelearning}, the tensor invariance and realizability properties are inherent in the training data $T_{ij}$, which informs the learned $F_{ij}$  in (\ref{E:6}) and thereby implicitly communicates these properties to $\tau_{ij}$  in (\ref{E:5}). Results in \Cref{ch:5} show that $\tau_{ij}(\mathbf{x},t)$  and $P_{ij}(\mathbf{x},t)$   from this autonomic closure methodology are far more accurate than those from common traditional prescribed closures that explicitly enforce tensor invariance properties.

Although any sufficiently general form for  $F_{ij}$ could be used in (\ref{E:5}) and (\ref{E:6}), we here choose a series \cite{\volterra} in $\mathbf{u}$  and  $p$, as in Refs. \cite{\autonomic}, namely the sum of all products of all orders of all variables at all points on the stencil, including all possible multi-point multi-variable products at each order. Such a representation is highly general. Even if truncated after second order with a $3\times3\times3$  stencil,  $F_{ij}$ for each $ij$ consists of $N = 5995$ zeroth- , first-, and second-order products of $\mathbf{u}$  and  $p$ in (\ref{E:5}) and (\ref{E:6}), each having a separate coefficient  ${h_{ij}^{(k)}}$ with $k = {1,\ldots,N}$  , where $N$ is the number of degrees of freedom in $F_{ij}$ . This set of coefficients for each $ij$ is thus an $N$-length column vector denoted $\mathbf{h}_{ij}$. To gain computational efficiency, $F_{ij}$  could be truncated at lower orders, or restricted to single-point products, or limited only to the velocities $\mathbf{u}$ on the stencil. Even if $F_{ij}$  is truncated after first order and limited to single-point velocities on a  $3\times3\times3$  stencil, it still contains $N=82$ first-order terms in the $\mathbf{u}$ components for each $ij$, and thus has far more degrees of freedom than do dynamic versions of traditional prescribed closure models.  

With such a series for $F_{ij}$, the stress value $\tau_{ij}$   at the stencil center point $\mathbf{x}$ can be written from (\ref{E:5}) as
%
%    EQUATION   %  
%%%%%%%%%%%%%%%%%
\begin{equation}
	\label{E:7}
	\tau_{ij}(\mathbf{x}) \approx  \tau_{ij}^{F}(\mathbf{x})
	= \mathbf{\widetilde{V}h_{ij}}, 
\end{equation}
%%%%%%%%%%%%%%%%%
%
%         
where $\widetilde{\mathbf{V}}$  is the $N$-length array containing the known values of all products of all orders of   $\widetilde{\mathbf{u}}$  and $\widetilde{p}$   in $F_{ij}$  at all points $\mathbf{x} +\mathbf{x'} $  on the stencil $\widetilde{\mathbf{S}}$  . If $\mathbf{h}_{ij}$  is known then the value of  $\tau_{ij}$ at the stencil center point $\mathbf{x}$ can be obtained from (\ref{E:7}). To determine  $\mathbf{h}_{ij}$, (\ref{E:6}) can be similarly written as $T_{ij} \approx T_{ij}^{F} = \mathbf{\widehat{V}h_{ij}}$, where $\mathbf{\widehat{V}}$  is the $N$-length array containing the known values of all products of all orders of the test-filtered primitive variables  $\widehat{\widetilde{\mathbf{u}}}$  and  $\widehat{{\widetilde{p}}}$   in  $F_{ij}$ at all points $\mathbf{x} +\mathbf{x'} $  on the stencil $\widehat{\mathbf{S}}$, and where the test stress value $T_{ij}$   at the stencil center point $\mathbf{x}$ is known. Repeating this with the stencil ${\widehat{\mathbf{S}}}$  centered at each of $M$ training points within a local bounding box around the point $\mathbf{x}$, in which variations in the turbulence state embodied in  $F_{ij}$ are taken to be negligible, then $\mathbf{\widehat{V}}$   becomes an  $M \times N$ matrix. With $\mathbf{T}_{ij}$  denoting the corresponding $M$-length column vector consisting of the known  $T_{ij}$ values at the $M$ training points, we then have for each $ij$
%
%    EQUATION   %  
%%%%%%%%%%%%%%%%%
\begin{equation}
	\label{E:8}
	T_{ij}(\mathbf{x}) \approx  T_{ij}^{F}(\mathbf{x})
	= \mathbf{\widehat{V}h_{ij}}.
\end{equation}
%%%%%%%%%%%%%%%%%
%
%      
The size of the bounding box, typically extending along homogeneous directions, determines the maximum number of available training points within it. The chosen number $M$ of training points and the bounding box volume $V_B$  determine the relative training point spacing $(V_B/M)^{1/3}$  , which together with $M$ determines how much effectively independent information is being used to characterize the local turbulence state via $F_{ij}$, or equivalently via $\mathbf{h}_{ij}$, within the bounding box around the point $\mathbf{x}$. The choice of bounding box size and the number of training points $M$ are part of any implementation of autonomic closure. Regardless of the $M$ and $N$ values, since the vector  $\mathbf{T}_{ij}$ and the matrix $\widehat{V}$   are known, the system in (\ref{E:8}) may be solved by any number of means. Here we use a damped least-squares solution \cite{king2016autonomic} of the form
%
%    EQUATION   %  
%%%%%%%%%%%%%%%%%
\begin{equation}
	\label{E:9}
	\mathbf{h}_{ij} 
	= \bigg( \mathbf{\widehat{V}}^T \mathbf{\widehat{V}} + \mathbf{\lambda I} \bigg)^{-1} 
	\mathbf{\widehat{V}}^T \mathbf{T_{ij}}.
\end{equation}
%%%%%%%%%%%%%%%%%
%
%      
where  $\lambda$ is the damping coefficient. When $M/N \gg 1$  the value of  $\lambda$ is unimportant and is set to $\lambda = 10^{-3}$ ; when $M/N \leq O(1)$  then $\lambda$   is set to $\lambda = 10^{-1}$. Once the coefficients $\mathbf{h}_{ij}$  at $\mathbf{x}$ have been determined via (\ref{E:9}), they are used in (\ref{E:7}) to evaluate   at $\mathbf{x}$. 

Note the resulting local $\mathbf{h}_{ij}$   is used only once to evaluate $\tau_{ij}(\mathbf{x},t)$   at the bounding box center point $\mathbf{x}$ for the current time $t$. A new set of coefficients  $\mathbf{h}_{ij}$ is obtained for each $(\mathbf{x},t)$  in the simulation. As a result, autonomic closure does not provide a fixed set of coefficients $\mathbf{h}_{ij}$  and thus does not provide a closure model for $\tau_{ij}$. Instead, it is the autonomic methodology itself that is the closure for $\tau_{ij}(\mathbf{x},t)$.

%%%%%%%%%%%%%%%%%%%%%%%%%%%%%%%%%%%%%%%%%%%%%%%%%%%%%%%%%%%%%%%%%%%%%%%%%%%%%%%%%%%%%%%%%%
\section{Implementations of Autonomic Closure }
\label{sec:IIB}

Implementing autonomic closure involves choices of $F_{ij}$, $N$, $M$, and $V_B$ that impact the generalizability of $\mathbf{h}_{ij}$, in a machine learning sense, from the test scale to the LES scale, and also determine the computational cost of this closure methodology. For example, the implementation in Ref. \cite{king2016autonomic} used a second-order, non-colocated, velocity-pressure  series for $F_{ij}$, which provided a large number ($N = 5995$) of coefficients $\mathbf{h}_{ij}$  in $F_{ij}$, and also used the largest possible bounding box, which allowed a large number ($M  = 15,625$) of training points to determine $\mathbf{h}_{ij}$. Results from that implementation verified that autonomic closure produces subgrid stress fields $\tau_{ij}^{F}(\mathbf{x},t)$   and subgrid production fields  $P^{F}(\mathbf{x},t) \equiv -\tau_{ij}^{F}\widetilde{S}_{ij}$ that represent the true $\tau_{ij}(\mathbf{x},t)$  and  $P(\mathbf{x},t)$   fields over essentially all resolved scales far more accurately than do existing prescribed closure models, such as the dynamic Smagorinsky model \cite{king2016autonomic}. 
However while the implementation in Ref. \cite{king2016autonomic} is accurate, as can be seen in Fig. 1, it is too computationally costly in comparison with traditional prescribed closure models to serve as a widely-usable alternative closure for practical applications of LES. Therefore, in the following sections we evaluate the effects of various implementation choices on the accuracy and computational cost of autonomic closure. Specifically, we consider specific combinations of the following key implementation choices: 

\begin{itemize}
	\item local vs. nonlocal forms based on the bounding box volume  $V_B = (n\widehat{\Delta})^3$ 
	\item velocity-only vs. velocity-pressure  series representations $F_{ij}$
	\item colocated vs. non-colocated products on the stencils $\widetilde{\mathbf{S}}$  and  $\widehat{\mathbf{S}}$ 
	\item first-order vs. second-order  series representations $F_{ij}$
	\item varying numbers $M$ of training points relative to the number $N$ of coefficients $\mathbf{h}_{ij}$  
	\item varying training point spacing  $(V_B/M)^{1/3}$
\end{itemize}

These allow implementations ranging from a large, second-order, velocity-pressure, non-colocated, nonlocal formulation with $M/N \gg 1$, which provides a large number $N$ of degrees of freedom in $F_{ij}$  but is computational costly, to a minimal first-order, velocity-only, colocated, local formulation with $M/N \ll 1$  , which has the lowest computational cost but can be expected to be less accurate. The main objective is to determine which implementations of autonomic closure provide high accuracy in  $\tau_{ij}(\mathbf{x},t)$ and $P(\mathbf{x},t)$   at acceptable computational cost.

For instance, because velocities at distant points affect local pressure values on the stencil, a velocity-pressure implementation adds nonlocality beyond the stencil points in  $F_{ij}$ and also increases its number of degrees of freedom $N$. Both effects can increase generalizability of the resulting $\mathbf{h}_{ij}$  from the test scale to the LES scale, but at increased computational cost over a velocity-only formulation. Similarly, including non-colocated products of the stencil-point variables or increasing the truncation order of the series for $F_{ij}$  increase $N$ and introduce additional physical information in $F_{ij}$, but at an increased computational cost that may not be merited. Larger numbers $M$ of training points should also lead to increased accuracy, but for any bounding box volume $V_B$  the spacing $(V_B/M)^{1/3}$  between training points becomes smaller as $M$ increases, thus providing relatively less additional independent training information to determine $F_{ij}$  while increasing the computational cost. These expectations suggest a non-trivial tradeoff between $N$, $M$, and  $V_B$.  
It can also be expected that large bounding box volumes $V_B$, which allow larger numbers $M$ of widely-spaced (and thus more independent) training points, will provide greater accuracy. However, large bounding boxes may contain substantially different turbulence states, and thus the training points will lead to coefficients  $\mathbf{h}_{ij}$ that pertain, in part, to turbulence states other than that at the point of interest $\mathbf{x}$. Smaller bounding boxes give a more-local implementation that ensures training points are relevant to the local turbulence state at $\mathbf{x}$, but inherently limit the number of available training points and their relative independence. This suggests that the most accurate implementation may involve a tradeoff between increased locality via smaller bounding boxes and increased training information via larger bounding boxes. 
In the following sections, we develop and apply quantitative assessment metrics to determine implementations of autonomic closure that provide high fidelity in the resulting  $\tau_{ij}(\mathbf{x},t)$   fields, and particularly in the corresponding $P(\mathbf{x},t)$   fields, yet do so at computational costs that are sufficiently low to enable practical use of autonomic closure in large eddy simulations. 
% etc. 
%                                         % Heading commands (in descending order): 
%                                         % \chapter
%                                         % \section
%                                         % \subsection
%                                         % \subsubsection
%                                         % \paragraph
%                                         % \subparagraph
% % \iftoggle{sample}{%
% %  \include{sample_chapter/sample_chapter}%
% % }{}
                                        
%%%%%%%%%%%%%%%%%%%%%%%%%%%%%%%%%%%%%%%
% Back matter
%%%%%%%%%%%%%%%%%%%%%%%%%%%%%%%%%%%%%%%
\SingleSpacing                          % Back matter should be single spaced 

\edef\defaulttolerance{\the\tolerance}
\tolerance 500                         % Increase tolerance to prevent material extending into margins
\hbadness 500  

\iftoggle{useendnotes}{%                % If you're using endnotes, output them here
  \setsecnumdepth{none}                 % No section numbering in end notes
  \printpagenotes                       
  \setsecnumdepth{all}%                 % Turn section numbering back on after printing
}{}

\chapter*{\bibheading}                  % In the running text, use a chapter-level heading 
                                        % for the bibliography section 
\phantomsection                         
\addcontentsline{toc}{chapter}{%        % In the TOC, add a custom chapter-level heading 
  \hspace{-\cftchapterindent}%          % that will be flush against the left margin 
  \bibheading%
}    
\phantomsection
\addtocontents{toc}%                    % Add this 'mark' to TOC so subsequent pages use
  {\protect\markboth{\bibheading}{Page}}%   the bibliography heading (unlikely since 
                                        %   the appendices follow quickly) 
\iftoggle{usebiblatex}{%                % Output the bibliography
  \printbibliography[heading=none]      % Using a 'biblatex' package; do not let 
                                        %   'biblatex' output a heading 
}{%                                     
  \renewcommand\bibsection{}            % Do not let 'natbib' output a heading 
  \bibliographystyle{\natbibstyle}      % Using 'natbib' to print bibliography 
  \bibliography{\bibfilename}
}

\appendix                               % Indicate start of appendices
                                        % Appendices are considered 'mainmatter' in this
                                        %   documentclass
\tolerance \defaulttolerance            % Set tolerance back to default
\hbadness \defaulttolerance  

\addtocontents{toc}{\protect%           % Only include appendix title in table of contents
  \setcounter{tocdepth}{0}}%            %   and omit sub-headings
\renewcommand*{\chapnamefont}%          % Reset font for 'Appendix' in chapter titles
    {\normalfont\MakeTextUppercase}                                        
\makeatletter                           % Clear page after printing appendix title
  \renewcommand{\memendofchapterhook}%
  {%
    \clearpage
    \m@mindentafterchapter
    \@afterheading
  }
\makeatother

\phantomsection                         % Need '\phantomsection' to place hyperref 
                                        %   bookmark more accurately
\addcontentsline{toc}{part}{Appendix}   %~Add "Appendix" to TOC here; comment out this 
                                        %   line if you're not including appendices

%\phantomsection                        %!This is the one part of the template that I 
%\addtocontents{toc}%                   %   could not get to work properly. After you 
%  {\protect\markboth{APPENDIX}{Page}}  %   start listing appendices in the TOC, 
                                        %   subsequent TOC pages should use "APPENDIX in 
                                        %   the header instead of "CHAPTER"; however, 
                                        %   this code will make "APPENDIX" appear on the 
                                        %   the same page that the *first* appendix 
                                        %   appears on. This problem won't affect most 
                                        %   people, but if it affects you, uncomment 
                                        %   these lines and move them below where 
                                        %   the appendices are listed. Keep moving these
                                        %   lines down and checking the output until
                                        %   the TOC headers appear correctly 
                                        
\include{Appendix/appendix1}                     %~Insert your appendices here; I recommend to use
%\include{appendix2}                     %   \include rather than \input for appendices. 
%\include{appendix3}% etc.               %   All heading commands are the same as above,
                                         %   e.g., \chapter, \section, etc. 

%\iftoggle{sample}{%
%  \include{appendix}%
%}{}


\backmatter                             % Start back matter according to documentclass
\makeatletter                           % Do not clear page after printing title for
  \renewcommand{\memendofchapterhook}%  %   biographical sketch
  {%
    \m@mindentafterchapter
    \@afterheading
  }
\makeatother
%\chapter{Biographical Sketch}           %~Biographical Sketch is optional
%\input{biography}                       %<Enter the name of the .tex file containing your
                                        %   biography or omit this line and type in
                                        %   your biography here (1 paragraph) 

\end{document}