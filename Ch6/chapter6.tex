\graphicspath{ {./Ch6/}  } 
\DeclareGraphicsExtensions{.png,.pdf,.jpg}

%%%%%%%%%%%%%%%%%%%%%%%%%%%%%%%%%%%%%%%%%%%%%%%%%%%%%%%%%%%%%%%%%%%%%%%%%%%%%%%%%%%%%%%%%%
\chapter{Comparisons with Traditional Closure Models}

From the preceding sections, the most accurate and efficient implementations of autonomic closure are those based on relatively local, second-order, velocity-only, colocated formulations. Among these, Case 3a was seen in Figs. 6-12 and Table 1 to give the best results for subgrid stress fields   and subgrid production fields   across essentially all resolved scales. We now compare its performance with that of the dynamic Smagorinsky model [27,46-59] and the Bardina scale similarity model [43-45]. The former is the most widely used subgrid model for   in LES, and the latter is the basis for various mixed models [50-57] in which it is typically combined with the Smagorinsky model. 

Figure 13 shows typical results comparing the true subgrid stress and production fields   and   with   and   from this implementation of autonomic closure, and with   and   from the dynamic Smagorinsky (DS) model and   and   from the Bardina scale similarity (BD) model. All results are for the same spectrally sharp LES-scale and test-scale filters and the same test-to-LES filter scale ratio   as used throughout. It is apparent in Fig. 13 that the results from autonomic closure compare with the true subgrid stress and production fields far better than do the results from either implementation of these two traditional closure models. While the Bardina scale similarity model produces stress fields   that show some of the detailed features in the true   fields, when contracted with the resolved strain rate tensor in (4) the resulting subgrid production field   compares relatively poorly with the true production field  . In comparing subgrid production fields in Fig. 13, it is particularly noteworthy that the locations, sizes, and shapes of regions in which large magnitudes of subgrid production are clustered in   from autonomic closure agree far better with those in the true   field than do corresponding results from either of the traditional models. 

Figures 14a,b show comparisons of resulting probability densities of the subgrid stress and subgrid production from each of these subgrid closures. It is apparent that the statistical distributions from autonomic closure match the true distributions very closely.  Additionally, Figs. 14c,d show the support-density metrics   and   comparing the spatial support on which large positive and negative values of subgrid production are concentrated in Fig. 13. It is apparent in Fig. 14, and consistent with the visual comparisons in Fig. 13, that the spatial structure of the support-density on which large values of positive and negative subgrid production values are concentrated in   from autonomic closure agrees far better with the true subgrid production field   than do the corresponding results   and   from the dynamic Smagorinsky and Bardina scale similarity models.

Note in Table 1 that the average subgrid production   from autonomic closure closely matches the true value  , corresponding to average energy transfer from the resolved scales into the subgrid scales at a rate just slightly higher than  . By comparison,   from the Bardina scale similarity model is seen in Table 1 to have the opposite sign, and thus on average transfers energy into the resolved scales. This is consistent with the widely reported observation that the scale similarity model leads to unstable simulations unless it is coupled with an added dissipative model to ensure net average energy transfer out of the resolved scales. Table 1 shows that   from the dynamic Smagorinsky model has the correct sign and thus on average transfers energy out of the resolved scales. However, Fig. 13 shows that   compares very poorly to the true   field, with large positive and negative values of subgrid production being highly overrepresented, and with regions in which large forward and backward scatter are seen in   poorly matching those in the true field  . These factors likely contribute to the widely reported need for limiters, added dissipation, or other ad hoc treatments to keep the dynamic Smagorinsky model computationally stable. 

In contrast, autonomic closure produces   and   in Figs. 10 and 13 that closely match even detailed features in the true fields   and  , and thus also closely match the stress and production statistics in Figs. 11, 12 and 14, and lead to   that closely matches the true value   in Table 1. Collectively, these factors suggest it may be possible for this Case 3a implementation of autonomic closure to be stable without the need for limiters, added dissipation, or other ad hoc treatments when implemented in large eddy simulations, though this can only be assessed via future a posteriori tests. 

Note also in Table 1 that the computational time for this Case 3a implementation of autonomic closure is only about an order of magnitude larger than that needed to evaluate the dynamic Smagorinsky model or the Bardina scale similarity model. Since the computational time for subgrid stress evaluation is typically only a small fraction of the total computational time needed for LES, the Case 3a implementation of autonomic closure should be efficient enough for use in practical large eddy simulations.
