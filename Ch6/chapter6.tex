\graphicspath{ {./Ch6/} } 
\DeclareGraphicsExtensions{.png,.pdf,.jpg}

%%%%%%%%%%%%%%%%%%%%%%%%%%%%%%%%%%%%%%%%%%%%%%%%%%%%%%%%%%%%%%%%%%%%%%%%%%%%%%%%%%%%%%%%%%
%
%  CITE GROUPS  %
%%%%%%%%%%%%%%%%%

% % DSmag2 [27,46-59]
% \newcommand{\DSmagtwo}{meneveau2000scale, germano1991dynamic, lilly1992proposed, moin1991dynamic, piomelli1993high, zang1993dynamic, akhavan2000subgrid, liu1995experimental, liu1999evolution, sarghini1999scale, lund1993numerical, ghosal1995dynamic, gravemeier2006consistent, king2015autonomic, king2016autonomic}

% % Bardina [43-45]
% \newcommand{\Bardina}{bardina1980improved, bardina1983improved, horiuti1989role}


%%%%%%%%%%%%%%%%%
%

%%%%%%%%%%%%%%%%%%%%%%%%%%%%%%%%%%%%%%%%%%%%%%%%%%%%%%%%%%%%%%%%%%%%%%%%%%%%%%%%%%%%%%%%%%
\chapter{Comparisons with Traditional Closure Models}
\label{ch:6}

From the preceding sections, the most accurate and efficient implementations of autonomic closure are those based on relatively local, second-order, velocity-only, colocated formulations. Among these, Case 3a was seen in \Crefrange{F:6}{F:12} and \Cref{tab:1} to give the best results for subgrid stress fields $\tau_{ij}(\mathbf{x},t)$  and subgrid production fields $P(\mathbf{x},t) \equiv -\tau_{ij}\widetilde{S}_{ij}$  across essentially all resolved scales. We now compare its performance with that of the dynamic Smagorinsky model \cite{\DSmagtwo} and the Bardina scale similarity model \cite{\Bardina}. The former is the most widely used subgrid model for $\tau_{ij}(\mathbf{x},t)$  in LES, and the latter is the basis for various mixed models \cite{\mixed} in which it is typically combined with the Smagorinsky model. 

%
%    FIGURE     % 
%%%%%%%%%%%%%%%%%
\begin{figure}
	\centering \hspace{1.5cm}
	\subfloat{\includegraphics[width=5in]{y43/Fig13_tau11.eps}}
	\subcaption{Typical subgrid stress fields $\tau_{11}(\mathbf{x},t)$.} \label{F:13a}
\end{figure}
%
\begin{figure}
	\ContinuedFloat
	\centering \hspace{1.5cm}
	\subfloat{\includegraphics[width=5in]{y43/Fig13_tau12.eps}}
	\subcaption{Typical subgrid stress fields $\tau_{12}(\mathbf{x},t)$.} \label{F:13a2}
\end{figure}
%
\begin{figure}
\ContinuedFloat
	\centering \hspace{1.5cm}
	\subfloat{\includegraphics[width=5in]{y43/Fig13_tau13.eps}}
	\subcaption{Typical subgrid stress fields $\tau_{13}(\mathbf{x},t)$.} \label{F:13a3}
\end{figure}
%
\begin{figure}
\ContinuedFloat
	\centering \hspace{1.5cm}
	\subfloat{\includegraphics[width=5in]{y43/Fig13_tau22.eps}}
	\subcaption{Typical subgrid stress fields $\tau_{22}(\mathbf{x},t)$.} \label{F:13a4}
\end{figure}
%
\begin{figure}
\ContinuedFloat
	\centering \hspace{1.5cm}
	\subfloat{\includegraphics[width=5in]{y43/Fig13_tau23.eps}}
	\subcaption{Typical subgrid stress fields $\tau_{23}(\mathbf{x},t)$.} \label{F:13a5}
\end{figure}
%
\begin{figure}
\ContinuedFloat
	\centering \hspace{1.5cm}
	\subfloat{\includegraphics[width=5in]{y43/Fig13_tau33.eps}}
	\subcaption{Typical subgrid stress fields $\tau_{33}(\mathbf{x},t)$.} \label{F:13a6}
\end{figure}
%
\begin{figure}
	\ContinuedFloat
	\centering \hspace{1.5cm}
	\subfloat{\includegraphics[width=5in]{y43/Fig13_P.eps}}
	\subcaption{Corresponding subgrid production fields $P(\mathbf{x},t)$.}\label{F:13b}
\end{figure}
%
\begin{figure}
	\ContinuedFloat
	\caption{Typical comparison of autonomic closure and traditional prescribed closure models, showing (\Crefrange{F:13a}{F:13a6}) typical subgrid stress $\tau_{ij}(\mathbf{x},t)$ and (\Cref{F:13b}) subgrid production $P(\mathbf{x},t)$; (a) true fields, (b) results from autonomic closure (Case 3a), (c) results from dynamic Smagorinsky (DS) model, and (d) results from Bardina scale-similarity (BD) model; all are for same scale ratio $\alpha \equiv \Delta_{\Gamma} / \widetilde{\Delta} = 2$.}
	\label{F:13}
\end{figure}
%%%%%%%%%%%%%%%%%
%
%



\Cref{F:13} shows typical results comparing the true subgrid stress and production fields $\tau_{ij}(\mathbf{x},t)$  and  $P(\mathbf{x},t)$ with  $\tau_{ij}^{F}(\mathbf{x},t)$  and  $P^{F}(\mathbf{x},t)$   from this implementation of autonomic closure, and with  $\tau_{ij}^{DS}(\mathbf{x},t)$  and  $P^{DS}(\mathbf{x},t)$   from the dynamic Smagorinsky (DS) model and   $\tau_{ij}^{BD}(\mathbf{x},t)$  and  $P^{BD}(\mathbf{x},t)$  from the Bardina scale similarity (BD) model. All results are for the same spectrally sharp LES-scale and test-scale filters and the same test-to-LES filter scale ratio $\alpha \equiv \widehat{\Delta}/ \widetilde{\Delta} = 2$  as used throughout. It is apparent in \Cref{F:13} that the results from autonomic closure compare with the true subgrid stress and production fields far better than do the results from either implementation of these two traditional closure models. While the Bardina scale similarity model produces stress fields  $\tau_{ij}^{BD}(\mathbf{x},t)$  that show some of the detailed features in the true $\tau_{ij}(\mathbf{x},t)$ fields, when contracted with the resolved strain rate tensor in (\ref{E:4}) the resulting subgrid production field $P^{BD}(\mathbf{x},t)$ compares relatively poorly with the true production field $P(\mathbf{x},t)$. In comparing subgrid production fields in \Cref{F:13}, it is particularly noteworthy that the locations, sizes, and shapes of regions in which large magnitudes of subgrid production are clustered in $P^{F}(\mathbf{x},t)$  from autonomic closure agree far better with those in the true  $P(\mathbf{x},t)$ field than do corresponding results from either of the traditional models. 

%
%    FIGURE     % 
%%%%%%%%%%%%%%%%%
\begin{figure}
	\begin{center}
	\includegraphics[width=\maxwidth{5in}]{Fig14.pdf}
	\caption{ Comparison of autonomic closure with traditional prescribed closure models, showing (top) pdfs of typical subgrid stress component $\tau_{ij}$ and subgrid production $P$, and (bottom) $M_1$ and $M_2$ variation with scale ratio $\Delta_{\Gamma} / \widetilde{\Delta}$ for subgrid production support-density fields; true fields (black), autonomic closure (blue), dynamic Smagorinsky model (red dashed), Bardina scale-similarity model (red dotted); all are for same scale ratio $\alpha \equiv \Delta_{\Gamma} / \widetilde{\Delta} = 2$.}
	\label{F:14}
	\end{center}
\end{figure}
%%%%%%%%%%%%%%%%%
%
%

\Cref{F:14}a,b show comparisons of resulting probability densities of the subgrid stress and subgrid production from each of these subgrid closures. It is apparent that the statistical distributions from autonomic closure match the true distributions very closely.  Additionally, \Cref{F:14}c,d show the support-density metrics $M_1$  and $M_2$   comparing the spatial support on which large positive and negative values of subgrid production are concentrated in \Cref{F:13}. It is apparent in \Cref{F:14}, and consistent with the visual comparisons in \Cref{F:13}, that the spatial structure of the support-density on which large values of positive and negative subgrid production values are concentrated in  $P^{F}(\mathbf{x},t)$  from autonomic closure agrees far better with the true subgrid production field $P(\mathbf{x},t)$  than do the corresponding results $P^{DS}(\mathbf{x},t)$  and $P^{BD}(\mathbf{x},t)$ from the dynamic Smagorinsky and Bardina scale similarity models.

Note in \Cref{tab:1} that the average subgrid production $\langle P^F \rangle$  from autonomic closure closely matches the true value $\varepsilon$, corresponding to average energy transfer from the resolved scales into the subgrid scales at a rate just slightly higher than $\varepsilon$. By comparison,  $\langle P^{BD} \rangle$ from the Bardina scale similarity model is seen in \Cref{tab:1} to have the opposite sign, and thus on average transfers energy into the resolved scales. This is consistent with the widely reported observation that the scale similarity model leads to unstable simulations unless it is coupled with an added dissipative model to ensure net average energy transfer out of the resolved scales. \Cref{tab:1} shows that $\langle P^{DS} \rangle$  from the dynamic Smagorinsky model has the correct sign and thus on average transfers energy out of the resolved scales. However, \Cref{F:13} shows that $P^{DS}(\mathbf{x},t)$  compares very poorly to the true  $P(\mathbf{x},t)$ field, with large positive and negative values of subgrid production being highly overrepresented, and with regions in which large forward and backward scatter are seen in  $P^{DS}(\mathbf{x},t)$ poorly matching those in the true field $P(\mathbf{x},t)$. These factors likely contribute to the widely reported need for limiters, added dissipation, or other \textit{ad hoc} treatments to keep the dynamic Smagorinsky model computationally stable. 

In contrast, autonomic closure produces $\tau_{ij}^F(\mathbf{x},t)$ and  $P^F(\mathbf{x},t)$ in \Cref{F:10,F:13} that closely match even detailed features in the true fields $\tau_{ij}(\mathbf{x},t)$ and  $P(\mathbf{x},t)$, and thus also closely match the stress and production statistics in \Cref{F:11,F:12,F:14}, and lead to $\langle P^F \rangle$  that closely matches the true value $\varepsilon$ in \Cref{tab:1}. Collectively, these factors suggest it may be possible for this Case 3a implementation of autonomic closure to be stable without the need for limiters, added dissipation, or other \textit{ad hoc} treatments when implemented in large eddy simulations, though this can only be assessed via future $a posteriori$ tests. 

Note also in \Cref{tab:1} that the computational time for this Case 3a implementation of autonomic closure is only about an order of magnitude larger than that needed to evaluate the dynamic Smagorinsky model or the Bardina scale similarity model. Since the computational time for subgrid stress evaluation is typically only a small fraction of the total computational time needed for LES, the Case 3a implementation of autonomic closure should be efficient enough for use in practical large eddy simulations.
